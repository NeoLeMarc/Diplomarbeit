\null\newpage
\Huge\textbf{Abstract}\normalsize\\\\
\vspace{-0.5cm}

\begin{small}
Bei einem Massenanfall von Verletzten (MANV) kommt es oft zu personellen Engpässen, so dass eine
ausreichende Überwachung der Patienten nicht gewährleistet ist. 
Abhilfe kann der am IBT entwickelte \emph{Erste-Hilfe-Sensor} schaffen, der mit einem neuartigen 
Verfahren Puls und Atmung eines einzelnen Patienten überwacht. Zur Überwachung \emph{mehrerer} Patienten 
muss ein kabelloses Netzwerk aus diesen Sensoren gebildet werden.
Entsprechende Netzwerktechnologien existieren zwar, müssen aber in die bestehende Lösung integriert 
und um eine geeignete Überwachungssoftware ergänzt werden. Dies ist Aufgabe der vorliegenden Arbeit.
Hierzu wird zunächst eine Untersuchung aller in Frage kommenden Funktechnologien unter besonderem 
Augenmerk auf Energieeffizienz und Stabilität vorgenommen.
\emph{ZigBee} stellt sich als bestgeeignetes Protokoll heraus, und es wird entschieden,
eine Umsetzung auf Basis eines Moduls des Herstellers \emph{Atmel} durchzuführen. 
Bei dessen genauerer Betrachtung werden einige Schwierigkeiten 
identifiziert\,--\,beispielsweise das Auftreten von asynchronen Ereignissen, das zum 
Synchronisationsverlust führen kann.
Lösungen für diese Schwierigkeiten werden gefunden und an Überwachungssoftware sowie \emph{Erste-Hilfe-Sensor} 
konkret umgesetzt.
\\                    
Zur Gewährleistung einer hohen Flexibilität wird ein komponentenbasierter Ansatz gewählt. 
Hierdurch wird nicht nur die Austauschbarkeit der einzelnen Komponenten\,--\,z.B. der eingesetzten
Funktechnologie\,--\,erreicht, sondern auch deren Verteilung über mehrere Rechner ermöglicht. 
Auf diesem Weg wird sowohl die Skalierbarkeit erhöht als auch eine Grundlage für verbesserte
Ausfallsicherheit geschaffen.
Die Verbindung der einzelnen Komponenten wird über \emph{Corba}-Schnittstellen realisiert. 
Diese sind so gestaltet, dass auch externe Software angebunden werden kann. Exemplarisch gezeigt wird dies
am Beispiel einer \emph{Weboberfläche}. Damit wird gleichzeitig die Verwendung von Mobiltelefonen
als Endgeräte zur Patientenüberwachung ermöglicht, ohne dass eine Installation von Software auf diesen Geräten
notwendig ist.
Darüber hinaus wird eine Platine entwickelt, auf der das \emph{ZigBee}-Modul und der
\emph{Mikrocontroller} des \emph{Erste-Hilfe-Sensors} integriert sind. Sie dient als Werkzeug
zur Firmware-Entwicklung für den erweiterten \emph{Erste-Hilfe-Sensor} und ist gleichzeitig auch Testumgebung.
\\
Bei einer eingehenden Evaluierung des Gesamtsystems wird die Interoperabilität aller Komponenten gezeigt. 
Außerdem wird die Leistungsaufnahme des \emph{ZigBee}-Moduls in Abhängigkeit von den Parametern des 
Energiesparmodus untersucht. Insgesamt zeigt sich, dass Laufzeit, Reichweite und die erreichbare Anzahl an 
Sensoren für MANV-Szenarien mit hunderten von Verletzten ausreichend sind.
%\begin{small}\,--\,Falls es nicht auf eine Seite passt darf die Schrift im Abstract kleiner gemacht werden, aber nur beim Abstract!
%... Die Zusammenfassung ("`Abstract"') ist nach dem Titel der zweitwichtigste Bestandteil einer
%wissenschaftlichen Arbeit. Sie sollten deshalb für die Zusammenfassung, ebenso wie für den
%Titel, besonders viel Mühe und Zeit verwenden, da die gesamte wissenschaftliche Arbeit nur
%von sehr wenigen Wissenschaftlern gelesen wird, die Zusammenfassung aber von vielen. Aus
%der Zusammenfassung muß hervorgehen, wovon die Arbeit handelt, worauf sie aufbaut, und
%vor allen Dingen, welche neuen Erkenntnisse gewonnen wurden. Die wichtigsten Ergebnisse
%der Arbeit müssen kurz und präzise aufgezählt werden. Es genügt nicht zu schreiben, daß dies
%und jenes in der Arbeit behandelt werden. Wichtig ist, daß die "`harten Fakten"', welche sich
%aus den Untersuchungen ergeben haben, aufgelistet sind.
%Handelt es sich um eine theoretische Arbeit, dann müssen Sie erwähnen, von welchen
%Gleichungen Sie ausgegangen sind und welche Näherungen Sie verwendet haben; bei einer
%experimentellen Arbeit müssen Sie erwähnen, welche Experimente Sie durchgeführt haben
%und eventuell auch, welche Auswertemethoden (falls nicht Standardmethoden) Sie verwendet
%haben.Beachten Sie, daß Ihre Arbeit von Wissenschaftlern unterschiedlicher Herkunft und
%Ausbildung gelesen wird. Bedenken Sie, daß sich auch Wissenschaftler für Ihre Arbeit
%interessieren können, die aus benachbarten Disziplinen stammen und nicht mit dem von Ihnen
%verwendeten wissenschaftlichen "`Jargon"' vertraut sind, oder solche, welche die in Ihrem Fach
%üblichen Abkürzungen nicht kennen. Deshalb soll die Zusammenfassung für alle
%(natur-)wissenschaftlich gebildeten Leser verständlich sein. Das bedingt, daß eventuell
%benutzte Abkürzungen erklärt werden müssen, und daß nur solche Begriffe vorkommen
%dürfen, die ein "`normaler"' Wissenschaftler üblicherweise kennt oder die er notfalls in einem
%Lexikon nachschlagen kann. 
%Die Zusammenfassung sollte keine Literaturhinweise enthalten. Die Zusammenfassung ist ein
%selbständiger Teil der Arbeit. Das bedeutet, daß die in der Zusammenfassung erklärten
%Abkürzungen im Hauptteil noch einmal erklärt werden müssen.
%Einerseits darf die Zusammenfassung nicht zu lang sein (max eine Seite), andererseits muß sie aber auch alle wichtigen
%Informationen über Ihre Untersuchungen enthalten. Auf präzise Formulierungen ist größten
%Wert zu legen. 
\end{small}
