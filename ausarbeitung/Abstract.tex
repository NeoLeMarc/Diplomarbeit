\chapter*{Abstract}
\begin{small}
In dieser Arbeit wird ein kabelloses Sensornetzwerk zur Überwachung von Patienten bei einem 
Massenanfall von Verletzten entworfen und implementiert. Als Grundlage dieser Entwicklung
dient der am IBT entwickelte \emph{Erste-Hilfe-Sensor}.
Hierzu wird zunächst eine Untersuchung aller in Frage kommenden Funktechnologien vorgenommen,
wobei ein besonderes Augenmerk auf Energieffizienz und Stabilität der Funktechnologie gelegt wird.
Hierbei stellt sich \emph{ZigBee} als am besten geeignetes Protokoll heraus, und es wird entschieden,
eine Umsetzung auf Basis eines \emph{ZigBee}-Moduls des Herstellers \emph{Atmel} durchzuführen. 
Dieses Modul wird einer genaueren Betrachtung unterzogen, und eine Reihe von Schwierigkeiten werden
identifiziert. Für diese Schwierigkeiten werden allgemeingültige Lösungen entwickelt. Auf Basis dieser
allgemeinen Lösungen werden später spezielle Lösungen für die \emph{Java-Plattform} sowie für den
Mikrocontroller des \emph{Erste-Hilfe-Sensors} entwickelt.
\\                    
Bei der Entwicklung des Gesamtsystems wird ein komponentenbasierter Ansatz gewählt. Hierdurch wird nicht
nur die Austauschbarkeit der einzelnen Komponenten erreicht, sondern auch deren Verteilung über
mehrere Rechner ermöglicht. Für die Verbindung der einzelnen Komponenten zueinander werden 
\emph{Corba}-Schnittstellen verwendet. Diese Schnittstellen sind so gestaltet, dass auch externe
Software angebunden werden kann. Die Anbindung exemplarisch am Beispiel einer \emph{Weboberfläche} 
durchgeführt.
Außerdem wird eine Platine entwickelt, auf der das \emph{ZigBee}-Modul und der
\emph{Mikrocontroller} des \emph{Erste-Hilfe-Sensors} integriert sind. Diese Platine wird zur 
Entwicklung eines \emph{ZigBee}-Treibers für die Firmware des \emph{Erste-Hilfe-Sensors} verwendet.
\\
Das Gesamtsystem wird zum Abschluß eingehend evaluiert. Hierbei wird gezeigt, dass alle Komponenten
erfolgreich miteinander interagieren. Außerdem wird der Stromverbrauch des \emph{ZigBee}-Moduls
untersucht, und die Abhängigkeit von den Parametern des Energiesparmodus berechnet. Hierbei ergibt 
sich ein Stromverbrauch von ca. 1,86~mA bei der Wahl von realistischen Parametern für den 
Energiesparmodus.
%\begin{small} -- Falls es nicht auf eine Seite passt darf die Schrift im Abstract kleiner gemacht werden, aber nur beim Abstract!
%... Die Zusammenfassung ("`Abstract"') ist nach dem Titel der zweitwichtigste Bestandteil einer
%wissenschaftlichen Arbeit. Sie sollten deshalb für die Zusammenfassung, ebenso wie für den
%Titel, besonders viel Mühe und Zeit verwenden, da die gesamte wissenschaftliche Arbeit nur
%von sehr wenigen Wissenschaftlern gelesen wird, die Zusammenfassung aber von vielen. Aus
%der Zusammenfassung muß hervorgehen, wovon die Arbeit handelt, worauf sie aufbaut, und
%vor allen Dingen, welche neuen Erkenntnisse gewonnen wurden. Die wichtigsten Ergebnisse
%der Arbeit müssen kurz und präzise aufgezählt werden. Es genügt nicht zu schreiben, daß dies
%und jenes in der Arbeit behandelt werden. Wichtig ist, daß die "`harten Fakten"', welche sich
%aus den Untersuchungen ergeben haben, aufgelistet sind.
%Handelt es sich um eine theoretische Arbeit, dann müssen Sie erwähnen, von welchen
%Gleichungen Sie ausgegangen sind und welche Näherungen Sie verwendet haben; bei einer
%experimentellen Arbeit müssen Sie erwähnen, welche Experimente Sie durchgeführt haben
%und eventuell auch, welche Auswertemethoden (falls nicht Standardmethoden) Sie verwendet
%haben.Beachten Sie, daß Ihre Arbeit von Wissenschaftlern unterschiedlicher Herkunft und
%Ausbildung gelesen wird. Bedenken Sie, daß sich auch Wissenschaftler für Ihre Arbeit
%interessieren können, die aus benachbarten Disziplinen stammen und nicht mit dem von Ihnen
%verwendeten wissenschaftlichen "`Jargon"' vertraut sind, oder solche, welche die in Ihrem Fach
%üblichen Abkürzungen nicht kennen. Deshalb soll die Zusammenfassung für alle
%(natur-)wissenschaftlich gebildeten Leser verständlich sein. Das bedingt, daß eventuell
%benutzte Abkürzungen erklärt werden müssen, und daß nur solche Begriffe vorkommen
%dürfen, die ein "`normaler"' Wissenschaftler üblicherweise kennt oder die er notfalls in einem
%Lexikon nachschlagen kann. 
%Die Zusammenfassung sollte keine Literaturhinweise enthalten. Die Zusammenfassung ist ein
%selbständiger Teil der Arbeit. Das bedeutet, daß die in der Zusammenfassung erklärten
%Abkürzungen im Hauptteil noch einmal erklärt werden müssen.
%Einerseits darf die Zusammenfassung nicht zu lang sein (max eine Seite), andererseits muß sie aber auch alle wichtigen
%Informationen über Ihre Untersuchungen enthalten. Auf präzise Formulierungen ist größten
%Wert zu legen. 
\end{small}
