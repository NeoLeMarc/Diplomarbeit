\chapter*{Abstract}
%\begin{small} -- Falls es nicht auf eine Seite passt darf die Schrift im Abstract kleiner gemacht werden, aber nur beim Abstract!
... Die Zusammenfassung ("`Abstract"') ist nach dem Titel der zweitwichtigste Bestandteil einer
wissenschaftlichen Arbeit. Sie sollten deshalb f�r die Zusammenfassung, ebenso wie f�r den
Titel, besonders viel M�he und Zeit verwenden, da die gesamte wissenschaftliche Arbeit nur
von sehr wenigen Wissenschaftlern gelesen wird, die Zusammenfassung aber von vielen. Aus
der Zusammenfassung mu� hervorgehen, wovon die Arbeit handelt, worauf sie aufbaut, und
vor allen Dingen, welche neuen Erkenntnisse gewonnen wurden. Die wichtigsten Ergebnisse
der Arbeit m�ssen kurz und pr�zise aufgez�hlt werden. Es gen�gt nicht zu schreiben, da� dies
und jenes in der Arbeit behandelt werden. Wichtig ist, da� die "`harten Fakten"', welche sich
aus den Untersuchungen ergeben haben, aufgelistet sind.
Handelt es sich um eine theoretische Arbeit, dann m�ssen Sie erw�hnen, von welchen
Gleichungen Sie ausgegangen sind und welche N�herungen Sie verwendet haben; bei einer
experimentellen Arbeit m�ssen Sie erw�hnen, welche Experimente Sie durchgef�hrt haben
und eventuell auch, welche Auswertemethoden (falls nicht Standardmethoden) Sie verwendet
haben.Beachten Sie, da� Ihre Arbeit von Wissenschaftlern unterschiedlicher Herkunft und
Ausbildung gelesen wird. Bedenken Sie, da� sich auch Wissenschaftler f�r Ihre Arbeit
interessieren k�nnen, die aus benachbarten Disziplinen stammen und nicht mit dem von Ihnen
verwendeten wissenschaftlichen "`Jargon"' vertraut sind, oder solche, welche die in Ihrem Fach
�blichen Abk�rzungen nicht kennen. Deshalb soll die Zusammenfassung f�r alle
(natur-)wissenschaftlich gebildeten Leser verst�ndlich sein. Das bedingt, da� eventuell
benutzte Abk�rzungen erkl�rt werden m�ssen, und da� nur solche Begriffe vorkommen
d�rfen, die ein "`normaler"' Wissenschaftler �blicherweise kennt oder die er notfalls in einem
Lexikon nachschlagen kann. 
Die Zusammenfassung sollte keine Literaturhinweise enthalten. Die Zusammenfassung ist ein
selbst�ndiger Teil der Arbeit. Das bedeutet, da� die in der Zusammenfassung erkl�rten
Abk�rzungen im Hauptteil noch einmal erkl�rt werden m�ssen.
Einerseits darf die Zusammenfassung nicht zu lang sein (max eine Seite), andererseits mu� sie aber auch alle wichtigen
Informationen �ber Ihre Untersuchungen enthalten. Auf pr�zise Formulierungen ist gr��ten
Wert zu legen. 
%\end{small}
