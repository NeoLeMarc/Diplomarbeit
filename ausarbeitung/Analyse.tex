\chapter{Analyse}

    \section{Kommunikation mit dem \emph{ZigBit}-Modul}
        \subsection{Problemstellung}
            Das \emph{SerialNet} Protokoll ist grundsätzlich synchron. Ein Programm, dass mit
            einem \emph{ZigBit}-Modul kommuniziert lässt sich daher am einfachsten nach folgendem
            Muster aufbauen:

            \begin{lstlisting}{frame=analyse1}
def sendeBefehl(befehl)
    # Senden des Befehls
    serialPort.send(befehl)

    # Antwort vom ZigBit-Modul lesen.
    # (Blockiert so lange, bis Antwort empfangen wurde)
    statusString = serialPort.readLine()

    # Antwort interpretieren
    if statusString == 'OK':
        return True
    
    else:
        return False
            \end{lstlisting}
           
        
            Bei dieser Vorgehensweise ergeben sich jedoch zwei Probleme:

            \begin{enumerate}
                \item{Asynchron auftretende Ereignisse:} Die Abarbeitung eines \emph{SerialNet}-Befehls kann 
                          bis zu einer halben Minute dauern\footnote{Dies tritt insbesondere dann auf, wenn eine
                          synchrone Datenübertragung zu einem entfernten Knoten durchgeführt werden soll, der sich
                          selbst gerade im Energiesparmodus befindet. Der sendende Knoten ist hierbei so lange
                          blockiert, bis entweder der Empfang der Daten quittiert wurde, oder aber ein 
                          Timeout aufgetreten ist}. Während das Steuerungprogramm auf die Quittierung des 
                          Befehls mit einer Status-Meldung wartet, können asynchrone Ereignisse auftreten.
                          Im oben abgebildeten Programm würde dieses Ereigniss fälschlicherweise als Antwort
                          auf den gesendeten Befehl interpretiert werden. Dies hätte zur Folge, dass nun 
                          Steurungsprogramm und \emph{ZigBit}-Modul nicht mehr synchron zueinander ("`out-of-sync"') 
                          sind.

                \item{Komplexe Rückgabewerte:} Bestimmte Befehle der \emph{SerialNet}-Firmware liefern nicht nur
                         einen einfachen Statuscode sondern eine komplexe, Mehrzeilige Antwort zurück. Auch hier ergibt
                         sich bei obigem Programm wieder das Problem, dass dieses bereits nach der ersten Zeile
                         fälschlicherweise davon ausgehen würde, dass die Antwort abgeschlossen wäre, obwohl noch weitere
                         Zeilen folgen können. Auch in diesem Falle wären Steurungsprogramm und \emph{ZigBit}-Modul nichtmehr
                         synchron zueinander.
            \end{enumerate}

        \subsection{Behandlung asynchron auftretender Ereignisse}

            Um asynchrone Ereignisse behandeln zu können, müssen diese zunächst erkannt werden. Dies kann mit
            Hilfe einer Liste aller Ereignisse und Statuscodes geschehen. In diesem Falle wird so lange
            von dem \emph{ZigBit}-Modul gelesen, bis ein Statuscode empfangen wurde. Zwischendurch empfangene
            Ereignisse können entweder direkt behandelt werden, oder zunächst in einer Liste zwischengespeichert
            werden, um sie zu einem späteren Zeitpunkt zu behandeln. Beide Verfahren haben sowohl Vor- als auch
            Nachteile:

            \begin{itemize}
                \item{Direktes Behandeln von Events:} Dieses Verfahren hat den Vorteil, dass auf empfangene 
                Ereignisse sofort reagiert werden kann; so lässt sich z.B. ein empfangener Alarm direkt
                auf dem Bildschirm darstellen und es muss nicht erst auf die Beendigung des vorherigen
                Befehls gewartet werden. Erfordert das behandelte Event den Zugriff auf das 
                \emph{ZigBit}-Modul um z.B. den Empfang einer Nachricht zu quittieren, so ist eine
                besondere Vorsicht erforderlich, da sonst entweder Verklemmungen ("'Deadlock"`) oder
                ein Verlust der Synchronisierung auftreten kann\footnote{Eine mögliche Lösung ist die
                Verwendung einer Warteschlange, in die die zu versendenden Daten eingereiht werden.}.
                Dieses Verfahren eignet sich besonders für Systeme, in denen auf eingehende Ereignisse besonders
                schnell reagiert werden muss.

                \item{Einreihen von Ereignisse in eine Warteschlange:} Diese Verfahren hat den Vorteil, dass
                es sehr einfach umgesetzt werden kann. Alle Ereignisse, die während der Bearbeitung eines Befehls
                auftreten werden zunächst in eine Warteschlange eingereiht. Mit der Abarbeitung der Ereignisse
                wird so lange gewartet, bis der laufende Befehl abgeschlossen wurde. Während der Abarbeitung der
                Ereignisse muss dann keine Rücksicht mehr darauf genommen werden, in welchem Zustand sich das
                \emph{ZigBit}-Modul befindet, da es sich nach Abarbeitung des Befehls sicher im Betriebsbereiten
                Zustand befindet\footnote{Ausnahme hiervon ist, wenn zwischenzeitlich die Netzwerkverbindung verloren
                gegangen ist, was natürlich gesondert behandelt werden muss.}. Nachteilig ist, dass Ereignisse
                erst behandelt werden, wenn Befehle komplett abgearbeitet wurden. Bei Ereignissen, zu deren Bearbeitung
                nicht auf das \emph{ZigBit}-Modul zugegriffen werden muss bedeutet dies eine unnötige Verzögerung.
                Dieses Verfahren eignet sich daher eher für Systeme, in denen deutlich mehr Befehle als Ereignisse
                auftreten.
            \end{itemize}

            Durch die Verwendung von Multithreading kann darüber hinaus eine Kombination aus beiden Verfahren
            verwenden werden. In diesem Fall wird von einem Thread aus die Bearbeitung der Befehle, inkl. des
            Befüllens der Ereigniswareteschlange vorgenommen, während von einem anderen Thread aus diese Warteschlange
            parallel dazu abgearbeitet wird.\\
            \\
            Das folgende Pseudocode-Programm zeigt exemplarisch das Behandeln von Events über das Einreihen in
            Warteschlangen:

            \begin{lstlisting}{frame=analyse2}
eventQueue = []

def sendeBefehl(befehl)
    # Senden des Befehls
    serialPort.send(befehl)

    # Antwort vom ZigBit-Modul lesen.
    # (Blockiert so lange, bis Antwort empfangen wurde)
    statusString = serialPort.readLine()

    # Antwort interpretieren
    if statusString == 'OK':
        return True
    
    elif statusString == 'ERROR':
        return False

    else:
        eventQueue.append(statusString)

            \end{lstlisting}

        \subsection{Behandlung von komplexen Antworten}


        \subsection{Kombination beider Lösungen}
                      

    \section{Leistungsaufnahme}
    
    \section{Powermanagement}

    \section{Skalierbarkeit}
