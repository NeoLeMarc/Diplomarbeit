
%%%------------------------------------------------STAND DER TECHNIK------------------------------------------------------------------
\chapter{Pläne und Diagramme}\label{anhang_diagramme}
    In diesem Kapitel werden Schaltpläne, Platinenlayouts und \emph{UML}-Diagramme der in dieser Arbeit 
    entworfenen Komponenten bereitgestellt. Mit Hilfe dieser Pläne und Diagramme kann die Konstruktion 
    der Komponenten nachvollzogen werden und ggf. ein Nachbau erfolgen.

    \Abbildungps{!htb}{0.4}{diagramme/schaltplan_usbstick.pdf}{label}{Schaltplan des \emph{USB-Sticks}.}
    \Abbildungps{!htb}{1.05}{diagramme/schaltplan_manvnode.pdf}{label}{Schaltplan der \emph{MANVNode}.}
    \Abbildungps{!htb}{0.75}{diagramme/platine_manvnode_top.pdf}{label}{Oberseite der Platine der MANVNode.}
    \Abbildungps{!htb}{0.75}{diagramme/platine_manvnode_bottom.pdf}{label}{Unterseite der Platine der MANVNode.}
    \begin{sidewaysfigure}
            \includegraphics*[width=1\textwidth]{uml_diagramme/manvconnector_uml.pdf}
            \caption{Klassendiagramm des MANV-Connectors.}
            \label{klassendiagramm_manvconnector}
    \end{sidewaysfigure}

\chapter{Beschreibung der beigelegten Software}\label{anhang_beschreibung_software}

    In diesem Kapitel wird die beigelegte Software beschrieben. Außerdem wird eine
    kurze Erläuterung gegeben, welche Funktionalität wo implementiert ist, und an
    welcher Stelle für den produktiven Einsatz ggf. noch Anpassungen erfolgen müssen.

    \section{MANV-Connector}
    Im Verzeichnis "`connector"' auf der beigelegten CD befinden sich
    Quellcode und Binaries des \emph{MANV-Connectors}. Zum Kompilieren
    und Starten des Connectors wird das Script "`start.sh"' verwendet.

    \subsection{Starten des Connectors}

    Zum Starten des Connectors muss das Skript "`start.sh"' mit dem Parameter
    "`start"' aufgerufen werden. Evtl. müssen in diesem Skript noch Pfade
    und die IP-Adresse des \emph{MANV-Servers} angepasst werden. 
    Dieser Server muss vor dem \emph{MANV-Connector} gestartet werden.

    \subsection{Kompilieren des Connectors}

    Zum Kompilieren der beigelegten Software wird ebenfalls das "`start.sh"'
    Skript verwendet. Hierzu wird dieses mit dem Parameter "`build"' aufgerufen.
    Zum erfolgreichen Kompilieren sind die \emph{IDL}-Dateien der \emph{MANV-Suite}
    von Jan Tepelmann erforderlich, die auf der CD im Verzeichnis "`jan"'
    mitgeliefert werden. Evtl. müssen hierfür die Pfade im "`start.sh"'-Skript 
    angepasst werden.

    \subsection{Verzeichnisstruktur des Quellcodes}

    Die Verzeichnisstruktur des Quellcodes folgt dem Standard für \emph{Java-Packages}.
    Der Quellcode ist auf mehrere Dateien und Unververzeichnisse aufgeteilt
    und findet sich im Verzeichnis \emph{connector/edu/kit/ibt/manv/connector}.
    
    \begin{table}
        \begin{tabular}{ll}
            \textbf{Verzeichnis} & \textbf{Beschreibung}\\

           commands/ & Befehle, die an den \emph{USB-Stick} gesendet werden.\\
           corba/    & Implementierung der \emph{CORBA}-Schnittstelle.\\
           events/   & Ereignisse, die von dem \emph{USB-Stick} empfangen wurden.\\
           results/  & Antworten auf gesendete Befehle.\\
           lib/      & Klassen, die thematisch nicht in andere Pakete passen.\\
        \end{tabular}
        \caption{Struktur des Quellcodeverzeichnisses}
    \end{table}

    \subsubsection{Inhalt des Verzeichnises \emph{commands/}}

    Innerhalb des \emph{commands/}-Verzeichnisses findet sich die Datei \emph{MANVCommands.java},
    in der die Kasse \emph{MANVCommand} spezifiziert wird. Objekte dieser Klasse kapseln jeweils
    einen \emph{SerialNet}-AT-Befehl (vergleiche Tabelle~\ref{serialnet_befehle}), der an den
    \emph{USB-Stick} gesendet wird. Dieser Befehl wird als einfacher ASCII-String an den Konstruktor
    übergeben. Optional kann bei wichtigen Befehlen noch eine Priorität mitgegeben werden;
    je höher diese Priorität ist, desto eher wird dieser Befehl an den \emph{USB-Stick}
    gesendet. Wenn der Befehl abgearbeitet wurde, wird in dem Objekt das Ergebniss
    der Ausführung gespeichert, welches mit der Methode \emph{MANVResult.getResult()}
    abgerufen werden kann.

    \subsubsection{Inhalt des Verzeichnises \emph{corba/}}

    In dem Verzeichniss \emph{corba/} findet sich der Quellcoode der Implementierung der \emph{Corba}-Schnittstelle
    zum \emph{MANV}-Server. Die einzelnen Dateien haben folgende Funktion:

    \begin{itemize}
        \item{CommandsImpl.java:} Definiert, wie auf vom Server empfangene Befehle reagiert werden soll.
                                  Leitet empfangene Befehle an das \emph{ZigBee}-Netzwerk weiter.
        \item{MessageContainer.java:} Container zur Auslieferung von \emph{ZigBee}-Nachrichten in \emph{Corba}-Nachrichten.
                                      Für jeden Nachrichtentyp gibt es eine spezielle Unterklasse 
                                      (\emph{CorbaDataMessageContainer},
                                      \emph{CorbaEventMessageContainer}). Der Inhalt der Container wird von der Methode 
                                      \emph{createCorbaMessages()} der Unterklassen von \emph{events/MANVEvent.java}
                                      konstruiert.
        \item{CobaSender.java:} Thread, der empfangene \emph{ZigBee}-Ereignisse an den \emph{MANV-Server} ausliefert.
                                Diese Ereignisse werden aus einer Queue (\emph{eventQueue}) entnommen und sind
                                bereits in \emph{MessageContainer} verpackt.
    \end{itemize}

    \subsubsection{Inhalt des Verzeichnisses \emph{events/}}
    
    Im Verzeichnis \emph{events/} befinden sich die Klassen, welche Ereignisse repräsentieren, die von dem 
    \emph{ZigBee}-Netzwerk empfangen wurden. Alle Ereignis-Klassen sind von der Klasse \emph{MANVEvent} abgeleitet 
    und besitzen damit automatisch eine Priorität. Über diese Priorität kann eingestellt werden, in welcher Reihenfolge 
    empfangene Ereignisse abgearbeitet werden sollen. So ist es z.B. sinnvoll, dass Ereignisse, welche einen Alarm 
    darstellen, eine höhere Priorität haben als eine Mitteilung, dass ein neuer Knoten im Netzwerk vorhanden ist. 
    Nachfolgend eine Übersicht über die Funktion der Methoden der Klasse \emph{MANVEvent}:

    \begin{itemize}
            \item{\emph{isResult()}:} Gibt zurück, ob es sich bei dem Ereignis um das Ergebnis eines gesendeten Befehls handelt. 
            \item{\emph{isImportant()}:} Unterscheidet wichtige von unwichtigen Ereignissen. Nur wichtige Ereignisse werden über 
                                  die \emph{CORBA}-Schnittstelle an den \emph{MANV-Server} weitergeleitet. Unwichtige 
                                  Ereignisse sind nur für die interne Verarbeitung im \emph{MANV-Connector} relevant. 
            \item{\emph{fromString()}:}  Factory\footnote{Eine Fectory-Methode übernimmt das 
                                                 Erstellen von Objekten.}-Methode, die aus einem 
                                                 \emph{SerialNet}-String ein \emph{MANV\-Event} erstellt. 
                                  Liefer automatisch eine Instanz einer Unterklasse von \emph{MANV\-Event} zurück.
            \item{\emph{createCorbaMessages()}:} Bereitet ein \emph{MANVEvent} für den Transport über die 
                                          \emph{CORBA}-Schnittstelle zum \emph{MANV-Server} vor. Hierzu werden ein oder 
                                          mehrere \emph{CorbaEvent}-Objekte erstellt, die wiederum in einen
                                          passenden \emph{CorbaContainer} verpackt werden.
            \item{\emph{getRaw()}:} Liefert den rohen \emph{SerialNet}-String zurück, auf Basis dessen das \emph{MANVEvent}-Objekt 
                             erzeugt wurde; dies ist vor allem für Debugging-Zwecke interessant.
    \end{itemize}

    Von der Klasse \emph{MANVEvent} sind folgende Unterklassen abgeleitet:

    \begin{itemize}
        \item{\emph{MANVChildJoined}:} Eine neue \emph{MANVNode} ist dem Netzwerk beigetreten.
        \item{\emph{MANVChildLost}:} Eine \emph{MANVNode} hat die Verbindung zum Netzwerk verloren.
        \item{\emph{MANVStatusMessage}:} Eine Datennachricht, die Statusinformationen (Alarmzustand, Puls, Atmung) enthält,
                                  wurde empfangen.
        \item{\emph{MANVDataReceived}:} Eine Datennachricht, die keine Statusinformationen enthält, wurde empfangen.
    \end{itemize}

    \subsubsection{Inhalt der Verzeichnises \emph{results/}}

    In dem Verzeichnis \emph{results/} befinden sich alle Klassen, die auf gesendete \emph{SerialNet}-Befehle empfangene 
    Antworten repräsentieren. Diese sind von der Basisklasse \emph{MANVResult} abgeleitet, welche wiederum eine Unterklasse 
    von \emph{MANVEvent} ist.  Damit besitzt diese Klasse alle Eigenschaften und Methoden, die auch die 
    Klasse \emph{MANVEvent} besitzt; insbesondere handelt es sich hierbei selbst wieder um eine Unterklasse
    von \emph{MANVPrioritized}. 
    Zusätzlich zu den von den Oberklassen ererbten Methoden werden folgende Methoden neu eingeführt:

    \begin{itemize}
        \item{\emph{getData()}:} Liefert die von der \emph{SerialNet-Firmware} empfangenen Daten in Stringform zurück. 
        \item{\emph{isComposite()}:} Gibt zurück, ob es sich um eine komplexe Antwort (vgl. 
                            Abschnitt~\ref{analyse_kommunikation_mit_zigbit_modul}) handelt. In diesem Fall befindet 
                            sich in dem Attribut \emph{subResult} eine gültige Referenz auf ein Unterergebnis.
        \item{\emph{getChildList()}:} Falls mit dem gesendeten \emph{SerialNet}-Befehl nach den assoziierten Kindknoten gefragt wurde,
                             gibt diese Methode eine Liste dieser Kindknoten zurück.
    \end{itemize}

    Neben den beiden Statusinformationen \emph{OK} und \emph{ERROR} kann die \emph{SerialNet-Firmware} auch 
    kompliziertere Antworten zurückliefern. Diese werden von folgenden Unterklassen von \emph{MANVEvent} repräsentiert:
        
    \begin{itemize}
        \item{\emph{MANVChildrenList}:} Repräsentiert eine Liste von Kindknoten, die von der \emph{SerialNet-Firmware}
                                 empfangen wurde.
        \item{\emph{MANVGsn}:} Liefert die \emph{Global Subscriber Number} (GSN) einer Node zurück. Entspricht in etwa einer 
                        \emph{MAC}-Adresse.
    \end{itemize}

    \subsubsection{Inhalt des Verzeichnisses \emph{lib/}}

    In dem Verzeichnis \emph{lib/} finden sich alle die Klassen, welche nicht in obiges Einordnungsschema passen. 
    Hierbei handelt es sich insbesondere um Helferklassen, die in den obigen Klassen verwendet werden:

    \begin{itemize}
        \item{\emph{iZigBit}:} Interface für einen entfernten \emph{ZigBit}-Knoten. Diese werden nochmal weiter
                        in vollwertige Knoten (\emph{ZigBit}, im wesentlichen also die \emph{MANV-Nodes})
                        und Knoten, von denen nur Daten empfangen werden können (\emph{readonlyZigBit})
                        unterschieden.
        \item{\emph{MANVPrioritized}:} Allgemeine Oberklasse für priorisierte Entitäten wie z.B.\\ \emph{MA\-NV\-Command},
                                \emph{MANVEvent} oder \emph{MANVResult}. Diese Entitäten enthalten 
                                sowohl eine numerische Priorität als auch eine \emph{compareTo()}-Operation,
                                über die eine Sortierung vorgenommen werden kann.
        \item{\emph{SocketReader}:} Definiert einen Thread, der von dem Socket liest, der die Verbindung zum 
                                    \emph{MANV-USB-Stick} beinhaltet. Dieser Thread empfängt Ereignisse und 
                                    Ergebnisse, wandelt sie in Instanzen vom Typ \emph{MANVEvent} und 
                                    \emph{MANVResult} um und sortiert
                             in die Ereigniswarteschlange (\emph{eventQueue}) ein. Empfangene Ergebnisse werden
                             automatisch dem passenden Objekt vom Typ \emph{MANVCommand} zugeordnet.
        \item{\emph{SocketWriter}:} Definiert einen Thread, der zu sendende Befehle aus der Befehlswarteschlange
                             (\emph{commandQueue}) entnimmt und über die Socketverbindung zum \emph{MANV-USB-Stick}
                             an das \emph{ZigBee}-Netzwerk weiterleitet. Sobald ein Befehl abgearbeitet ist,
                             wird er von dem \emph{SocketWriter}-Thread automatisch in die Ergebniswarteschlange 
                             (\emph{resultQueue}) einsortiert.
    \end{itemize}


    \paragraph{Interface eines ZigBit-Knotens}

    Es ist nicht immer möglich, Daten an einen \emph{ZigBit}-Knoten zu senden. Daher gibt
    es zwei Arten von \emph{ZigBit}-Knoten: \emph{ZigBit} und {ReadonlyZigBit}. Um diese trotzdem
    generisch ansprechen zu können, implementieren beide das Interface \emph{iZigBit}. Dieses definiert folgende Methoden:

    \begin{itemize}
        \item{\emph{getNodeID()}:} Liefert die logische Adresse (\emph{WSRC}) eines Knotens zurück.
        \item{\emph{getMacID()}:} Liefert die Hardware-Adresse (\emph{GSN}) eines Knotens zurück.
        \item{\emph{sendData()}:} Sendet Daten an einen Knoten. Blockiert so lange, bis die Bearbeitung des
                           Befehls abgeschlossen ist. Wird diese Operation auf einem \emph{ReadonlyZigBit} wird
                           kein Versand der Daten ausgeführt und statt dessen ein Fehler zurückgegeben.
        \item{\emph{isendData()}:} Senden von Daten an einen Knoten. Kehrt sofort zurück, ohne auf die Bearbeitung
                            des Befehls zu warten. 
        \item{\emph{toggleAlertStatus()}:} Umschalten des Alarmstatus einer \emph{MANV-Node}. Keine Funktionalität
        \item{\emph{disableAlert()}:} Deaktivieren des Alarms einer \emph{MANV-Node}.
        \item{\emph{enableAlert()}:} Aktivieren des Alarms einer \emph{MANV-Node}.
        \item{\emph{muteAlert()}:} Alarm stummschalten.
        \item{\emph{itoggleAlertStatus()}, \emph{idisableAlertStatus()}, \emph{ienableAlert()}, \emph{imuteAlert()}}: Selbe Funktionalität
            wie die Versionen ohne i, jedoch ohne Warten auf Abarbeitung (analog zu \emph{sendData()} / \emph{isendData()}).
    \end{itemize}

    \section{Serial-To-Socket}

    In dem Verzeichnis \emph{serial\_to\_socket} befinden sich eine Reihe von Python-Skripten, die die
    Verbindung zwischen \emph{MANVConnector} und \emph{MANV-USB-Stick} herstellen. Hierzu 
    wird zunächst mit Hilfe des Skriptes \emph{serial\_to\_socket.py} eine automatische Suche nach dem 
    \emph{MANV-USB-Sticks} (vgl. Abschnitt~\ref{usb_stick_erkennen}) durchgeführt. Der Status
    dieser Suche wird graphisch ausgegeben. Sobald ein \emph{MANV-USB-Stick} erfolgreich
    erkannt wurde, wird das Skript \emph{serial\_to\_socket/start.sh} gestartet und die
    Adresse des seriellen Ports als Parameter übergeben. In diesem Skript wird
    dann unter anderem das Pythonskript "`serial\_to\_socket.py"' gestart. Dieses leitet einen
    angegebenen seriellen Port auf den Netzwerkport \emph{4711} um, über den dann der
    \emph{MANVConnector} eine Verbindung zu dem \emph{USB-Stick} herstellen kann.
    
    \section{Firmware}

    In dem Verzeichnis \emph{firmware/} befindet sich der Quellcode der \emph{Firmware} für die \emph{MANVNode}.
    Diese besteht aus folgenden Dateien:

    \begin{itemize}
        \item{\emph{main.c}:} Einstiegspunkt der \emph{Firmware}. Initialisiert alle Ringpuffer, \emph{Interrupts} und 
                       die \emph{Peripherie} des \emph{ADuC}-Mikrocontrollers. Ruft danach die 
                       Hauptscheife des \emph{ZigBit}-Treibers auf.
        \item{\emph{isr.c}:}  Definition der verwendeten \emph{Interrupt-Service-Routinen}, also Behandlungsroutinen
                       für auftretende \emph{Interrupts}. Ist unter anderem für die Ansteuerung des \emph{UART} 
                       verantwortlich.
        \item{\emph{random.c}:} Implementierung eines sehr einfachen Pseudozufallszahlengenerators.
        \item{\emph{ringpuffer.c}:} Implementierung der Datenstruktur "`Ringpuffer"'. Diese kann 5 Zeilen
                             von jeweils 80 Zeichen Länge speichern. Werden mehr als 5 Zeilen in die
                             Datenstruktur gelegt, dann wird das älteste Element überschrieben.
                             Ringpuffer werden unter anderem zur Implementierung des \emph{UART}-Treibers
                             sowie zur Zwischenspeicherung von empfangenen Befehlen zur späteren
                             Abarbeitung verwendet.
        \item{\emph{uart.c}:} Beinhaltet die Routinen, die zur Interaktion mit dem als \emph{ISR} implementierten
                       Treiber für den \emph{UART} notwendig sind. Im wesentlichen handelt es sich um 
                       Routinen, die die entsprechenden Ringpuffer befüllen oder auslesen.
        \item{\emph{zigbit.c}:} Beispiel einer Interaktion mit dem \emph{ZigBit}-Modul. Enthält unter anderem Routinen
                         zur Befehlsverareitung, Extraktion von Rückgabewerten, Initialisierung des \emph{ZigBit}-Moduls
                         sowie ein Beispiel, wie die Übertragung von Statusinformationen realisiert werden kann.
    \end{itemize}

    Bei der \emph{Firmware} handelt es sich um eine Demonstration, wie die Kommunikation zwischen \emph{ADuC}-Mikrocontroller
    und \emph{ZigBit}-Modul realisiert werden kann. Zur späteren Verwendung auf dem \emph{Erste-Hilfe-Sensor} müssen 
    einige Anpassungen vorgenommen werden. Hierzu wird folgende Vorgehensweise vorgeschlagen:

    \begin{enumerate}
        \item Übernahme der \emph{UART}-spezifischen Routinen aus \emph{isr.c} und \emph{uart.c}. 
        \item Übernahme der Datenstruktur \emph{Ringpuffer} aus \emph{ringpuffer.c}.
        \item Übernahme der Funktionen \emph{extractStatus()}, \emph{waitForStatus()}, \emph{isCommand()}
              und \emph{parseCommand()} aus der Datei \emph{zigbit.c}.
        \item Anpassung der Funktion \emph{initZigBit()} aus der Datei \emph{zigbit.c}:
              Zu beachten ist hierbei Zeile 154: Hier wird eine \emph{GPIO}-Leitung des \emph{ADuC} Mikrocontrollers 
              dazu verwendet,
              die Resetleitung des \emph{ZigBit}-Moduls zu betätigen. Dies ist für die Synchronisierung
              zwingend erforderlich und muss im Hardwarelayout entsprechend berücksichtigt werden.
              Nach Zeile 154 sollten alle Zeilen mit Zugriff auf \emph{GP2DAT} entfernt werden, 
              da diese dazu dienen, die \emph{LEDs} der \emph{MANVNode} anzusteuern. Diese \emph{LEDs} sind
              auf dem Erste-Hilfe-Sensor nicht vorhanden. Die Aufrufe von \emph{sleep()} sind zur Synchronisierung
              zwingend erforderlich und dürfen nicht entfernt werden.
        \item Anpassung der Funktion \emph{zigBitLoop()} aus der Datei \emph{zigbit.c} Hier werden aktuell noch 
              Dummy-Informationen übertragen, die durch einen Zufallszahlengenerator erzeugt werden. Diese
              müssen durch die tatsächlich gemessenen Werte des Erste-Hilfe-Sensors ersetzt werden.
    \end{enumerate}                


    \section{MANV-Web}
    Im Verzeichnis \emph{manvweb/} befinden sich die Quelldateien des Webinterfaces der
    \emph{MANV\-Suite}. Dabei verteilt sich die Funktionalität auf folgende Unterdateien:

    \begin{itemize}
        \item{\emph{MANVWeb}:} Einstiegsklasse des Webservers. Stellt die \emph{CORBA}-Verbindung her
                        und übergibt dann die Steuerung an die Klasse \emph{MANVWebserver}.
        \item{\emph{MANVWebserver}:} Implementierung des Webervers. Öffnet einen Serversocket und
                              wartet auf eingehende Verbindungen. Für jede eingehende Verbindung
                              wird nun ein neuer Thread gestartet. In diesem Thread
                              wird die Eingabe geparsed und eine entsprechende Reaktion
                              aufgerufen. 
        \item{\emph{ToQueries}:} Adapterklasse zum \emph{CORBA}-Zugriff auf den \emph{MANV-Server}. 
    \end{itemize}

    \section{Platine}

    In dem Verzeichnis \emph{platine/} befinden sich die \emph{Eagle}-Dateien, in 
    denen Schaltplan und Platinenlayout des \emph{MANV-USB-Sticks} und der
    \emph{MANV-Node} spezifiziert sind.

    \section{Simulator}

    In dem Verzeichnis \emph{simulator/} befindet sich ein in Python programmierter
    \emph{MANVNode}-Simulator. Dieser kann dazu verwendet werden, mit Hilfe
    eines \emph{MANV-USB-Sticks}, \emph{MANVNodes} zu simulieren. Beim Start des
    Simulators muss diesem die Adresse des seriellen Ports, auf dem sich der
    \emph{USB-Stick} befindet, übergeben werden. Das Format ist Betriebssystem spezifisch. Unter
    Linux wird einfach der Dateiname der entsprechenden \emph{Device-Node} (z.B. \emph{/dev/ttyUSB1}) 
    angegeben, unter Windows der Name des \emph{COM-Ports} (z.B. \emph{COM1:}).

    \section{Zigbit Firmware}
    In dem Verzeichnis \emph{ZigBit-Firmware} befindet sich eine aktuelle Kopie der
    \emph{SerialNet-Firmware} inkl. dem zum Programmieren der \emph{ZigBit}-Module
    nötigen Bootloader in einer Version für Windows.
