%%%------------------------------------------------ZUSAMMENFASSUNG UND AUSBLICK------------------------------------------------------

\chapter{Zusammenfassung und Ausblick}\label{Ausblick}

\section{Zusammenfassung}
\nomenclature{ANT+}{Proprietärer Funkstandard für die Datenübertragung im Nahbereich, z.B. für Pulsuhren.}
In dieser Arbeit wurde ein kabelloses Sensornetzwerk zur Überwachung von Patienten bei einem 
Massenanfall von Verletzten (\emph{MANV}) entworfen und implementiert. Es wurden 
mehrere kabellose Sensornetzprotokolle vorgestellt und es fand eine Tauglichkeitsprüfung dieser statt.
Bereits in diesem Schritt wurde die Entscheidung
getroffen, eine Lösung auf Basis des \emph{ZigBee}-Protokolles zu implementieren. Hierfür sprachen
der geringste Strombedarf aller verfügbaren Lösungen, die hohe Störsicherheit sowie die
gute Verfügbarkeit passender Hardware auf dem Markt. Es wurde weiterhin festgestellt, dass sich die
Marktsituation in den kommenden Jahren auf Grund der Einführung neuer Standards wie \emph{ANT+},
\emph{Bluetooth Low Energy} sowie \emph{Wireless USB} ändern könnte, so dass beim späteren Entwurf
insbesondere auf die Austauschbarkeit des Netzwerkprotokolles zu achten ist. Im Anschluss daran wurden
verschiedene aktuell auf dem Markt erhältliche Lösungen zur kabellosen Patientenüberwachung betrachtet 
und mit den Erkenntnissen aus der Analyse der verschiedenen Netzwerkprotokolle auf ihre Eignung untersucht.
Hierbei wurde festgestellt, dass alle betrachteten Lösungen eher für den Einsatz im Krankenhaus
gedacht und für den Einsatz während eines \emph{MANVs} nicht geeignet sind. Weiterhin wurden verwandte
Projekte vorgestellt, die sich ebenfalls mit der technischen Patientenversorgung während eines \emph{MANVs} beschäftigen
und die Ähnlichkeiten und Unterschiede gezeigt.\\
Mit den aus der Betrachtung der Stand der Technik gewonnenen Erkenntnissen konnte nun zur Analyse 
übergegangen werden. Hier wurde zunächst eine Reihe von Anforderungen erarbeitet, die die zu entwerfende 
Lösung erfüllen soll. Als besonders wichtig sind hierbei eine hohe Teilnehmerzahl (mehrere hundert), 
ein geringer Strombedarf sowie eine ausreichend hohe Reichweite zu nennen. 
Außerdem wurde die für den späteren Einsatz ausgewählte \emph{ZigBee}-Hardware 
der Firma \emph{Atmel} untersucht, und es wurde eine Reihe von Eigenheiten im Kommunikationsprotokoll 
zwischen \emph{ZigBee}-Modul und Mikrokontroller entdeckt, für die jeweils eine passende Lösung in 
allgemeiner Form vorgestellt wurde.  Außerdem wurden kurz die zwei verschiedenen Arten, 
den Energiesparmodus der Module aufzurufen, erläutert, und der manuelle Aufruf als bessere Lösung ausgewählt.\\
Nach der Analyse folgte der Entwurf eines Systems zur Lösung der gestellten Aufgabe. Dieser erfolgte in 3 
Schritten: Zunächst erfolgte der Entwurf des Gesamtsystems im Groben. In Hinblick auf die in der Analyse
gewünschte Austauschbarkeit des Funkprotokolls wurde entschieden, wichtige Designentscheidungen zu kapseln und
so das System aus einzelnen, austauschbaren Komponenten aufzubauen. Im Groben erfolgte eine Aufteilung in 
einer Kontrollsoftware (\emph{MANVSuite}), einer Test- und Entwurfplatine (\emph{MANVNode}), 
dem \emph{MANV-USB-Stick}, sowie der Schnittstelle zwischen Hard- und Software (\emph{MANVConnector}).\\
\\
Die \emph{MANVSuite} wurde in einer zu dieser Diplomarbeit parallel laufenden Studienarbeit von 
Herrn cand. inform Jan Tepelmann genauer entworfen und implementiert (\cite{Jan}).\\
\\
Nach dem groben Entwurf folgt der Entwurf im Feinen. Hierbei wurden Hardware (\emph{MANVNode} und \emph{USB-Stick})
sowie Software (\emph{MANVConnector} und Firmware der \emph{MANVNode}) getrennt betrachtet.\\
\\
Auf Grundlage dieses Entwurfs wurde nun eine Implementierung durchgeführt. 
Bei der Bestückung der Platine der \emph{MANVNode} sind einige Probleme mit der Stromversorgung aufgetreten,
die jedoch durch eine Änderung des Platinenlayouts behoben werden konnten. Außerdem waren praktische
Probleme, wie die automatische Erkennung des \emph{MANV-USB-Sticks}, sowie der Zugriff auf den seriellen Port
mittels Java zu lösen. In beiden Fällen konnte durch die Kapselung von betriebssystemspezifischem Code in
\emph{Python}-Skripte Abhilfe geschaffen werden. \\
Die Implementierung wurde einer Reihe von Tests unterzogen. Ziel hierbei war es,
festzustellen, wie gut die vorgestellte Lösung die in der Analyse aufgestellten Anforderungen erfüllt. Neben
Reichweite, Strombedarf und Erweiterbarkeit des Funknetzwerks wurde hierzu auch die Leistungsfähigkeit der 
Software in Hinblick auf Anzahl der Teilnehmer, Austauschbarkeit des \emph{MANVConnectors} und Interoperabilität
zwischen \emph{MANVSuite}, \emph{MANVConnector} und Sensornetzwerk intensiv untersucht. 
Besondere Augenmerk lag bei dieser Untersuchung auf der Interoperabilität aller Komponenten, die durch 
Integrationgstests und Austausch einzelner Teile gezeigt werden konnte. Außerdem wurde die Strombedarf der 
\emph{ZigBee}-Module sowie deren Reichweite eingehend untersucht. Eine Betrachtung der Anzahl der Benutzer
erfolgte analytisch anhand des \emph{ZigBee}-Standards. Durch diese Tests konnte gezeigt werden, dass die
Lösung die anfangs Erhobenen Anforderungen erfüllen kann.  
Lediglich bei der Datensicherheit innerhalb des \emph{ZigBee}-Netzwerkes müssen 
Einschränkungen hingenommen werden, da die Firmware des Herstellers \emph{Atmel} nicht den versprochenen
Funktionsumfang bereit stellt. Dies wird in Abschnitt~\ref{Sicherheit} genauer untersucht. Hier werden
auch mehrere Vorschläge gemacht, wie dieses Problem gelöst werden kann. Die Arbeit wird abgerundet durch ein
umfangreiches Softwarepaket inkl. Quellcode und Dokumentation, die die zeitnahe Realisierung als Produkt ermöglichen.

\section{Sicherheit des Netzwerkes}
\label{Sicherheit}
Das Thema Sicherheit spielt\,--\,gerade für ein medizinisches Netzwerk\,--\,eine nicht zu vernachlässigende Rolle.
Daher sollen an dieser Stelle einige Anregungen gegeben werden, wie ein Sicherheitskonzept für das entworfene Netzwerk 
aussehen könnte.

\subsection{Aktueller Stand}
Grundsätzlich unterstützt \emph{ZigBee} eine 128-Bit \emph{AES}-Verschlüsselung. Die \emph{SerialNet-Firmware} der in 
dieser Diplomarbeit eingesetzten \emph{ZigBee-Module} der Firma \emph{Atmel} bietet jedoch in der aktuell vorliegenden 
Version (1.8.0) keine Möglichkeit, diese Verschlüsselung zu verwenden. Laut Auskunft des E-Mail-Supports der Firma 
\emph{Atmel} ist dieses Feature auch für zukünftige Versionen nicht geplant.\\
\subsection{Angriffsszenarien}

Zur Beurteilung der Gefahrenlage, sowie zur Entwicklung von Abwehrmaßnahmen erfolgt zunächst eine Betrachtung von
möglichen Angriffsszenarien. 

\subsubsection{Ausspähen von Patientendaten}
Aktuell werden über das Netzwerk Vitaldaten von Patienten (Puls und Atmung) pseudonym versendet. Als Pseudonym dient 
hierbei die Hardware-Adresse des versendenden \emph{ZigBee}-Moduls. Passives Mitschneiden der Daten ist so lange kein Problem,
so lange sich keine Zuordnung zwischen der Identität des Patienten und der Hardware-Adresse des verwendeten Moduls
herstellen lässt. Hierfür ist eine direkte räumliche Nähe zwischen Patient und Angreifer notwendig. 
Während eines \emph{MANV}-Szenarios sollte dies jedoch aktiv durch die Rettungskräften, insbesondere durch die Absperrung 
durch die Polizei vor Ort verhindert werden. Sollte es einem Angreifer gelingen, nahe genug an das entsprechende Modul
heranzukommen, um eine Zuordnung vorzunehmen, ist sowieso davon auszugehen, dass er auch in der Lage wäre, Bild oder
Videoaufnahmen des Patienten herzustellen, was im Zweifelsfall eine deutlich größeren Eingriff in die Privatsphäre 
darstellen würde.

\subsubsection{Aktiver Angriff gegen das Netzwerk}
\nomenclature{Replay}{Angriff, bei der legitime Datenpakete abgefangen und zu einem späteren Zeitpunkt wieder ins
                      Netzwerk eingegeben werden.}
Ein größeres Problem als vom passiven Belauschen des Datenverkehrs geht von aktiven Angriffen gegen das Netzwerk
aus. Innerhalb eines unverschlüsselten Netzwerkes kann ein Angreifer die Identität eines beliebigen \emph{ZigBee}-Moduls 
übernehmen; er hat somit die Möglichkeit, entweder falsche Vitaldaten oder aber Befehle an die einzelne Sensoren zu
versenden. Dies könnte im schlimmsten Falle bedeuten, dass Alarme im Netzwerk verloren gehen, und somit Patienten
zu Schaden kommen. Da selbst bei aktivierter \emph{ZigBee}-Verschlüsselung kein Schutz gegenüber 
\emph{Replay}-Angriffen besteht, hätte
ein Angreifer weiterhin die Möglichkeit, eine aufgefangene Statusnachricht oder einen aufgegangen Befehl beliebig oft
wieder in das Netzwerk einzubringen und so eine Störung zu erreichen.

\subsubsection{Störung des Netzwerkes}
Ein Angreifer hat mehrere Möglichkeiten, ein bestehendes Netzwerk empfindlich zu stören oder sogar jegliche Kommunikation
zum Erliegen zu bringen. Im Falle eines unverschlüsselten Netzwerkes kann dies einfach dadurch geschehen, dass 
der Angreifer die Identität des Netzwerkkoordinators annimmt, und somit die Statusnachrichten ins Leere geleitet werden. Eine
andere Möglichkeit besteht darin, das Netzwerk mit Paketen zu überfluten und so den Datenverkehr der anderen Teilnehmer
zu blockieren. Aber auch bei Einsatz von Verschlüsselung ist eine Störung des Netzwerkes möglich. Hierzu muss ein Angreifer
einfach mit sehr großer Sendeleistung auf allen \emph{ZigBee}-Frequenzen senden, um diese somit zu blockieren. 

\subsection{Abwehrmaßnahmen}

Zur Abwehr der vorgestellten Angriffsszenarien kommen eine Reihe von Abwehrmaßnahmen in Betracht. Diese werden im
folgenden kurz vorgestellt und auf Ihre Eignung untersucht.

\subsubsection{ZigBee-Verschlüsselung}
Die verwendeten \emph{ZigBee}-Module der Firma \emph{Atmel} unterstützen hardwareseitig die im 
\emph{ZigBee}-Standard definierte 128-Bit-\emph{AES}-Verschlüsselung. Die Integrität der Nachricht wird hierbei durch
die Bildung einer Prüfsumme sichergestellt. Um die Verschlüsselung einsetzen zu können, muss die momentan 
verwendete \emph{SerialNet-Firmware} gegen eine eigene Firmware ausgetauscht werden. \emph{Atmel} 
stellt hierzu einen \emph{ZigBee-Pro}-basiertes 
\emph{SDK} zur Verfügung, mit dessen Hilfe eine entsprechende \emph{Firmware} mit vertretbarem Aufwand entwickelt 
werden kann. Hierbei bietet es sich an, einen an die \emph{SerialNet-Firmware} angelehnten \emph{AT-Befehlssatz} zu 
implementieren, um alle bestehenden Komponenten wie den \emph{MANV-Connector} und die Firmware des 
\emph{ADuC-Mikrocontrollers} weiterverwenden zu können.

\subsubsection{Befehlszähler}
Die \emph{ZigBee}-Verschlüsselung schützt nicht gegen \emph{Replay}-Angriffe. Eine einfache aber effektive Methode 
zur Abwehr solcher
Angriffe ist die Verwendung einer Sequenznummer. Hierbei wird jeder Nachricht eine fortlaufende Nummer hinzugefügt. 
Der Empfänger führt nun eine Liste, in der für jeden Absender die zuletzt verwendete Sequenznummer enthalten ist.
Nachrichten, die eine niedrigere Sequenznummer enthalten, werden verworfen. Diese Methode lässt sich mit geringem
Speicheraufwand ( O(N) für den \emph{Coordinator}, O(1) für einen Endknoten) implementieren. Als Ort der Implementierung
bietet sich entweder die Firmware des \emph{ADuC-Mikrocontrollers} sowie der \emph{MANV-Connector} oder aber direkt 
die \emph{ZigBit-Firmware} an.

\subsubsection{Schutz auf Anwendungsebene}
\nomenclature{OpenSSL}{Open Source Bibliothek, die viele Verschlüsselungsverfahren implementiert.}
\nomenclature{SSL}{Secure Socket Layer}
\nomenclature{TLS}{Transport Layer Security\,--\,Nachfolger von SSL.}
\nomenclature{Denial-Of-Service}{Blockieren eines Dienstes durch Überfluten mit Anfragen oder Ausnutzen eines 
                                Softwarefehlers.}
Alternativ zur Modifikation der \emph{ZigBit-Firmware} besteht auch die Möglichkeit, auf Anwendungsebene zu verschlüsseln.
Sowohl für Java (\emph{OpenSSL}) als auch für den \emph{ADuC-Mikrocontroller} (\emph{WAKAN-Crypto-Toolkit}) sind 
entsprechende Bibliotheken
verfügbar. Eine Implementierung auf Applikationsebene hätte den Vorteil, dass keine Änderung an der \emph{ZigBit-Firmware}
notwendig wären und sie daher ggf. einfacher durchzuführen ist. Von Nachteil ist jedoch, dass diese Lösung keinen Schutz
gegen den Beitritt fremder Teilnehmer in das Netzwerk bietet. Zwar können diese keine Befehle schicken oder Daten ausspähen,
allerdings ist es immer noch möglich, durch die Verwendung von gefälschten Absenderadressen eine 
\emph{Denial-Of-Service-Angriff} gegen das Netzwerk durchzuführen. 

\section{Verbesserung der Reichweite und Störsicherheit}
Die in dieser Arbeit eingesetzten \emph{ZigBee}-Module verfügen über eingebaute Antenne sowie eine sehr 
niedrige Sendeleistung von nur 1~mW. Damit einhergehend ist eine begrenzte Reichweite. Die Störsicherheit dieser Module
ist zwar bereits sehr gut, allerdings kann diese weiter erhöht werden. Im Folgenden werden einige Möglichkeiten
vorgestellt, wie diese Ziele zu erreichen sind.

\subsection{Anderer Frequenzbereich}
In dieser Arbeit wurden lediglich \emph{ZigBee}-Module, die im 2,4~GHz-\emph{ISM-Band} operieren, untersucht. Es gibt
jedoch auch Module, die das 868- (Europa) bzw. 915-MHz-Band (USA) nutzen. Diese versprechen eine bessere Durchdringung
von Hindernissen und damit eine bessere Reichweite und Störsicherheit. Diese Verbesserung kommt zum Preis einer
niedrigeren Datenrate (20 bzw. 40~kbit/sec) und ist zudem länderspezifisch. Inwieweit diese Einschränkungen problematisch
sind, wäre gesondert zu evaluieren. Für den Einsatz in schwierigem Terrain könnte dieser Frequenzbereich jedoch einige
Vorteile bieten.

\subsection{Verwendung leistungsstärkerer \emph{ZigBee}-Module und Antennen}
Für diese Arbeit war der Strombedarf der Module eines der wichtigsten Kriterien. Daher wurden Module mit der geringsten
Leistungsaufnahme und damit auch der geringsten Sendeleistung für den Entwurf der Hardware verwendet. Neben diesen
Modulen sind auch Varianten mit deutlich höherer Sendeleistung erhältlich. Diese kann meist sogar noch zusätzlich durch
den Einsatz von externen Antennen verbessert werden. Mit der entsprechenden Kombination aus Modul und Antenne lässt sich eine
Sendeleistung erreichen, die mehr als 10mal so hoch ist, wie diejenige der in dieser Arbeit verwendeten Module. Dies ist 
insbesondere für den Einsatz in \emph{ZigBee}-Routern interessant, da diese einfach mit einer größeren Stromquelle versehen
werden können als die einzelnen Sensoren. Es wäre daher interessant zu evaluieren, wie sich die Verwendung 
leistungsfähigerer Module sowie größerer Antennen in den Routern auf die erreichbare Reichweiten des Netzwerkes auswirkt.

\subsection{Genauere Betrachtung des \emph{Bluetooth-Low-Energy}-Stan\-dards}
Der \emph{Bluetooth-Low-Energy}-Standard verspricht, eine interessante Alternative zum \emph{ZigBee}-Standard zu werden.
Der einzige Grund, weshalb in dieser Arbeit keine genauere Betrachtung erfolgen konnte, war die Nichtverfügbarkeit
entsprechender Hardware zum Zeitpunkt der Anforderungsanalyse. Bereits im Februar 2010 wurde von der Firma 
\emph{Nordic Semiconductors} unter der Bezeichnung \emph{ISP091201} ein entsprechendes Hardwaremodul angekündigt,
das jedoch bis jetzt (Stand: Oktober 2010) nicht auf dem Markt erhältlich ist. Sobald entsprechende Module verfügbar
sind, sollte eine erneute Evaluation erfolgen. Beim Entwurf der hier vorgestellten Lösung wurde eine spätere Änderung
des Funkprotokolls eingeplant und bei der Implementierung entsprechend berücksichtigt.
