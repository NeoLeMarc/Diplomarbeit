%%%------------------------------------------------ZUSAMMENFASSUNG UND AUSBLICK------------------------------------------------------

\chapter{Zusammenfassung und Ausblick}
***Beispiel wie ein Zitat auf Armins~\citep{Bolz} Buch funktioniert***

\section{Sicherheit des Netzwerkes}
\subsection{Aktueller Stand}
Grundsätzlich unterstützt ZigBee eine 128-Bit AES-Verschlüsselung. Die SerialNet-Firmware der in dieser Diplomarbeit
eingesetzten ZigBee-Module der Firma Atmel bietet jedoch in der aktuell vorliegenden Version (1.8.0) keine Möglichkeit,
diese Verschlüsselung zu verwenden. Laut Auskunft des E-Mail Supportes der Firma Atmel ist dieses Feature auch für
zukünftige Versionen nicht geplant.\\
\\
Da das Thema Sicherheit -- gerade für ein medizinisches Netzwerk -- natürlich eine nicht zu vernachlässigende Rolle spielt,
An dieser Stelle sollen einigen Anregungen gegeben werden, wie eine Sicherheitskonzept für das entworfene Netzwerk aussehen 
könnte.

\subsection{Angriffsszenarien}

\subsubsection{Ausspähen von Patientendaten}
Aktuell werden über das Netzwerk Vitaldaten von Patienten (Puls und Atmung) pseudonym versendet. Als Pseudonym dient 
hierbei die Hardware-Adresse des versendenden ZigBee-Moduls. Passives Mitschneiden der Daten ist so lange kein Problem,
so lange sich keine Zuordnung zwischen der Identität des Patienten und der Hardware-Adresse des verwendeten Moduls
herstellen lässt. Hierfür ist eine direkte räumliche Nähe zwischen Patient und Angreifer notwendig. 
Während eines MANV-Szenarios sollte dies jedoch aktiv durch die Rettungskräftem, insbesondere durch die Absperrung durch
die Polizei vor Ort verhindert werden. Sollte es einem Angreifer gelingen, nahe genug an das entsprechende Modul
heranzukommen, um eine Zuordnung vorzunehmen, ist sowieso davon auszugehen, dass er auch in der Lage wäre, Bild oder
Videoaufnahmen des Patienten herzustellen, was im Zweifelsfall eine deutlich größeren Eingriff in die Privatsphäre 
darstellen würde.

\subsubsection{Aktiver Angriff gegen das Netzwerk}
Ein größeres Problem als durch das passive Belauschen des Datenverkehrs geht von aktiven Angriffen gegen das Netzwerk
aus. Innerhalb eines unverschlüsselten Netzwerkes kann ein Angreifer die Identität eines beliebigen ZigBee-Moduls 
übernehmen; er hat somit die Möglichkeit entweder falsche Vitaldaten oder aber Befehle an die einzelne Sensoren zu
versenden. Dies könnte im schlimmsten Falle bedeuten, dass Alarme im Netzwerk verloren gehen, und somit Patienten
zu Schaden kommen. Da selbst bei aktivierter ZigBee-Verschlüsselung kein Schutz gegenüber Replay-Attacken besteht hätte
ein Angreifer weiterhin die Möglichkeit, eine aufgefangene Statusnachricht oder einen aufgegangen Befehl beliebig oft
wieder in das Netzwerk einzubringen und so eine Störung zu erreichen.

\subsubsection{Störung des Netzwerkes}
Ein Angreifer hat mehrere Möglichkeiten, ein bestehendes Netzwerk empfindlich zu stören oder sogar jegliche Kommunikation
zum erliegen zu bringen. Im Falle eines unverschlüsselten Netzwerkes kann des einfach dadurch geschehen, in dem
der Angreifer die Identität des Netzwerkkoordinators anzunehmen, und somit die Statusnachrichten ins Leere zu leiten. Eine
andere Möglichkeit besteht darin, das Netzwerk mit Paketen zu überfluten, und so den Datenverkehr der anderen Teilnehmer
zu blockieren. Aber auch bei Einsatz von Verschlüsselung ist eine Störung des Netzwerkes möglich. Hierzu muss ein Angreifer
einfach mit sehr großer Sendeleistung auf allen ZigBee-Frequenzen senden, um diese somit zu blockieren. 

\subsection{Abwehrmaßnahmen}

\subsubsection{ZigBee-Verschlüsselung}
Die verwendeten ZigBee-Module der Firma Atmel unterstützen hardwareseitig die im ZigBee-Standard definierte 128-Bit 
AES-Verschlüsselung. Um diese einsetzen zu können, muss die momentan verwendete SerialNet-Firmware gegen eine eigene
Firmware ausgetauscht werden. Atmel stellt hierzu einen ZigBee Pro basiertes SDK zur Verfügung, mit Hilfe dessen
eine entsprechende Firmware mit vertretbarem Aufwand entwickelt werden kann. Hierbei bietet es sich an, einen an die
SerialNet-Firmware angelehnten AT-Befehlssatz zu implementieren, um alle bestehenden Komponenten wie den MANV-Connector
und die Firmware des ADuC-Mikrocontrollers weiterverwenden zu können.

\subsubsection{Befehlszähler}
Die ZigBee-Verschlüsselung schützt nicht gegen Replay-Attacken. Eine einfache aber effektive Methode zur Abwehr solcher
Angriffe ist die Verwendung einer Sequenznummer. Hierbei wird jeder Nachricht eine fortlaufende Nummer hinzugefügt. 
Der Empfänger führt nun eine Liste, in der für jeden Absender die zuletzt verwendete Sequenznummer enthalten ist.
Nachrichten, die eine niedrigere Sequenznummer enthalten werden verworfen. Diese Methode lässt sich mit geringem
Speicheraufwand ( O(N) für den Coordinator, O(1) für einen Endknoten) implementieren. Als Ort der Implementierung
bietet sich entweder die Firmware des ADuC-Mikrocontrollers sowie der MANV-Connector oder aber direkt die ZigBit-Firmware
an.

\subsubsection{Schutz auf Applikationsebene}
Alternativ zur Modifikation der ZigBit-Firmware besteht auch die Möglichkeit, auf Applikationsebene zu Verschlüsseln.
Sowohl für Java (OpenSSL) als auch für den ADuC-Mikrocontroller (WAKAN-Crypto-Toolkit) sind entsprechende Bibliotheken
verfügbar. Eine Implementierung auf Applikationsebene hätte den Vorteil, dass keine Änderung an der ZigBit-Firmware
notwendig wären, und daher ggf. einfacher durchzuführen ist. Von Nachteil ist jedoch, dass diese Lösung keinen Schutz
gegen den Beitritt fremder Teilnehmer in das Netzwerk bietet. Zwar können diese keine Befehle schicken oder Daten ausspähen,
allerdings ist es immer noch möglich, durch die Verwendung von gefälschten Absenderadressen eine Denial-Of-Service-Attacke
gegen das Netzwerk durchzuführen. 

\subsubsection{Schutz gegen Störsender}


    
\section{Verbesserung der Reichweite und Störsicherheit}
\subsection{Anderer Frequenzbereich}
\subsection{Richtantennen für Router}
\subsection{Adaptive Pfadkosten für Meshrouting}
