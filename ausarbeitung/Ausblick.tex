%%%------------------------------------------------ZUSAMMENFASSUNG UND AUSBLICK------------------------------------------------------

\chapter{Zusammenfassung und Ausblick}

\section{Zusammenfassung}
In dieser Arbeit wurde ein kabelloses Sensornetzwerk zur Überwachung von Patienten bei einem 
Massenanfall von Verletzten (MANV) entworfen und implementiert. Zunächst wurden 
verschiedene kabellose Sensornetzprotokolle vorgestellt und auf ihre Eignung als Grundlage für
die Patientenüberwachung in einem MANV überprüft. Bereits in diesem Schritt wurde die Entscheidung
getroffen, eine Lösung auf Basis des \emph{ZigBee}-Protokolles zu implementieren. Hierfür sprachen
der geringste Stromverbrauch aller verfügbaren Lösungen, die hohe Störsicherheit sowie die
gute Verfügbarkeit passender Hardware auf dem Markt. Es wurde weiterhin festgestellt, dass sich die
Marktsituation in den kommenden Jahren auf Grund der Einführung neuer Standards wie \emph{ANT+},
\emph{Bluetooth Low Energy} sowie \emph{Wireless USB} ändern könnte, so dass beim späteren Entwurf
insbesondere auf die Austauschbarkeit des Netzwerkprotokolles zu achten ist. Im Anschluss daran wurden
verschiedene aktuell auf dem Markt erhältliche Lösungen zur kabellosen Patientenüberwachungen betrachtet, 
und mit den Erkenntnissen aus der Analyse der verschiedenen Netzwerkprotokolle auf ihre Eignung untersucht.
Hierbei wurde festegestellt, dass alle betrachteten Lösungen eher für den Einsatz im Krankenhaus
gedacht sind, und für den Einsatz in während eines MANVs nicht geeignet sind. Weiterhin wurden verwandte
Projekte, die sich ebenfalls mit der technischen Patientenversorgung während eines MANVs beschäftigen
vorgestellt, und die Ähnlichkeiten und Unterschiede gezeigt.\\
Mit den aus der Betrachtung der Stand der Technik gewonnen Erkenntnissen konnte nun zur Analyse 
übergegangen werden. Hier wurde zunächst eine Reihe von Anforderungen erarbeitet, die die zu entwerfende 
Lösung erfüllen soll. Ausserdem wurden für den späteren Einsatz ausgewählte \emph{ZigBee}-Hardware 
der Firma \emph{Atmel} untersucht, und es wurde eine Reihe von Eigenheiten im Kommunikationsprotokoll 
zwischen \emph{ZigBee}-Modul und Mikrokontroller entdeckt, für die jeweils eine passende Lösung in 
allgemeiner Form vorgestellt wurde.  Ausserdem wurden kurz die zwei verschiedenen Arten, 
den Energiesparmodus der Module aufzurufen, erläutert, und sich dafür entschieden, dass der manuelle 
Aufruf hier die bessere Lösung ist.\\
Nach der Analyse folgte der Entwurf eines Systems zur Lösung der gestellten Aufgabe. Dieser erfolgte in 3 
Schritten. Zunächst erfolgte der Entwurf des Gesamtsystems im groben. In Hinblick auf die in der Analyse
gewünschten Austauschbarkeit des Funkprotokolls wurde entschieden, alle Designentscheidungen zu kapseln und
so das System aus einzelnen, austauschbaren Komponenten aufzubauen. Im groben wurde hierbei zwischen
einer Kontrollsoftware, der sogenannten \emph{MANVSuite}, der Hardware der Sensoren, in Form einer 
Test- und Entwurfplatine (\emph{MANVNode}), einem \emph{USB-Stick} zur Verbindung zwischen Netzwerk und Computer
sowie dem sogenannten \emph{MANVConnector}, der die Schnittstelle zwischen Hard- und Software bildet und zudem
alle Hardwarespezifischen Designentscheidungen kapselt. Die \emph{MANVSuite} wurde in einer
zu dieser Diplomarbeit parallel laufenden Studienarbeit von Herrn cand. inform Jan Tepelmann genauer entworfen
und implementiert.\cite{Jan}\\
Nach dem groben Entwurf folgt der Entwurf im feinen. Hierbei wurden Hardware (\emph{MANVNode} und \emph{USB-Stick})
sowie Software (\emph{MANVConnector} und Firmware der \emph{MANVNode}) getrennt betrachtet.\\
Auf Grundlage dieses Entwurfes wurde nun eine Implementierung durchgeführt. Die Vorgehensweise, welche Probleme
dabei aufgetreten sind, und wie diese gelöst wurden ist Thema des Kapitels~\ref{implementierung}.
Hier wird unter anderem beschrieben, welche Probleme bei der Bestückung der Platine der \emph{MANVNode} aufgetreten
sind, wie der Zugriff auf die \emph{UART}-Schnittstelle des \emph{ADuC}-Mikrocontrollers in der Firmware
realisiert wurde, wie von \emph{Java} aus auf den seriellen Port zugegriffen wird und wie die automatische
Erkennung des \emph{USB-Sticks} auf dem Betriebssystem \emph{Linux} realisiert wurde.\\
Nachdem die Implementierung abgeschlossen war, folgte eine Reihe von umfassenden Tests. Ziel hierbei war es,
festzustellen, wie gut die vorgestellte Lösung die in der Analyse aufgestellten Anforderungen erfüllt. Neben
Reichweite, Stromverbrauch und Erweiterbarkeit des Funknetzwerkes wurde hierzu auch die Leistungsfähigkeit der 
Software in Hinblick auf Anzahl der Teilnehmer, Austauschbarkeit des \emph{MANVConnectors} und Interoperabilität
zwischen \emph{MANVSuite}, \emph{MANVConnector} und Sensornetzwerkkeit intensiv untersucht. Alle Tests verliefen
positiv, und die entworfene Lösung konnte in fast allen Punkten die Anforderungen, die in Abschnitt~\ref{anforderungen}
aufgestellt wurden, erfüllen. Lediglich bei der Datensicherheit innerhalb des \emph{ZigBee}-Netzwerkes mussten
Einschränkungen hingenommen werden, da die Firmware des Herstellers \emph{Atmel} nicht den versprochenen
Funktionsumfang bereit stellen konnten. Dies wird in Abschnitt~\ref{Sicherheit} genauer untersucht, hier werden
auch mehrere Vorschläge gemacht, wie dieses Problem gelöst werden kann. Die Arbeit wird abgerundet durch ein
umfangreiches Softwarepaket inkl. Quellcode und Dokumentation, die die zeitnahe Realisierung als Prdodukt ermöglichen.

\section{Sicherheit des Netzwerkes}
\label{Sicherheit}
\subsection{Aktueller Stand}
Grundsätzlich unterstützt ZigBee eine 128-Bit AES-Verschlüsselung. Die SerialNet-Firmware der in dieser Diplomarbeit
eingesetzten ZigBee-Module der Firma Atmel bietet jedoch in der aktuell vorliegenden Version (1.8.0) keine Möglichkeit,
diese Verschlüsselung zu verwenden. Laut Auskunft des E-Mail Supportes der Firma Atmel ist dieses Feature auch für
zukünftige Versionen nicht geplant.\\
\\
Da das Thema Sicherheit -- gerade für ein medizinisches Netzwerk -- natürlich eine nicht zu vernachlässigende Rolle spielt,
An dieser Stelle sollen einigen Anregungen gegeben werden, wie eine Sicherheitskonzept für das entworfene Netzwerk aussehen 
könnte.

\subsection{Angriffsszenarien}

\subsubsection{Ausspähen von Patientendaten}
Aktuell werden über das Netzwerk Vitaldaten von Patienten (Puls und Atmung) pseudonym versendet. Als Pseudonym dient 
hierbei die Hardware-Adresse des versendenden ZigBee-Moduls. Passives Mitschneiden der Daten ist so lange kein Problem,
so lange sich keine Zuordnung zwischen der Identität des Patienten und der Hardware-Adresse des verwendeten Moduls
herstellen lässt. Hierfür ist eine direkte räumliche Nähe zwischen Patient und Angreifer notwendig. 
Während eines MANV-Szenarios sollte dies jedoch aktiv durch die Rettungskräftem, insbesondere durch die Absperrung durch
die Polizei vor Ort verhindert werden. Sollte es einem Angreifer gelingen, nahe genug an das entsprechende Modul
heranzukommen, um eine Zuordnung vorzunehmen, ist sowieso davon auszugehen, dass er auch in der Lage wäre, Bild oder
Videoaufnahmen des Patienten herzustellen, was im Zweifelsfall eine deutlich größeren Eingriff in die Privatsphäre 
darstellen würde.

\subsubsection{Aktiver Angriff gegen das Netzwerk}
Ein größeres Problem als durch das passive Belauschen des Datenverkehrs geht von aktiven Angriffen gegen das Netzwerk
aus. Innerhalb eines unverschlüsselten Netzwerkes kann ein Angreifer die Identität eines beliebigen ZigBee-Moduls 
übernehmen; er hat somit die Möglichkeit entweder falsche Vitaldaten oder aber Befehle an die einzelne Sensoren zu
versenden. Dies könnte im schlimmsten Falle bedeuten, dass Alarme im Netzwerk verloren gehen, und somit Patienten
zu Schaden kommen. Da selbst bei aktivierter ZigBee-Verschlüsselung kein Schutz gegenüber Replay-Attacken besteht hätte
ein Angreifer weiterhin die Möglichkeit, eine aufgefangene Statusnachricht oder einen aufgegangen Befehl beliebig oft
wieder in das Netzwerk einzubringen und so eine Störung zu erreichen.

\subsubsection{Störung des Netzwerkes}
Ein Angreifer hat mehrere Möglichkeiten, ein bestehendes Netzwerk empfindlich zu stören oder sogar jegliche Kommunikation
zum erliegen zu bringen. Im Falle eines unverschlüsselten Netzwerkes kann des einfach dadurch geschehen, in dem
der Angreifer die Identität des Netzwerkkoordinators anzunehmen, und somit die Statusnachrichten ins Leere zu leiten. Eine
andere Möglichkeit besteht darin, das Netzwerk mit Paketen zu überfluten, und so den Datenverkehr der anderen Teilnehmer
zu blockieren. Aber auch bei Einsatz von Verschlüsselung ist eine Störung des Netzwerkes möglich. Hierzu muss ein Angreifer
einfach mit sehr großer Sendeleistung auf allen ZigBee-Frequenzen senden, um diese somit zu blockieren. 

\subsection{Abwehrmaßnahmen}

\subsubsection{ZigBee-Verschlüsselung}
Die verwendeten ZigBee-Module der Firma Atmel unterstützen hardwareseitig die im ZigBee-Standard definierte 128-Bit 
AES-Verschlüsselung. Um diese einsetzen zu können, muss die momentan verwendete SerialNet-Firmware gegen eine eigene
Firmware ausgetauscht werden. Atmel stellt hierzu einen ZigBee Pro basiertes SDK zur Verfügung, mit Hilfe dessen
eine entsprechende Firmware mit vertretbarem Aufwand entwickelt werden kann. Hierbei bietet es sich an, einen an die
SerialNet-Firmware angelehnten AT-Befehlssatz zu implementieren, um alle bestehenden Komponenten wie den MANV-Connector
und die Firmware des ADuC-Mikrocontrollers weiterverwenden zu können.

\subsubsection{Befehlszähler}
Die ZigBee-Verschlüsselung schützt nicht gegen Replay-Attacken. Eine einfache aber effektive Methode zur Abwehr solcher
Angriffe ist die Verwendung einer Sequenznummer. Hierbei wird jeder Nachricht eine fortlaufende Nummer hinzugefügt. 
Der Empfänger führt nun eine Liste, in der für jeden Absender die zuletzt verwendete Sequenznummer enthalten ist.
Nachrichten, die eine niedrigere Sequenznummer enthalten werden verworfen. Diese Methode lässt sich mit geringem
Speicheraufwand ( O(N) für den Coordinator, O(1) für einen Endknoten) implementieren. Als Ort der Implementierung
bietet sich entweder die Firmware des ADuC-Mikrocontrollers sowie der MANV-Connector oder aber direkt die ZigBit-Firmware
an.

\subsubsection{Schutz auf Applikationsebene}
Alternativ zur Modifikation der ZigBit-Firmware besteht auch die Möglichkeit, auf Applikationsebene zu Verschlüsseln.
Sowohl für Java (OpenSSL) als auch für den ADuC-Mikrocontroller (WAKAN-Crypto-Toolkit) sind entsprechende Bibliotheken
verfügbar. Eine Implementierung auf Applikationsebene hätte den Vorteil, dass keine Änderung an der ZigBit-Firmware
notwendig wären, und daher ggf. einfacher durchzuführen ist. Von Nachteil ist jedoch, dass diese Lösung keinen Schutz
gegen den Beitritt fremder Teilnehmer in das Netzwerk bietet. Zwar können diese keine Befehle schicken oder Daten ausspähen,
allerdings ist es immer noch möglich, durch die Verwendung von gefälschten Absenderadressen eine Denial-Of-Service-Attacke
gegen das Netzwerk durchzuführen. 

\section{Verbesserung der Reichweite und Störsicherheit}
\subsection{Anderer Frequenzbereich}
In dieser Arbeit wurden lediglich \emph{ZigBee}-Module, die im 2,4~GHz-\emph{ISM-Band} operieren, untersucht. Es gibt
jedoch auch Module, die das 868 (Europa) bzw. 915~MHz-Band (USA) nutzen. Diese versprechen eine bessere Durchdringung
von Hindernissen und damit eine bessere Reichweite und Störsicherheit. Diese Verbesserung kommt zum Preis von einer
niedrigeren Datenrate (20 bzw. 40~kbit/sec) und ist zudem Länderspezifisch. In wie weit diese Einschränkungen problematisch
sind wäre gesondert zu evaluieren, für den Einsatz in schwierigem Terrain könnte dieser Frequenzbereich jedoch einige
Vorteile bieten.

\subsection{Verwendung leistungsstärkerer \emph{ZigBee}-Module und Antennen}
Für diese Arbeit war der Stromverbrauch der Module eines der wichtigsten Kriterien. Daher wurden Module mit der geringsten
Leistungsaufnahme und damit auch der geringsten Sendeleistung für den Entwurf der Hardware verwendet. Neben diesen
Modulen sind auf Varianten mit deutlich stärkerer Sendeleistung erhältlich. Diese kann meist sogar noch zusätzlich durch
den Einsatz von externen Antennen verstärkt werden. Mit der entsprechenden Kombination aus Modul und Antenne lässt sich eine
Sendeleistung erreichen, die mehr als 10 mal so hoch ist, wie die der in dieser Arbeit verwendeten Module. Dies ist 
insbesondere für den Einsatz in \emph{ZigBee}-Routern interessant, da diese einfach mit einer größeren Stromquelle versehen
werden können als die einzelnen Sensoren. Es wäre daher interessant zu evaluieren, wie sie dich Verwendung stärkerer Module
sowie größerer Antennen in den Routern auf die erreichbare Reichweiten des Netzwerkes auswirkt.

\subsection{Genauere Betrachtung des \emph{Bluetooth-Low-Energy}-Standards}
Der \emph{Bluetooth-Low-Energy}-Standard verspricht eine interessante Alternative zum \emph{ZigBee}-Standard zu werden.
Der einzige Grund, weshalb in dieser Arbeit keine genauere Betrachtung erfolgen konnte war die Nichtverfügbarkeit
entsprechender Hardware zum Zeitpunkt der Anforderungsanalyse. Bereits im Februar 2010 wurde von der Firma 
\emph{Nordic Semiconductors} unter der Bezeichnung \emph{ISP091201} ein entsprechendes Hardwaremodul angekündigt,
dass jedoch bis jetzt (Stand Oktober 2010) nicht auf dem Markt erhältlich ist. Sobald entsprechende Module verfügbar
sind, sollte eine erneute Evaluation erfolgen. Beim Entwurf der hier vorgestellten Lösung wurde eine spätere Änderung
des Funkprotokolls eingeplant und bei der Implementierung entsprechend berücksichtigt.
