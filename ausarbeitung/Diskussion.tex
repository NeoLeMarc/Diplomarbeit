%%%------------------------------------------------DISKUSSION(BEWERTUNG DER ERGEBNISSE)---------------------------------------------

\chapter{Diskussion}\label{Diskussion}
In Abschnitt~\ref{Ergebnisse} wurde die Leistungsfähigkeit des Funknetzwerkes untersucht.
Es wurde gezeigt, dass sowohl Reichweite als auch Leistungsaufnahme die Anforderungen
dieser Arbeit erfüllen. Auch wurde die Erweiterbarkeit des Netzwerkes durch \emph{Router} gezeigt.\\
\\
Darüber hinaus wurde gezeigt, dass eine Verbindung des verwendeten \emph{ZigBee}-Modul
mit dem \emph{ADuC}-Mikrocontroller möglich ist, und dass das so entstehende Gesamtsystem
in der Lage ist, die zu überwachenden Vitaldaten zuverlässig an die Basistation zu übermitteln.
Es wurde ebenfalls gezeigt, dass es möglich ist, Befehle über das \emph{ZigBee}-Netzwerk
an den Mikrocontroller zu senden. Da der für diese Tests eingesetzte \emph{ADuC-7026}-Mikrocontroller
der selben Familie wie der in dem \emph{Erste-Hilfe-Sensor} verwendete \emph{Mikrocontroller} angehört,
können diese Ergebnisse ohne weiteres auf die Hardware des \emph{Erste-Hilfe-Sensors} übertragen werden; 
die Integrierbarkeit in diesen konnte somit gezeigt werden.\\
Die durch die Integration des \emph{ZigBit}-Moduls in den \emph{Erste-Hilfe-Senosr} erhöhte Leistungsaufnahme, 
insbesondere bei Lastspitzen (\emph{Peaks}), muss beachtet werden. Eventuell muss hier eine Anpassung der 
Stromversorgung des \emph{Erste-Hilfe-Sensors} erfolgen.
Die genaue Analyse auftretender Lastspitzen findet sich in Abschnitt~\ref{leistungsaufnahme_normalbetrieb}.
Insbesondere sollten hierbei auch die Ergebnisse in Abschnitt~\ref{leistungsaufnahme_sonderfaelle} beachtet 
werden, die auftretende Sonderfälle beschreiben.\\
\\
Softwareseitig konnte gezeigt werden, dass eine vollständige Kapselung aller hardwarespezifischen Teile
in einer austauschbaren Komponente möglich ist. Dies bietet den Vorteil, dass bei einem Wechsel 
der Technologie des Sensornetzwerks lediglich der \emph{MANVConnector} ausgewechselt werden muss,
alle anderen Teile können ohne Änderung weiterverwendet werden.\\
\\
Durch die in Abschnitt~\ref{anbindbarkeit_an_externe_software} durchgeführte schnelle Umsetzung eines Adapters 
zur Anbindung der \emph{MANVSuite} an eine Weboberfläche konnte gezeigt werden, dass durch die hohe Modularität
und Einsatz der Komponententechnologie \emph{Corba} ein hohes Maß an Flexibilität und Adaptierbarkeit der
Gesamtlösung erreicht wird und eine Anbindung an bestehende Systeme mit geringem Aufwand realisiert werden kann.
Eine genauere Analyse und Evaluation der \emph{MANVSuite} inkl. der beschriebenen Schnittstellen findet sich
in \cite{Jan}. In dieser verwandten Arbeit wird unter anderem gezeigt, wie eine Anbindung der \emph{MANVSuite}
an ein Mobiltelefon des Herstellers Nokia erfolgen kann.\\
\\
Als besonders positiv ist die Skalierbarkeit der Gesamtlösung zu bewerten: Zwar musste aus Kostengründen 
der praktische Skalierbarkeitstest auf 12 Module beschränkt werden.  Durch eine analytische Betrachtung konnte 
jedoch gezeigt werden, dass das \emph{ZigBee}-Netzwerk zumindest theoretisch in der Lage ist, mindestens 1250 Sensoren
zu unterstützen, was selbst
für ein sehr großes \emph{MANV}-Szenario ausreichen würde.  Entsprechende
praktische Tests müssen vor dem Praxiseinsatz natürlich nachgeholt werden.
Auch auf Softwareseite sind keine Lastprobleme zu erwarten, da
die die Verbindung der Softwarekomponenten untereinander netzwerkfähig ist; im Extremfall kann jede
Komponente auf einem eigenen Rechner betrieben werden. Hierbei skaliert die Anforderung an die Hardware mit
der Größe des zu verwaltenden Netzwerkes. Im Prinzip ist jedes linuxfähige Gerät mit mindestens 16~MB Speicher
und einer \emph{USB-Schnittstelle}, für die eine Implementierung der \emph{Java Virtual Machine}\footnote{Hierbei muss
es sich nicht zwangsläufig um die Implementierung von Oracle/Sun handeln. Die \emph{MANVSuite} kann z.B. auch
mit der \emph{JVM} des \emph{Eclipse-Projektes} ausgeführt werden.} verfügbar ist, in der Lage, als \emph{Server}
für das Sensornetzwerk zu dienen. Dies kann z.B. ein \emph{Android} basiertes Handy oder ein offener
\emph{DSL}-Router in Verbindung mit dem Betriebssystem \emph{OpenWRT} sein. Sollte keine \emph{JVM} für die
Plattform verfügbar sein, besteht sogar die Möglichkeit, dass nur der in \emph{Python} geschriebene,
hardwarespezifische Teil des \emph{MANVConnectors} auf dem Gerät selbst ausgeführt wird\footnote{Die Implementierung
in \emph{Python} erfolgte lediglich aus Gründen der Einfachheit. Sollte kein \emph{Python} verfügbar sein,
kann dieser Teil in jeder anderen Programmiersprache, die \emph{Socketverbindungen} erlaubt, neu implementiert werden.}. 
Alle restlichen Teile der \emph{MANVSuite} können nachgelagert auf einem Server im Internet ausgeführt werden; 
die Verbindung der beiden Teile erfolgt über eine einfache \emph{Socketverbindung}. Diese Lösung bietet sich insbesondere 
für die Überwachung einzelner Patienten als eine Art intelligenter Hausnotruf an. Eine nähere Betrachtung dieser
Anwendung erfolgt in \cite{AAL}.
\\
\\
Vom bisherigen Stand der Technik hebt sich die hier vorgestellte Lösung in vielen Punkten ab:

\nomenclature{Open Source}{Softwareprodukt, bei dem der Quellcode frei verfügbar ist.}
\nomenclature{GPIO}{General Purpose Input/Output}
\begin{itemize}
    \item{Konsequenter Einsatz offener Standards:} In der gesamten Arbeit wurde konsequent auf den
          Einsatz offener Standards gesetzt. Bei der Verwendung von externen Softwarepaketen wurde
          darauf geachtet, dass es sich um \emph{Open-Source} Produkte handelt. Zur Implementierung
          wurde keinerlei proprietäre Technologie eingesetzt.

    \item{Kapselung aller plattformspezifischen Designentscheidungen:} Alle Plattformabhängigen
          Designentscheidungen wie die Ansteuerung der \emph{ZigBee}-Module oder der
          physische Zugriff auf die serielle Schnittstelle sind bewusst austauschbar gehalten.
          Hierzu wurden diese in Komponenten gekapselt und die Schnittstelle zwischen diesen
          Komponenten wurde standardisiert. Bei Änderung einer dieser Entscheidungen (z.B.
          weil ein anderes Funkprotokoll oder ein anderes Betriebssystem zum Einsatz kommen soll)
          können diese einfach ausgetauscht werden, ohne den Rest des Systems anpassen zu 
          müssen.

    \item{Geringe Stückkosten:} Bei der Auswahl des Funkprotokolls wurde insbesondere auch auf den
          Preis der verfügbaren Hardwaremodule geachtet. Dieser bewegt sich selbst bei der
          Einzelabnahme der Module in einem Bereich unter 15\EUR{}. Auf den Einsatz teurer
          Basisbauteile wie z.B. \emph{TinyOS}-basierten Sensoren wurde konsequent verzichtet.

    \item{Geringe Leistungsaufnahme ("`Strombedarf"'):} Durch den Einsatz des von \emph{ZigBee}
          angebotenen Energiesparmodus konnte der mittlere Strombedarf auf einen Wert von ca. 
          1,86~mA gesenkt werden. Dies bedeutet eine Steigerung der Stromaufnahme des 
          \emph{Erste-Hilfe-Sensors} um ca. 10\%. 
          Laut der Analyse in Abschnitt~\ref{batterielaufzeit} ist selbst mit einer einfachen Knopfzelle
          eine Laufzeit von über 12 Stunden zu erreichen, was für einen \emph{MANV}-Einsatz mehr als
          ausreichend ist.

    \item{Einfache Implementierbarkeit:} Zur Anbindung des \emph{ZigBit}-Moduls an den \emph{Erste-Hilfe-Sensor}
          sind lediglich vier Signalleitungen notwendig: \emph{UART\_RX}, \emph{UART\_TX}, 
          \emph{UART\_CTS}\footnote{UART\_RTS sollte auf Seite des \emph{ZigBit}-Moduls auf \emph{GND}
              geführt werden.} sowie eine \emph{GPIO}\footnote{General Purpose Input/Output: 
              Generische Ein/Ausgabe-Leitung}-Leitung, 
              die mit dem \emph{RESET}-Eingan des
          \emph{ZigBit}-Moduls verbunden ist (Details zur Implementierung in die Platine können dem
          Schaltplan in Anhang~\ref{anhang_diagramme} entnommen werden). 

    \item{Plattformunabhängigkeit:} Durch den Einsatz plattformunabhängiger Programmiersprachen wie \emph{Java}
          und \emph{Python} sowie durch Virtualisierung\footnote{Die Virtualisierung dient dazu, die
              eingesetzte Software auf einer möglichst großen Palette von Systemen ausführbar zu machen. Sie
              soll aber auch den Installationsaufwand für die Software verringern, der insbesonders durch den
              eingesetzten Datenbankservern relativ hoch ist. Für den späteren produktiven Einsatz wäre zu
              überlegen, diesen Datenbankserver direkt als Bibliothek in die Software einzubetten. Eine
              genauere Diskussion dieses Themas findet sich in \cite{Jan}.} ist die vorgestellte Lösung auf
          einer breiten Palette von Systemen einsatzfähig.

    \item{Kompaktheit:} Die hier vorgestellte Prototyp-Platine besitzt eine Abmessung von nur
          50~mm auf 56~mm und bietet noch viel Potential zur weiteren Optimierung der Größe (z.B. durch
          die beidseitige Platzierung von Bauteilen oder Fertigung in 4-Schicht-Bauweise).

    \item{Anzahl Teilnehmer:} Laut der Betrachtung in Abschnitt~\ref{maximale_teilnehmerzahl} wird eine ausreichend
                              große Anzahl von Teilnehmern zu unterstützen. Dies reicht selbst für sehr große
                              \emph{MANV}-Szenarien gut aus.
\end{itemize}

In dieser Arbeit nicht betrachtet wurde das Thema Datensicherheit und Verschlüsselung, welches alleine bereits
umfangreich genug wäre, eine eigene Diplomarbeit zu rechtfertigen. Ursprünglich war geplant, die hardwareseitige
Verschlüsselung der \emph{ZigBit}-Module einzusetzen. Leider ist diese Funktion jedoch nicht in die 
\emph{SerialNet}-Firmware integriert und es standen auch keine fertig nutzbaren \emph{SerialNet}-Alternativen zur Verfügung. 
Daher wäre für die Verwendung der Verschlüsselung eine Neuimplementierung der
\emph{ZigBit-Firmware} notwendig, was den Rahmen dieser Arbeit sprengen würde. Daher beschränkt 
sich diese Arbeit darauf, in Abschnitt~\ref{Sicherheit} eine Analyse des Problems und 
Vorschläge zur Lösung vorzustellen. Diese können dann bei einer späteren Umsetzung in ein Produkt als Grundlage
dienen, die notwendigen Sicherheitsmaßnahmen zu implementieren.
