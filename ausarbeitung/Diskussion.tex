%%%------------------------------------------------DISKUSSION(BEWERTUNG DER ERGEBNISSE)---------------------------------------------

\chapter{Diskussion} 
In Abschnitt~\ref{Ergebnisse} wurde die Leistungsfähigkeit des Funknetzwerkes untersucht.
Es wurde gezeigt, dass sowohl Reichweite als auch Leistungsaufnahme die Anforderungen
dieser Arbeit erfüllen. Auch wurde die Erweiterbarkeit des Netzwerkes durch Router gezeigt.\\
\\
Darüber hinaus wurde gezeigt, dass eine Verbindung des verwendeten \emph{ZigBee}-Modul
mit dem \emph{ADuC}-Mikrocontroller möglich ist, und dass das so entstehende Gesamtsystem
in der Lage ist, die zu überwachenden Vitaldaten zuverlässig an die Basistation zu übermitteln.
Es wurde ebenfalls gezeigt, dass es möglich ist, Befehle über das \emph{ZigBee}-Netzwerk
an den Mikrocontroller zu senden. Da der für diese Tests eingesetzt \emph{ADuC-7026}-Mikrocontroller
der selben Familie wie der in dem Erste-Hilfe-Sensor verwendete Mikrocontroller handelt,
können diese Ergebnisse ohne weiteres auf die Hardware des Erste-Hilfe-Sensors übertragen werden, 
die Integrierbarkeit in diesen konnte somit gezeigt werden.\\
Durch die erhöhte Leistungsaufnahme, insbesondere bei Lastspitzen (Peaks), muss eine Anpassung
der Stromversorgung des Erste-Hilfe-Sensors erfolgen, wenn das \emph{ZigBee}-Modul integriert
werden soll. Die genaue Analyse auftretender Lastspitzen findet sich in Abschnitt~\ref{leistungsaufnahme_normalbetrieb}.
Insbesondere sollten hierbei auch die Ergebnisse in Abschnitt~\ref{leistungsaufnahem_sonderfaelle} beachtet 
werden, die auftretende Sonderfälle beschreiben.\\
\\
Softwareseitig konnte gezeigt werden, dass eine vollständige Kapselung aller Hardwarespezifischen Teile
in einer austauschbaren Komponente möglich ist. Dies bietet den Vorteil, dass alle bei einem Wechsel 
der Technologie des Sensornetzwerk lediglich der \emph{MANVConnector} ausgewechselt werden muss,
alle anderen Teile können ohne Änderung weiterverwendet werden.\\
\\
Durch die in Abschnitt~\ref{anbindbarkeit_an_externe_software} durchgeführte schnelle Umsetzung eines Adapters 
zur Anbindung der \emph{MANVSuite} an eine Weboberfläche konnte gezeigt werden, dass durch die hohe Modularität
und Einsatz der Komponententechnologie \emph{CORBA} ein hohes Maß an Flexibilität und Adaptierbarkeit der
Gesamtlösung erreicht, und eine Anbindung an bestehende Systeme mit geringem Aufwand realisiert werden kann.
Eine genauere Analyse und Evaluation der \emph{MANVSuite} inkl. der beschriebenen Schnittstellen findet sich
in \cite{Jan}. In dieser verwandten Arbeit wird unter anderem gezeigt, wie eine Anbindung der \emph{MANVSuite}
an ein Mobiltelefon des Herstellers Nokia erfolgen kann.\\
\\
Als besonders positiv ist die Skalierbarkeit der Gesamtlösung zu bewerten. Durch die konsequente Verwendung
gängiger Standards wie \emph{ZigBee}, \emph{Java}, \emph{SQL} und \emph{CORBA} wird ein hohes Maß an Flexibilität
erreicht. Die Anbindung der einzelnen Komponenten untereinander ist Netzwerkfähig, im Extremfall kann jede
Komponente auf einem eigenen Rechner betrieben werden. Hierbei skaliert die Anforderung an die Hardware mit
der Größe des zu verwaltenden Netzwerkes. Im Prinzip ist jedes linuxfähige Gerät mit mindestens 16MB Speicher
und einer USB-Schnittstelle, für die eine Implementierung der \emph{Java Virtual Machine}\footnote{Hierbei muss
es sich nicht zwangsläufig um die Implementierung von Oracle/Sun handeln. Die \emph{MANVSuite} kann z.B. auch
mit der \emph{JVM} des \emph{Eclipse-Projektes} ausgeführt werden.} verfügbar ist, in der Lage, als Server
für das Sensornetzwerk zu dienen. Dies kann z.B. ein \emph{Android} basiertes Handy oder ein offener
\emph{DSL}-Router in Verbindung mit dem Betriebssystem \emph{OpenWRT} sein. Sollte keine \emph{JVM} für die
Platform verfügbar sein, besteht sogar die Möglichkeit, dass nur der in \emph{Python} geschriebene,
hardwarespezifische Teil des \emph{MANVConnectors} auf dem Gerät selbst ausgeführt wird\footnote{Die Implementierung
in \emph{Python} erfolgte lediglich aus Gründen der Einfachheit. Sollte kein \emph{Python} verfügbar sein,
kann dieser Teil in jeder anderen Programmiersprache, die Socketverbindungen erlaubt, neu implementiert werden.}. 
Alle restlichen Teile der \emph{MANVSuite} können nachgelagert auf einem Server im Internet ausgeführt werden; 
Die Verbindung der beiden Teile erfolgt über eine einfache Socketverbindung. Diese Lösung bietet sich insbesondere 
für die Überwachung einzelner Patienten als eine Art intelligenter Hausnotruf an.\\
\\
Vom bisherigen Stand der Technik hebt sich die hier vorgestellte Lösung in vielen Punkte ab:

\begin{itemize}
    \item{Konsequenter Einsatz offener Standards:} In der gesamten Arbeit wurde konsequent auf den
          Einsatz offener Standards gesetzt. Bei der Verwendung von externen Softwarepaketen wurde
          darauf geachtet, dass es sich um \emph{Open-Source} Produkte handelt. Zur Implementierung
          wurde keinerlei proprietäre Technologie eingesetzt.

    \item{Kapselung aller Platformspezifischen Designentscheidungen:} Alle Platformabhängige
          Designentscheidungen wie die Ansteuerung der \emph{ZigBee}-Module oder der
          physische Zugriff auf die serielle Schnittstelle sind bewusst austauschbar gehalten.
          Hierzu wurden diese in Komponenten gekapselt und die Schnittstelle zwischen diesen
          Komponenten wurde standardisiert. Bei Änderung einer dieser Entscheidung (z.B.
          weil ein anderes Funkprotokoll oder ein anderes Betriebssystem zum Einsatz kommen soll)
          können diese einfach ausgetauscht werden, ohne den Rest des Systems anpassen zu 
          müssen.

    \item{Geringe Stückkosten:} Bei der Auswahl des Funkprotokolls wurde insbesondere auch auf den
          Preis der verfügbaren Hardwaremodule geachtet. Dieser bewegt sich selbst bei der
          Einzelabnahme der Modul in einem Bereich unter 15\EUR{}. Auf den Einsatz teurer
          Basisbauteile wie z.B. \emph{Tiny-OS} basierten Sensoren wurde konsequent verzichtet.

    \item{Geringe Leistungsaufnahme ("`Stromverbrauch"'):} Durch den Einsatz des von \emph{ZigBee}
          angebotenen Energiesparmodus konnte der mittlere Stromverbrauch auf einen Wert von ca. 
          1,86~mA gesenkt werden.

    \item{Einfache Implementierbarkeit:} Zur Anbindung des \emph{ZigBit}-Moduls an den Erste-Hilfe-Sensor
          sind lediglich vier Signalleitungen notwendig: \emph{UART\_RX}, \emph{UART\_TX}, 
          \emph{UART\_CTS}\footnote{UART\_RTS sollte auf Seite des \emph{ZigBit}-Moduls auf \emph{GND}
              geführt werden.} sowie einer \emph{GPIO}-Leitung, die mit dem \emph{RESET}-Eingan des
          \emph{ZigBit}-Moduls verbunden ist (Details zur Implementierung in die Platine kann dem
          Schaltplan in Anhang~\ref{anhang_diagramme} entnommen werden). 

    \item{Platformunabhängigkeit:} Durch den Einsatz platformunabhängiger Programmiersprachen wie \emph{Java}
          und \emph{Python} sowie durch Hardwarevirtualisierung ist die vorgestellte Lösung auf
          einer breiten Palette von Systemen einsatzfähig.

    \item{Platzsparend und Handlich:} Die hier vorgestellte Prototyplatine besitzt eine Abmessung von nur
          50~mm auf 56~mm und bietet noch viel Potential zur weiteren Optimierung der Größe (z.B. durch
          die beidseitige Platzierung von Bauteilen oder der Fertigung in 4-Schicht-Bauweise).
\end{itemize}

In dieser Arbeit nicht betrachtet wurde das Thema Datensicherheit und Verschlüsselung, welches alleine bereits
umfangreich genug wäre, eine eigene Diplomarbeit zu rechtfertigen. Ursprünglich war geplant, die hardwareseitige
Verschlüsselung der \emph{ZigBit}-Module einzusetzen. Leider ist diese Funktion jedoch nicht in die 
\emph{SerialNet}-Firmware integriert, so dass zur Verwendung der Verschlüsselung eine Neuimplementierung der
Firmware der \emph{ZigBit}-Firmware notwendig wäre, was den Rahmen dieser Arbeit sprengen würde. Daher wird
sich in dieser Arbeit darauf beschränkt, in Abschnitt~\ref{Sicherheit} eine kurze Analyse des Problems und einige
Vorschläge zur Lösung vorzustellen. Diese können dann bei einer späteren Umsetzung in ein Produkt als Grundlage
dafür dienen, die notwendigen Sicherheitsmaßnahmen zu implementieren.
