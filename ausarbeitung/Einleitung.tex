%%%------------------------------------------------EINLEITUNG-------------------------------------------------------------------------
\chapter{Einleitung}

In diesem Kapitel wird zunächst die Arbeit motiviert. Danach werden die Aufgaben und
Ziele definiert. Abschliessend wird die Gliederung der folgenden Kapitel erläutert.

\pagenumbering{arabic} % Seitennummerierung durch arabische Ziffern
\section{Motivation der Arbeit}
\nomenclature{MANV}{Massenanfall von Verletzten}
Unter dem Begriff „Massenanfall von Verletzten“ (MANV) versteht man ein Ereignis, bei dem innerhalb
kurzer Zeit eine große Anzahl von Verletzten versorgt werden muss. Hierbei mögen dem Leser zuerst
Vorfälle wie die Terroranschläge am 11. September 2001, die Reaktorkatastrophe von Tschernobyl oder
spektakuläre Naturereignisse wie der Hurrikan Katrina, die Flutkatastrophen der letzten Jahre oder die
Erdbeben in Haiti in den Sinn kommen.
Doch bereits Ereignisse geringeren Ausmaßes -- wie z.B. ein Auffahrunfall auf der Autobahn, ein
Zugunglück, eine Massenpanik bei einem Konzert oder starke Schneefälle, wie sie im vergangengen
Winter aufgetreten sind -- können den Rettungsdienst einer Region sehr schnell an die Grenzen seiner
Leistungsfähigkeit bringen.
Beispielhaft seien hier folgende Ereignisse der letzten Jahre und Jahrzehnte genannt:

\begin{itemize}
    \item 1980 –- Oktoberfestattentat: 13 Tote, 211 Verletzte
    \item 1988 –- Flugtagunglück von Ramstein: 70 Tote, 1000 Verletzte
    \item 1998 –- ICE-Katastrophe von Eschede: 101 Tote, 88 Verletzte
    \item 2000 –- Zugunglück von Brühl: 9 Tote, 149 Verletzte
    \item Dazu kommen jedes Jahr ca. 400.000 Verletzte im Straßenverkehr.
\end{itemize}

Im Falle eines Massenanfalls von Verletzten treffen vor Ort viele unterschiedliche Einheiten von
Rettungskräften aufeinander. Um die Zusammenarbeit dieser Kräfte überhaupt zu ermöglichen und ein
geordnetes Vorgehen zu unterstützen, gibt es klar definierte Abläufe und Verantwortlichkeiten. Zunächst
versuchen die Rettungskräfte, die Patienten außerhalb des unmittelbaren Gefahrenbereichs abzulegen.
An dieser Ablagestelle werden die Patienten vom Rettungsdienst übernommen, welcher auch bereits
lebensrettende Sofortmaßnahmen einleitet und eine Erstversorgung durchführt. Wenn möglich erfolgt
bereits hier eine erste Registrierung der Patienten (Fundort, Name usw.).
Nun werden die Patienten einer Triage, also der Einordnung in verschiedene Schweregrade,
unterzogen, um zu entscheiden, in welcher Reihenfolge diese Patienten versorgt werden.
Diese Einordnung ist notwendig, da meist nicht genügend Ressourcen (sowohl Material als auch
Personal) vorhanden sind, um sofort alle Patienten zu versorgen. Daher gilt das Ziel, dass möglichst
viele Patienten den Vorfall mit einem möglichst geringen Schaden überstehen. Um eine Zuteilung der
vorhandenen Ressourcen gemäß dieses Ziels zu gewährleisten, erfolgt eine Einordnung der Patienten in
4 Kategorien:

\begin{itemize}
    \item{Grün:} Patient ist ansprechbar und kann sich selbstständig vom Unfallort wegbewegen.
    \item{Gelb:} Patient hat Behandlungsbedarf und kann sich nicht mehr selbstständig vom Unfallort entfernen, 
                 befindet sich jedoch nicht in akuter Lebensgefahr.
    \item{Rot:}  Patient befindet sich in einem akut lebensbedrohlichen Zustand (z.B. Kreislaufstillstand, hoher
                 Blutverlust) und benötigt sofortige medizinische Behandlung.
    \item{Schwarz:} Patient ist verstorben.
\end{itemize}

Obwohl diese Vorgehensweise eine schnelle Versorgung von Schwerverletzten gewährleistet, stellt
sich eine Reihe von Problemen:

\begin{enumerate}
    \item Patienten können in die falsche Gruppe eingeordnet werden.
    \item Möglicherweise verschlechtert sich der Zustand eines Patienten nach seiner Einteilung in die Gruppe
          "`Gelb"' rapide, so dass er sofortiger Versorgung bedarf.
    \item Es besteht ein Konflikt zwischen sorgfältiger Untersuchung und zu wenig zur Verfügung stehender
          Zeit. Hierbei ist zu bemerken, dass die Sichtung immer noch ohne besondere technische Hilfsmittel
          stattfindet. Die Patienten werden mit Hilfe von einfachen Papierkarten markiert, die Überprüfung von
          Atmung, Puls und Blutdruck erfolgt durch einfache physische Methoden wie Anfassen, Hören und
          Beobachten.
\end{enumerate}

Insbesondere der zweite Punkt obiger Aufzählung könnte mit technischen Mitteln entschieden
verbessert werden. Da eine dauerhafte Überwachung nur deswegen nicht erfolgt, weil nicht genügend
Personal zur Verfügung steht, könnte dieses durch ein automatisches System ersetzt werden, welches
die Patienten überwacht und im Falle einer Zustandsänderung eine Alarmierung durchführt.
Zwar gibt es automatische Überwachungssysteme (sogenannte Monitore), die in Krankenhäusern oder
Rettungsfahrzeugen zur Patientenüberwachung eingesetzt werden. Allerdings sind diese Systeme für
den Einsatz in einem MANV-Szenario ungeeignet, da sie zu groß, zu teuer und zu kompliziert sind.

\section{Aufgaben und Ziele der Arbeit}
Ziel dieser Arbeit ist die Entwicklung und prototypische Implementierung eines
kabellosen Netzwerkes zur Überwachung von Patienten während eines \emph{MANVs}. Als
Ausgangsbasis hierfür dient der am IBT von Marc Jäger in \cite{Marc}
entwickelte \emph{Erste-Hilfe-Sensor}. Hierbei soll der Sensor in die Lage versetzt
werden, die Vitaldaten des angeschlossenen Patienten kabellos an eine zentrale
Überwachungsstation zu übertragen. Die zu entwickelnde Lösung soll eine
große Anzahl von Teilnehmern unterstützen, zugleich aber auch möglichst kostengünstig
sein. Darüber hinaus soll sie möglichst stromsparend sein, damit der \emph{Erste-Hilfe-Sensor}
auch längere Zeit über eine kleine Batterie (idealerweise eine Knopfzelle) betrieben 
werden kann.\\
\\
Parallel zu dieser Arbeit wird von Jan Tepelmann in \cite{Jan} eine 
Steuerrungssoftware ("`\emph{MANVSuite}"') für die Überwachungsstation des Netzwerkes 
entwickelt. Zu dieser Software soll eine Schnittstelle zur Verfügung gestellt werden, über die zum einen
die Vitaldaten ausgeliefert, zum anderen Befehle an die einzelnen Sensoren empfangen 
werden. Diese Schnittstelle soll netzwerkfähig sein. 
\nomenclature{MANVSuite}{Verwaltungssoftware des Sensornetzwerks.}

\section{Gliederung und Vorgehensweise der Arbeit}
Zunächst wird in Kapitel~\ref{Grundlagen} eine kurze Einführung in die Grundlagen,
die zum Verständnis dieser Arbeit wichtig sind, gegeben. Dies ist insbesondere deshalb
wichtig, da es sich um eine interdisziplinäre Arbeit zwischen Informatik und Elektrotechnik handelt.\\
\\
Im darauf folgenden Kapitel wird der aktuelle Stand der Technik genauerer Betrachtung
unterzogen. Hierbei werden zunächst kabellose Netzwerkprotokolle betrachtet und auf
ihre Eignung zur Lösung der gestellten Problemstellung untersucht. Hierbei erweist
sich das \emph{ZigBee}-Protokoll als besonders geeignet. Daraufhin erfolgt
ein kurzer Überblick über die Marktsituation bei Produkten zur kabellosen
Patientenüberwachung. Im Abschluss des Kapitels werden einige verwandte Projekte
vorgestellt und auf Ähnlichkeiten und Unterschiede untersucht.\\
\\
Aus dem Stand der Technik werden in Kapitel~\ref{Analyse} nun die Anforderungen an
die zu entwickelnde Lösung aufgestellt. Anschließend werden die Bauteile vorgestellt,
die zur Implementierung der Lösung verwendet werden. Diese werden auf Ihre Eignung
untersucht und es werden zu beachtende Eigenheiten herausgearbeitet. Für diese
Eigenheiten werden verschiedene Lösungsmöglichkeiten vorgestellt und analysiert,
die im späteren Verlauf der Arbeit als Grundlage für Entwurf und Implementierung
dienen werden.\\
\\
Der eigentliche Entwurf wird in Kapitel~\ref{Entwurf} beschrieben. Hier wird als Erstes 
ein grober Entwurf des Gesamtsystems durchgeführt, in dem zunächst die zentralen
Komponenten der Lösung identifiziert werden. Im weiteren Verlauf des Kapitels
wird der detailierte Entwurf jeder dieser Komponenten dargestellt, so dass diese 
implementiert werden können. Bei dem Entwurf wird zwischen \emph{Hardware}
und \emph{Software} unterschieden.\\
\\
Kapitel~\ref{Implementierung} behandelt die praktische Umsetzung des theoretischen Entwurfs.
Ein Schwerpunkt bildet
hier die Beschreibung der praktischen Probleme, die bei der Umsetzung aufgetreten sind, 
sowie deren Lösung. Zu diesem Kapitel gehört eine Reihe von Entwurfszeichnungen, Schaltplänen
und Platinenlayouts, die sich in Anhang~\ref{anhang_diagramme} finden.\\
\nomenclature{CD}{Compact Disc}
\\
In Kapitel~\ref{Ergebnisse} wird sodann eine genaue Evaluierung der entwickelten Lösung 
in Hinblick auf Erfüllung der in Kapitel~\ref{Analyse} aufgestellten Anforderungen, 
durchgeführt; die Ergebnisse dieser Evaluation werden dann im Kapitel~\ref{Diskussion}
diskutiert.\\
\\
Abschließend wird der Inhalt dieser Arbeit in Kapitel~\ref{Ausblick} zusammengefasst,
und es werden im Rahmen eines Ausblicks Vorschläge gemacht, wie die in dieser Arbeit entwickelte
Lösung weiter ausgebaut und verbessert werden kann.\\
\\
Abgerundet wird die Arbeit durch einen umfangreichen Anhang, in dem neben Schaltplänen und 
Entwurfszeichnungen auch eine Hilfestellung zur Umsetzung der in der Arbeit entwickelten
Lösung in ein fertiges Produkt geliefert wird.
