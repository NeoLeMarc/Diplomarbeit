%%%------------------------------------------------EINLEITUNG-------------------------------------------------------------------------
\chapter{Einleitung}
\pagenumbering{arabic} % Seitennummerierung durch arabische Ziffern
\section{Motivation der Arbeit}

Der Begriff „Massenanfall von Verletzten“ (MANV) versteht man ein Ereignis, bei dem innerhalb
kurzer Zeit eine große Anzahl von Verletzten versorgt werden muss. Hierbei mögen einem zuerst
Vorfälle wie die Terroranschläge am 11. September 2001, die Reaktorkatastrophe von Tschernobyl oder
spektakuläre Naturereignisse wie der Hurrikan Katrina, die Flutkatastrophen der letzten Jahre oder die
Erdbeben in Haiti in den Sinn kommen.
Doch bereits Ereignisse geringeren Ausmaßes - wie z.B. ein Auffahrunfall auf der Autobahn, ein
Zugunglück, eine Massenpanik bei einem Konzert oder starke Schneefälle, wie sie in den letzten
Monaten aufgetreten sind - können den Rettungsdienst einer Region sehr schnell an die Grenzen seiner
Leistungsfähigkeit bringen.
Beispielhaft seien hier folgende Ereignisse der letzten Jahre und Jahrzehnte genannt:

\begin{itemize}
    \item 1980 – Oktoberfestattentat: 13 Tote, 211 Schwerverletzte
    \item 1988 – Flugtagunglück von Ramstein: 70 Tote, 1000 Verletzte
    \item 1998 – ICE-Katastrophe von Eschede: 101 Tote, 88 Schwerverletzte
    \item 2000 – Zugunglück von Brühl: 9 Tote, 149 Verletzte
    \item Dazu kommen jedes Jahr ca. 400.000 Verletzte im Straßenverkehr.
\end{itemize}

Im Falle eines Massenanfalls von Verletzten treffen vor Ort viele unterschiedliche Einheiten von
Rettungskräften aufeinander. Um die Zusammenarbeit dieser Kräfte überhaupt zu ermöglichen und ein
geordnetes Vorgehen zu unterstützen gibt es klar definierte Abläufe und Verantwortlichkeiten. Zunächst
versuchen die Rettungskräfte die Patienten außerhalb des unmittelbaren Gefahrenbereichs abzulegen.
An dieser Ablagestelle werden die Patienten vom Rettungsdienst übernommen, welcher auch bereits
Lebensrettende Sofortmaßnahmen einleitet und eine Erstversorgung durchführt. Wenn möglich erfolgt
bereits hier eine erste Registrierung der Patienten (Fundort, Name usw.).
Nun werden die Patienten einer Triage, also der Einordnung in verschiedene Schweregrade,
unterzogen, um zu entscheiden, in welcher Reihenfolge diese Patienten versorgt werden.
Diese Einordnung ist notwendig, da meist nicht genügend Ressourcen (sowohl Material als auch
Personal) vorhanden sind, um sofort alle Patienten zu versorgen. Daher gilt das Ziel, dass möglichst
viele Patienten den Vorfall mit einem möglichst geringen Schaden überstehen. Um eine Zuteilung der
vorhandenen Ressourcen gemäß dieses Ziels zu gewährleisten, erfolgt eine Einordnung der Patienten in
4 Kategorien. 

Obwohl diese Vorgehensweise eine schnelle Versorgung von Schwerverletzten gewährleistet, stellen
sich eine Reihe von Problemen:

\begin{enumerate}
    \item Patienten können in die falsche Gruppe zugeordnet werden.
    \item Möglicherweise verschlechtert sich der Zustand eines Patienten nach seiner Einteilung in die Gruppe
          T2 rapide, so dass dieser sofortiger Versorgung bedarf.
    \item Es besteht ein Konflikt zwischen sorgfältiger Untersuchung und zu wenig zur Verfügung stehender
          Zeit.  Hierbei ist zu bemerken, dass die Sichtung immer noch ohne besondere technische Hilfsmittel
          stattfindet. Die Patienten werden mit Hilfe von einfachen Papierkarten markiert, die Überprüfung von
          Atmung, Puls und Blutdruck erfolgt durch einfache physische Methoden wie Anfassen, Hören und
          Beobachten.
\end{enumerate}

Insbesondere der zweite Punkt obiger Aufzählung könnte mit technischen Mitteln entschieden
verbessert werden. Da eine dauerhafte Überwachung nur deswegen nicht erfolgt, weil nicht genügend
Personal zur Verfügung steht, könnten diese durch ein automatisches System ersetzt werden, welches
die Patienten überwacht, und im Falle einer Zustandsänderung eine Alarmierung durchführt.
Zwar gibt es automatische Überwachungssysteme (sogenannte Monitore), die in Krankenhäusern oder
Rettungsfahrzeugen zur Patientenüberwachung eingesetzt werden, allerdings sind diese Systeme für
den Einsatz in einem MANV-Szenario ungeeignet, da sie zu groß, zu teuer und zu kompliziert sind.

\section{Aufgaben und Ziele der Arbeit}
Ziel dieser Arbeit ist die Entwicklung und prototypische Implementierung eines
kabellosen Netzwerkes zur Überwachung von Patienten während eines MANVs. Als
Ausgangsbasis hierfür dient der von Herrn Dr. Marc Jäger in \ref{diss_marc}
entwickelte Erste-Hilfe-Sensor. Hierbei soll der Sensor in die Lage versetzt
werden, die Vitaldaten des angeschlossenen Patienten kabellos an eine zentrale
Überwachungsstation zu übertragen. Die zu entwickelnde Lösung soll eine möglichst
große Anzahl von Teilnehmern unterstützen, zu gleich aber auch möglichst kostengünstig
sein. Darüber hinaus soll sie möglichst stromsparend sein, damit der Erste-Hilfe-Sensor
auch längere Zeit über eine möglichst kleine Batterie betrieben werden kann.\\
\\
Parallel zu dieser Arbeit wird von Herrn cand. inform Jan Tepelmann eine 
Steurungssoftware für die Überwachungsstation des Netzwerkes entwickelt. Zu dieser
Software soll eine Schnittstelle zur Verfügung gestellt werden, über die zum einen
die Vitaldaten ausgeliefert, zum anderen Befehle an die einzelnen Sensoren empfangen 
werden. Diese Schnittstelle ist mittels CORBA zu realisieren.

\section{Gliederung und Vorgehensweise der Arbeit}
...
