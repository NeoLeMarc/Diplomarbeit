

%%%------------------------------------------------ERGEBNISSE(KEINE BEWERTUNG)----------------------------------------------------

\chapter{Ergebnisse}
\section{Leistungsaufnahme des ZigBit-Moduls}
Zur Bestimmung der Leistungsaufnahme, Umgangssprachlich auch Stromverbrauch genannt, wurden mehrere Messungen durchgeführt.
Da der Stromverbrauch insbesondere bei Einsatz des Energiesparmoduses sehr stark schwankt, sind herkömmliche Messungen
z.B. über ein Multimeter nicht geeignet. Daher wurde eine indirekte Messung über den Spannungsabfall über einem 
Shunt-Widerstand durchgeführt. Dieser wurde in die Stromversorgung des ZigBit-Moduls eingebracht. Der Spannungsabfall
über diesem Widerstand wurde nun sowohl mit einem Oszilloskop als auch mit einem Multimeter gemessen. Hierzu wurde zunächst
die Form des Spannungsverlaufes analysiert. Dann wurden einzelne Messpunkte innerhalb des Spannungsverlaufs bestimmt,
für die nun der Wert der abfallenden Spannung bestimmt wurde. Über das Ohmsche-Gesetz konnte nun aus der abfallenden
Spannung die Stromstärke bestimmt werden. Für die einzelnen Messungen wurde ein Oszilloskop der Marke \emph{Tektronix}, 
Modell \emph{TDS2002B} verwendet. Die Messungen wurden zusätzlich mit einem Multimeter der Marke \emph{Fluke} verifiziert,
um Kalibrationsfehler des Oszilloskops auszuschliessen. Als Shunt wurde ein Widerstand mit einem Wert von $20,1\Omega$
verwendet.

\subsection{Leistungsaufnahme im Normalbetrieb}
\Abbildungps{!htb}{0.75}{oszi/normalbetrieb_router.jpg}{oszi_normalbetrieb_router}{Router oder Koordinator im Normalbetrieb. 
    Der Stromverbrauch beträgt hierbei konstant 23,4~mA, was ca. 70~mW entspricht.} 
\Abbildungps{!htb}{0.75}{oszi/normalbetrieb_client.jpg}{oszi_normalbetrieb_client}{Endknotens im Normalbetrieb ohne 
    Energiesparmodus. Auffällig sind die Peaks, in denen sich der Knoten im Empfangsmodus befindet. Die Peaks haben eine 
    Länge von 40ms und einen Betrag von 23,4~mA. Ausserhalb eines Peaks beträgt der Stromverbrauch 10,7~mA. Insgesamt ergibt 
    sich ein Mittelwert von 11,18~mA, was ca. 33,54~mW entspricht. Dies ist weniger als die Hälfte des Stromverbrauchs
    eines Routers oder Koordinators.} 
\Abbildungps{!htb}{1}{diagramme/spannungsverlauf_client_normal.pdf}{spannungsverlauf_client_normal}{Analyse des 
    Spannungsverlaufs eines Clients im Normalbetrieb.}
\Abbildungps{!htb}{0.75}{oszi/empfangen_details.jpg}{oszi_empfangen_details}{Detailaufnahme eines Empfangsmodus-Peaks.
    Erkennbar ist die Impulsbreite von 40ms.}

Als Normalbetriebn wurde der Zustand angenommen, in dem das \emph{ZigBit}-Modul einem \emph{ZigBee}-Netzwerk beigetreten
ist, und der Energiesparmodus nicht zur Anwendung kommt. Bei der Messung wurde schnell klar, dass sich hierbei die
Funktionsweise von eines \emph{FFD} (also Router und Connector) drastisch von der eines \emph{RFD}s (also eines Clients) 
unterscheiden. Ein \emph{FFD} befindet sich immer im Empfangsmodus, d.h. es hat immer eine maximale Leistungsaufnahme. 
Die entsprechende Messung ist in Abbildung~\ref{oszi_normalbetrieb_router} zu erkennen. Hier wurde ein Spannungsabfall
von 473~mV gemessen, was bei dem verwendeten Messwiderstand einem Strom von 23,5mA oder ca. 70mW entspricht.\\
Komplett anders ist hingegen das Verhalten eines Clients. Betrachtet man die Messung in
Abbildung~\ref{oszi_normalbetrieb_client} ist eine Basislinie von 215~mV mit periodischen Peaks erkennbar. 
In Abbildung~\ref{oszi_empfangen_details} ist eine Detailmessung eines solchen Peaks zu sehen.  Gut zu Erkennen ist zum 
einen der Betrag des Spannungsabfalls von rund 473~mV sowie einer Dauer von genau 40ms. Zwischen den Peaks kehrt der 
Spannungabfall für genau eine Sekunde auf den Wert der Basislinie, also 215~mV zurück, auf die wieder ein Peak folgt.
Der Wert von 473~mV legt die Vermutung nahe, dass es sich bei den Peaks um einen periodischen Empfangsvorgang handelt,
d.h. das das \emph{ZigBit}-Modul jede Sekunde für genau 40ms Empfangsbereit ist. Diese Vermutung konnte durch weitere
Tests bestätigt werden. Zur Bestimmung des Stromverbrauchs eines Clients ergibt sich daher folgende Rechnung:

\begin{align}
    U_{Client} &= \frac{t_{Basislinie} \cdot U_{Basislinie} + t_{Peak} \cdot U_{Peak}}{t_{Basisline} + t_{Peak}}
   \label{eqn:UClient}\\
    I_{Client} &= \frac{U_{Client}}{R_{Shunt}}
    \label{eqn:IClient}
\end{align}

Setzt man nun folgende Werte 

\begin{center}
    $t_{Basislinie} = 1000\,\text{ms}$,\quad
    $t_{Peak}       = 40\,\text{ms}$,\quad
    $U_{Basislinie} = 215\,\text{mV}$,\quad
    $U_{Peak}       = 473\,\text{mV}$
\end{center}

in Formel~\ref{eqn:UClient} ein, folgt für den Spannungsabfall

\begin{align*}
    &U_{Client} = \frac{1000\,\text{ms} \cdot 215\,\text{mV} + 40\,\text{ms} \cdot 470\text{mV}}{1000\,\text{ms} + 40,\text{ms}} = 224,9\,\text{mV}
\end{align*}

wodurch sich mit einem Widerstand von 

\begin{center}$
    R_{Shunt} = 20,1\,\Omega
    $
\end{center}

ein Wert von 11,19~mA für $I_{Client}$ ergibt. Dies entspricht bei einer Versorgungsspannung von 3,0~V einer Leistungsaufnahme
von 33,57~mW.






\subsection{Leistungsaufnahme bei Verwendung des Energiesparmodus}
\Abbildungps{!htb}{0.75}{oszi/erwachen_1.jpg}{oszi_erwachen}{Dieser Client ist gerade aus dem Energiesparmodus erwacht. 
    Erkennbar ist, dass während der Schlafphase der Stromverbrauch praktisch auf 0 sinkt. Direkt nach dem Erwachen
    wird ein Empfangsvorgang durchgeführt; die danach folgende Phase ist analgo zu Abbildung~\ref{oszi_normalbetrieb_client}.
    Gut zu Erkennen ist, dass der Energiesparmodus praktisch keinen zusätzlichen Overhead bedeutet.}
\Abbildungps{!htb}{1}{diagramme/spannungsverlauf_client_energiesparmodus.pdf}{spannungsverlauf_client_energiesparmodus}{Analyse des Spannungsverlaufs eines Clients bei Nutzung des Energiesparmodus.}

\Abbildungps{!htb}{0.75}{oszi/powersave_periodic.jpg}{oszi_powersave_periodic}{Dieser Client nutzt das periodische Aufrufen
    des Energiesparmodus. Die Schlafphase hat hier eine Länge von 5 Sekunden, die Wachphase ist genau 1 Sekunde lang. Der
    Stromverbrauch beträgt im Mittel nur noch 1,86mA, was ca. 5,58mW entspricht.}


\subsection{Sonderfälle}
\Abbildungps{!htb}{0.75}{oszi/not_joined.jpg}{oszi_not_joined}{Bei diesem Client wurde der Empfangsmodus deaktiviert. Der 
    Stromverbrauch beträgt konstant 10,7mA.}
\Abbildungps{!htb}{0.75}{oszi/joined.jpg}{oszi_joined}{Beitrittsvorgang eines Clients in ein Netzwerk. Bevor der Client dem 
    Netzwerk beigetreten ist, befindet er sich durchgehend im Empfangsmodus und benötigt 23,4mA. Sobald er dem Netzwerk
    beigetreten ist, entspricht entspricht der Stromverbrauch dem in Abbildung~\ref{oszi_normalbetrieb_client} dargestellten
    Verhalten.} 
\Abbildungps{!htb}{0.75}{oszi/joined2.jpg}{oszi_connection_lost}{Dieser Client hat kurzzeitig die Verbindung zum Netzwerk 
    verloren. In dieser Zeit befand er sich im Dauerempfangsmodus und hat 23,4mA. benötigt.}



\section{Reichweite}
