

%%%------------------------------------------------ERGEBNISSE(KEINE BEWERTUNG)----------------------------------------------------

\chapter{Ergebnisse}
\section{Leistungsaufnahme}
\Abbildungps{!htb}{0.75}{oszi/normalbetrieb_router.jpg}{oszi_normalbetrieb_router}{Router oder Koordinator im Normalbetrieb. 
    Der Stromverbrauch beträgt hierbei konstant 23,4mA, was ca. 70mW entspricht.} 
\Abbildungps{!htb}{0.75}{oszi/normalbetrieb_client.jpg}{oszi_normalbetrieb_client}{Endknotens im Normalbetrieb ohne 
    Energiesparmodus. Auffällig sind die Peaks, in denen sich der Knoten im Empfangsmodus befindet. Die Peaks haben eine 
    Länge von 40ms und einen Betrag von 23,4mA. Ausserhalb eines Peaks beträgt der Stromverbrauch 10,7mA. Insgesamt ergibt 
    sich ein Mittelwert von 11,18mA, was ca. 33,54mW entspricht. Dies ist weniger als die Hälfte des Stromverbrauchs
    eines Routers oder Koordinators.} 
\Abbildungps{!htb}{0.75}{oszi/empfangen_details.jpg}{oszi_empfangen_details}{Detailaufnahme eines Empfangsmodus-Peaks.
    Erkennbar ist die Impulsbreite von 40ms.}
\Abbildungps{!htb}{0.75}{oszi/not_joined.jpg}{oszi_not_joined}{Bei diesem Client wurde der Empfangsmodus deaktiviert. Der 
    Stromverbrauch beträgt konstant 10,7mA.}
\Abbildungps{!htb}{0.75}{oszi/joined.jpg}{oszi_joined}{Beitrittsvorgang eines Clients in ein Netzwerk. Bevor der Client dem 
    Netzwerk beigetreten ist, befindet er sich durchgehend im Empfangsmodus und benötigt 23,4mA. Sobald er dem Netzwerk
    beigetreten ist, entspricht entspricht der Stromverbrauch dem in Abbildung~\ref{oszi_normalbetrieb_client} dargestellten
    Verhalten.} 
\Abbildungps{!htb}{0.75}{oszi/joined2.jpg}{oszi_connection_lost}{Dieser Client hat kurzzeitig die Verbindung zum Netzwerk 
    verloren. In dieser Zeit befand er sich im Dauerempfangsmodus und hat 23,4mA. benötigt.}
\Abbildungps{!htb}{0.75}{oszi/erwachen_1.jpg}{oszi_erwachen}{Dieser Client ist gerade aus dem Energiesparmodus erwacht. 
    Erkennbar ist, dass während der Schlafphase der Stromverbrauch praktisch auf 0 sinkt. Direkt nach dem Erwachen
    wird ein Empfangsvorgang durchgeführt; die danach folgende Phase ist analgo zu Abbildung~\ref{oszi_normalbetrieb_client}.
    Gut zu Erkennen ist, dass der Energiesparmodus praktisch keinen zusätzlichen Overhead bedeutet.}
\Abbildungps{!htb}{0.75}{oszi/powersave_periodic.jpg}{oszi_powersave_periodic}{Dieser Client nutzt das periodische Aufrufen
    des Energiesparmodus. Die Schlafphase hat hier eine Länge von 5 Sekunden, die Wachphase ist genau 1 Sekunde lang. Der
    Stromverbrauch beträgt im Mittel nur noch 1,86mA, was ca. 5,58mW entspricht.}

\section{Reichweite}
