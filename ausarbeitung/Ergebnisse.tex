

%%%------------------------------------------------ERGEBNISSE(KEINE BEWERTUNG)----------------------------------------------------

\chapter{Ergebnisse}
%\section{Baselinewandering}
%\subsection{Messungen unter verschiedenen Bedingungen}
%... nur die reinen Messergebnisse kommen hier rein mit Erläuterung/Begründung etc...
%Bilder von den Messungen und Fakten
%
%\subsection{Vergleichende Messung mit Refenzgerät}
%... Ebenfalls nur Bilder und Zahlen im Vergleich zu der Referenzmethode (am Besten Bilder, in denen die Atmungskurve von der neuen Methode und der Refernzmethode gleichzeitig zu sehen sind)
%
%\section{HRV-Variation}
%\subsection{Messungen unter verschiedenen Bedingungen}
%... wie oben
%\subsection{Vergleichende Messung mit Refenzgerät}
%... wie oben
%
%\section{QRS-Komplexe}
%\subsection{Messungen unter verschiedenen Bedingungen}
%... wie oben
%\subsection{Vergleichende Messung mit Refenzgerät}
%... wie oben
%
%
%\section{Vergleiche der Verfahren zueinander}
%Hier nur Grafiken, Fakten, Zahlen etc. reinmachen, die die verschiedenen Verfahren überlappend zeigen und kurz erläutern, aber nicht bewerten.
%
%\vspace{5cm}
%Der Ergebnisteil (Ergebnisse, results) sollte die wesentlichen Befunde der aktuellen Arbeit in
%nachvollziehbarer, durch geeignete Präsentation (Tabellen, Grafiken) unterstützter Weise darbieten.
%Die Auswahl der dargebotenen Ergebnisse ist nach der Relevanz im Hinblick auf die
%Fragestellung zu treffen. Dies gilt gleichermaßen für Positivergebnisse, welche die Argumentation
%der Autoren stützen, wie auch für Negativergebnisse und Probleme bei der Durchführung
%der Untersuchung, sofern diese einen Einfluss auf das Ergebnis gehabt haben könnten. Die
%Datenpräsentation sollte einen unverfälschten, aber durch geeignete Aufarbeitung der Daten
%(Mittelwertbildung, andere zusammenfassende deskriptive Statistik, etc.) fokussierten Überblick
%geben. Außerdem sollte der Ergebnisteil verschiedene Teilergebnisse nicht isoliert präsentieren,
%sondern den Leser in einer zusammenhängenden Beschreibung durch die Resultate führen.
%Dies schließt eine Beschreibung der wichtigsten Befunde aus Tabellen und Grafiken ein.
%--------
