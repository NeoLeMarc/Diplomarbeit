

%%%------------------------------------------------ERGEBNISSE(KEINE BEWERTUNG)----------------------------------------------------

\chapter{Ergebnisse}
\section{Realisierbarkeit}
\subsection{Integrierbarkeit in Erste-Hilfe-Sensor}
Die \emph{MANVNode} dient neben Ihrer Funktion als Test- und Entwicklungsplatine für die Firmware zur 
Ansteuerung der \emph{ZigBit}-Module als Nachweis der Integrierbarkeit der in dieser Arbeit vorgestellten
Lösung in den \emph{Erste-Hilfe-Sensor}. Hierzu wurde beim Entwurf darauf geachtet, dass neben dem
\emph{ZigBit}-Modul keine weiteren Hardwarebauteile notwendig sind, die nicht auf der Platine des
\emph{Erste-Hilfe-Sensors} vorhanden sind. Darüber hinaus wurde die selbe Mikrocontroller-Architektur
verwendet, die auch im \emph{Erste-Hilfe-Sensor} zum Einsatz kommt, so dass die entsprechenden
Teile der Firmware 1:1 übernommen werden können.

\subsection{Interoperabilität aller Komponenten}
\Abbildungps{!htb}{1}{diagramme/integrationstest_setup.jpg}{intergrationstest_setup}{Aufbau des Integrationstests.}
\Abbildungps{!htb}{1}{diagramme/integrationstest_gui.png}{intergrationstest_gui}{Screenshot der \emph{MANVGui} während des Integrationstests.}
Da \emph{MANVSuite} und \emph{MANVConnector} in zwei verschiedenen Arbeiten (Vgl. \cite{Jan}) 
entworfen wurden, muss die Interoperabilität zwischen allen Komponenten gezeigt werden. Hierzu
wurde ein umfassender Integrationstest durchgeführt.

\subsubsection{Durchführung}
In einem ersten Test wurde eine \emph{MANVNode} über den \emph{MANV-USB-Stick} an den \emph{MANV-Connector}
angeschlossen und die \emph{MANVSuite} ausgeführt. Nun wurde gezeigt, dass sowohl Übertragung von \emph{MANVNode}
zur \emph{MANVSuite} sowie umgekehrt die Übertragung von Befehlen von der \emph{MANVSuite} zu der \emph{MANVNode}
funktionieren.\\
In einem zweiten Test wurde nun obiges Setup durch das Hinzufügen von 4 weiteren \emph{MANVNodes} erweitert. Hierzu
wurden auch wieder Daten in beide Richtungen übertragen. Hierbei wurde insbesondere darauf geachtet, dass sich die
einzelnen \emph{MANVNodes} nicht untereinander stören, sowie dass alle Daten in der \emph{MANVSuite} korrekt zugeordnet
und dargestellt wurden.\\
Obiges Setup wurde nun um 5 \emph{ZigBee-Router}, und alle Tests wurden erneut durchgeführt.\\
Im letzten Schritt wurde nun noch die Skalierbarkeit des Systems überprüft. Hierzu wurde mit Hilfe zweier Notebooks
sowie dem Einsatz von Hardwarevirtualisierung jede Komponente auf eine eigene Virtuelle Maschine installiert. Diese
wurden nun abwechselnd über beide Notebooks verteilt, so dass jede Kommunikation zwangsläufig über das Netzwerk
stattfinden musste. Nun wurden alle obigen Tests noch einmal wiederholt.

\subsubsection{Ergebnisse}
Alle Tests konnten erfolgreich durchgeführt werden. Die Interoperabilität aller Komponenten sowie die Verteilbarkeit
über verschiedene Rechner konnte gezeigt werden.

\subsection{Austauschbarkeit des \emph{MANVConnectors}}
Zur Demonstation der Austauschbarkeit des \emph{MANVConnectors} wurde in \cite{Jan} ein Simulator für den 
\emph{MANVConnector} geschrieben. Dieser Implementiert die selbe \emph{CORBA}-Schnittstelle wie der richtige
\emph{MANVConnector}, führt jedoch keine Kommunikation über das Netzwerk durch. Statt dessen besitzt
der Simulator eine Graphische Benutzeroberfläche, über die zum einen Ereignisse von Hand ausgelöst, zum
anderen Empfangene Befehle in einem Fenster dargestellt werden können. Die Übertragung von Daten wurde
in beide Richtungen erfolgreich getestet, die Austauschbarkeit des \emph{MANVConnectors} damit
erfolgreich demonstriert.

\subsection{Anbindbarkeit an externe Software}
Zentrale Designentscheidung war die Verwendung einer offenen \emph{CORBA}-Schnittstelle zur Anbindung
von externer Software an die \emph{MANVSuite}. Dass dies effizient Durchführbar ist zeigt der folgende
Test.

\subsubsection{Durchführung}
Zum Nachweis der Anbindbarkeit von externer Software über die \emph{CORBA}-Schnittstelle wurde entschieden,
eine Schnittstelle zu einem \emph{Webinterface} zu implementieren. Das \emph{Webinterface} selbst ist
eine einfache \emph{HTML}-Seite, die über \emph{Javascript} eine Graphische Benutzeroberfläche für das
Sensornetzwerk implementiert. Muss nun zunächst mit Daten aus dem Sensornetzwerk versorgt werden. Hierzu
findet das sogenannte \emph{JSON}-Format Anwendung. Ausserdem muss es möglich sein, Befehle an das
Sensornetzwerk zu senden. Dieser werden über spezielle \emph{HTTP-GET}-Anfragen kodiert. Insgesamt wird
diese Vorgehensweise auch als \emph{AJAX} bezeichnet und bildet die Grundlage der \emph{Web-2.0}-Technologie.
Insgesamt musste also ein Adapter von \emph{CORBA} zu \emph{AJAX} realisiert werden. Die Details dieser
Implementierung sind in Anhang~\ref{anhang_beschreibung_software} beschrieben.\\
Darüber hinaus wird in \cite{Jan} beschriebenen, wie ein \emph{Symbian} basiertes \emph{Nokia} Mobiltelefon 
an die \emph{MANVSuite} angebunden wird. Hierzu wurde eine in \emph{C++} programmierte Lösung 
eingesetzt.

\subsubsection{Ergebnisse}
Die Anbindung des \emph{Webinterfaces} an die \emph{MANVSuite} konnte erfolgreich gezeigt werden. Die
Ergebnisse lassen sich ohne weiteres auf andere Schnittstellenformate übertragen. Über die Anbindung
des \emph{Symbian} basierten Mobiltelefons konnte darüber hinaus gezeigt werden, dass eine solche
Anbindung sogar über Plattform- und Programmiersprachengrenzen hinweg realisiert werden kann. 

\section{Leistungsaufnahme des ZigBit-Moduls}
\label{ergebnis_normal}
Zur Bestimmung der Leistungsaufnahme~-- umgangssprachlich auch Strom\-ver\-brauch genannt~-- wurden mehrere Messungen 
durchgeführt.
Da der Stromverbrauch insbesondere bei Einsatz des Energiesparmoduses sehr stark schwankt, sind herkömmliche Messungen
z.B. über ein Multimeter nicht geeignet. Daher wurde eine indirekte Messung über den Spannungsabfall über einem 
Shunt-Wi\-der\-stand durchgeführt. Dieser wurde in die Stromversorgung des ZigBit-Moduls eingebracht. Der Spannungsabfall
über diesem Widerstand wurde nun sowohl mit einem Oszilloskop als auch mit einem Multimeter gemessen. Hierzu wurde zunächst
die Form des Spannungsverlaufes analysiert. Dann wurden einzelne Messpunkte innerhalb des Spannungsverlaufs bestimmt,
für die nun der Wert der abfallenden Spannung bestimmt wurde. Über das Ohmsche-Gesetz konnte nun aus der abfallenden
Spannung die Stromstärke bestimmt werden. Für die einzelnen Messungen wurde ein Oszilloskop der Marke \emph{Tektronix}, 
Modell \emph{TDS2002B} verwendet. Die Messungen wurden zusätzlich mit einem Multimeter der Marke \emph{Fluke} verifiziert,
um Kalibrationsfehler des Oszilloskops auszuschliessen. Als Shunt wurde ein Widerstand mit einem Wert von $20,1\Omega$
verwendet.

\subsection{Leistungsaufnahme im Normalbetrieb}\label{leistungsaufnahme_normalbetrieb}
\Abbildungps{!htb}{0.75}{oszi/normalbetrieb_router.jpg}{oszi_normalbetrieb_router}{Router oder Koordinator im Normalbetrieb. 
    Der Stromverbrauch beträgt hierbei konstant 23,4~mA, was ca. 70~mW entspricht.} 
\Abbildungps{!htb}{0.75}{oszi/normalbetrieb_client.jpg}{oszi_normalbetrieb_client}{Endknotens im Normalbetrieb ohne 
    Energiesparmodus. Auffällig sind die Peaks, in denen sich der Knoten im Empfangsmodus befindet. Die Peaks haben eine 
    Länge von 40ms und einen Betrag von 23,4~mA. Ausserhalb eines Peaks beträgt der Stromverbrauch 10,7~mA. Insgesamt ergibt 
    sich ein Mittelwert von 11,18~mA, was ca. 33,54~mW entspricht. Dies ist weniger als die Hälfte des Stromverbrauchs
    eines Routers oder Koordinators.} 
\Abbildungps{!htb}{1}{diagramme/spannungsverlauf_client_normal.pdf}{spannungsverlauf_client_normal}{Analyse des 
    Spannungsverlaufs eines Clients im Normalbetrieb.}
\Abbildungps{!htb}{0.75}{oszi/empfangen_details.jpg}{oszi_empfangen_details}{Detailaufnahme eines Empfangsmodus-Peaks.
    Erkennbar ist die Impulsbreite von 40ms.}

Als Normalbetriebn wurde der Zustand angenommen, in dem das \emph{ZigBit}-Modul einem \emph{ZigBee}-Netzwerk beigetreten
ist, und der Energiesparmodus nicht zur Anwendung kommt. Bei der Messung wurde schnell klar, dass sich hierbei die
Funktionsweise von eines \emph{FFD} (also Router und Connector) drastisch von der eines \emph{RFD}s (also eines Clients) 
unterscheiden. Ein \emph{FFD} befindet sich immer im Empfangsmodus, d.h. es hat immer eine maximale Leistungsaufnahme. 
Die entsprechende Messung ist in Abbildung~\ref{oszi_normalbetrieb_router} zu erkennen. Hier wurde ein Spannungsabfall
von 473~mV gemessen, was bei dem verwendeten Messwiderstand einem Strom von 23,5mA oder ca. 70mW entspricht.\\
Komplett anders ist hingegen das Verhalten eines Clients. Betrachtet man die Messung in
Abbildung~\ref{oszi_normalbetrieb_client} ist eine Basislinie von 215~mV mit periodischen Peaks erkennbar. 
In Abbildung~\ref{oszi_empfangen_details} ist eine Detailmessung eines solchen Peaks zu sehen.  Gut zu Erkennen ist zum 
einen der Betrag des Spannungsabfalls von rund 473~mV sowie einer Dauer von genau 40ms. Zwischen den Peaks kehrt der 
Spannungabfall für genau eine Sekunde auf den Wert der Basislinie, also 215~mV zurück, auf die wieder ein Peak folgt.
Der Wert von 473~mV legt die Vermutung nahe, dass es sich bei den Peaks um einen periodischen Empfangsvorgang handelt,
d.h. das das \emph{ZigBit}-Modul jede Sekunde für genau 40ms Empfangsbereit ist. Diese Vermutung konnte durch weitere
Tests bestätigt werden. Zur Bestimmung des Stromverbrauchs eines Clients ergibt sich daher folgende Rechnung:

\begin{align}
    U_{Client} &= \frac{t_{Basislinie} \cdot U_{Basislinie} + t_{Peak} \cdot U_{Peak}}{t_{Basisline} + t_{Peak}}
   \label{eqn:UClient}\\
    I_{Client} &= \frac{U_{Client}}{R_{Shunt}}
    \label{eqn:IClient}
\end{align}

Setzt man nun folgende Werte 

\begin{center}
    $t_{Basislinie} = 1000\,\text{ms}$,\quad
    $t_{Peak}       = 40\,\text{ms}$,\quad
    $U_{Basislinie} = 215\,\text{mV}$,\quad
    $U_{Peak}       = 473\,\text{mV}$
\end{center}

in Formel~\ref{eqn:UClient} ein, folgt für den Spannungsabfall

\begin{align*}
    &U_{Client} = \frac{1000\,\text{ms} \cdot 215\,\text{mV} + 40\,\text{ms} \cdot 470\text{mV}}{1000\,\text{ms} + 40,\text{ms}} = 224,9\,\text{mV}
\end{align*}

wodurch sich mit einem Widerstand von 

\begin{center}$
    R_{Shunt} = 20,1\,\Omega
    $
\end{center}

ein Wert von 11,19~mA für $I_{Client}$ ergibt. Dies entspricht bei einer Versorgungsspannung von 3,0~V einer Leistungsaufnahme
von 33,57~mW.

\subsection{Leistungsaufnahme bei Verwendung des Energiesparmodus}
\Abbildungps{!htb}{0.75}{oszi/erwachen_1.jpg}{oszi_erwachen}{Dieser Client ist gerade aus dem Energiesparmodus erwacht. 
    Erkennbar ist, dass während der Schlafphase der Stromverbrauch praktisch auf 0 sinkt. Direkt nach dem Erwachen
    wird ein Empfangsvorgang durchgeführt; die danach folgende Phase ist analgo zu Abbildung~\ref{oszi_normalbetrieb_client}.
    Gut zu Erkennen ist, dass der Energiesparmodus praktisch keinen zusätzlichen Overhead bedeutet.}
\Abbildungps{!htb}{1}{diagramme/spannungsverlauf_client_energiesparmodus.pdf}{spannungsverlauf_client_energiesparmodus}{Analyse des Spannungsverlaufs eines Clients bei Nutzung des Energiesparmodus.}
\Abbildungps{!htb}{0.75}{oszi/powersave_periodic.jpg}{oszi_powersave_periodic}{Dieser Client nutzt das periodische Aufrufen
    des Energiesparmodus. Die Schlafphase hat hier eine Länge von 5 Sekunden, die Wachphase ist genau 1 Sekunde lang. Der
    Stromverbrauch beträgt im Mittel nur noch 1,86mA, was ca. 5,58mW entspricht.}

Die Verwendung des Energiesparmodus~-- im Datenblatt auf als \emph{Sleepmode}, also Schlafmodus, bezeichnet~-- ist eine
Möglichkeit, den Stromverbrauch eines als Client konfigurierten \emph{ZigBit}-Moduls weiter zu reduzieren. Wie in 
Abbildung~\ref{oszi_erwachen} zu sehen ist, sinkt der Stromverbrauch während des Energiesparmodus auf praktisch 
0\footnote{Im Datenblatt ist ein Stromverbrauch von weniger als $6~\mu{}A$ angegeben.}. Sobald das Modul aus dem 
Energiesparmodus in den Normalbetrieb zurück kehrt, wird sofort ein Empfangsvorgang gestartet. Bis zum nächsten Aufrufen des 
Energiesparmoduses verhält sich das Modul wie in Abschnitt~\ref{ergebnis_normal} beschrieben. Der Energiesparmodus kann
entweder manuell\footnote{z.B. mit dem Befehl AT+WSLEEP bei Verwendung der Serialnet Firmware.} oder in einem 
einstellbaren Intervall periodisch aufgerufen werden. Der Spannungsabfall während des periodischen Aufrufens ist in
Abbildung~\ref{oszi_powersave_periodic} zu sehen. Bei der Betriebsart des periodischen Aufrufens wird über zwei Parameter,
$t_{sleep}$ und $t_{awake}$ die Dauer der Schlaf- bzw. Wachphase angegeben. Der Parameter $t_{awake}$ darf hierbei die Dauer
einer Empfangsphase, also 40~ms nicht unterschreiten. Für $t_{sleep}$ gilt eine Mindestdauer von 100ms.\\
Interessant ist nun die Bestimmung des durchschnittlichen Stromverbrauchs bei Verwendung des Energiesparmoduses
in Abhängigkeit von den Parametern $t_{awake}$ und $t_{sleep}$. In erster Näherung kann für den Stromverbrauch im Wachzustand
Gleichung~\ref{eqn:IClient} verwendet werden\footnote{Dies ist eingentlich nicht ganz korrekt: Der Übergang zwischen 
Basislinie und Peak ist nicht stetig. Zur genauen Bestimmung des Stromverbrauchs müssten die einzelnen Intervalle eigentlich
diskret betrachtet werden. Allerdings ist der sich hieraus ergebende Fehler so gering, dass er in der Praxis zu 
vernachlässigen ist.}. Insgesamt ergibt sich der Stromverbrauch zu:

\begin{equation}
   I_{Powersave} = \frac{I_{Client} \cdot t_{awake}}{t_{awake} + t_{sleep}}
\end{equation}

In Tabelle~\ref{werte_Ipowersave} ist der Stromverbrauch für einige exemplarische Werte von $t_{sleep}$ und $t_{awake}$ 
aufgeführt.

\begin{table}
    \begin{center}
        \begin{tabular}{lll}
            \textbf{t\textsubscript{sleep}} & \textbf{t\textsubscript{awake}} & \textbf{I\textsubscript{Powersave}} \\
            1~s  & 40~ms  & 0,45~mA \\
            1~s  & 500~ms & 5,59~mA \\
            1~s  & 1~s    & 5,59~mA \\
            5~s  & 500~ms & 1,02~mA \\
            5~s  & 1~s    & 1,86~mA \\
            10~s & 500~ms & 0,54~mA \\
            10~s & 1~s    & 1,01~mA 
        \end{tabular}
    \end{center}
    \caption{Berechnung von $I_{Powersave}$ für exemplarische Werte von $t_{sleep}$ und $t_{awake}$.}
    \label{werte_Ipowersave}
\end{table}

\subsection{Sonderfälle}\label{leistungsaufnahme_sonderfaelle}
\Abbildungps{!htb}{0.75}{oszi/not_joined.jpg}{oszi_not_joined}{Bei diesem Client wurde der Empfangsmodus deaktiviert. Der 
    Stromverbrauch beträgt konstant 10,7mA.}
\Abbildungps{!htb}{0.75}{oszi/joined.jpg}{oszi_joined}{Beitrittsvorgang eines Clients in ein Netzwerk. Bevor der Client dem 
    Netzwerk beigetreten ist, befindet er sich durchgehend im Empfangsmodus und benötigt 23,4mA. Sobald er dem Netzwerk
    beigetreten ist, entspricht entspricht der Stromverbrauch dem in Abbildung~\ref{oszi_normalbetrieb_client} dargestellten
    Verhalten.} 
\Abbildungps{!htb}{0.75}{oszi/joined2.jpg}{oszi_connection_lost}{Dieser Client hat kurzzeitig die Verbindung zum Netzwerk 
    verloren. In dieser Zeit befand er sich im Dauerempfangsmodus und hat 23,4mA. benötigt.}

So lange ein Client keinem Netzwerk beigetreten ist, befindet er sich Dauerhaft im Empfangsmodus. Das selbe Verhalten liegt
vor, wenn eine bestehende Netzwerkverbindung verloren gegangen ist. Wie in Abbildung~\ref{oszi_connection_lost} zu sehen
ist, steigt der Stromverbrauch in diesem Fall dauerhaft auf 23,4mA an. Dazu kommt, dass so lange keine Verbidnung zu
einem Netzwerk besteht keine Nutzung des Energiesparmmoduses möglich ist. Dies kann eine Steigerung des durchschnittlichen
Stromverbrauchs um über Faktor 10 bedeuten, was eine entsprechend schnelle Entladung der verwendeten Stromquelle bedeuten
kann. Es ist daher wichtig, darauf zu achten, dass alle verwendeten Sensoren möglichst selten die Verbindung zum 
Netzwerk verlieren.


\section{Reichweite}
Zur Bestimmung der Reichweite wurden sowohl innerhalb von Gebäuden als auch im freien Messungen durchgeführt. 
Es wurde schnell klar, dass die Ergebnisse sehr stark von den Gegebenheiten der Umgebung sowie der Höhe von
Sender und Empfänger abhängig sind. Optimale Ergebnisse sind bei direkter Sichtverbindung
sowie einer Höhe von mindestens 1m über dem Boden für Sender und Empfänger erreichen. 
Unterhalb des Senders sollten sich möglichst keine Metalle befinden, als besonders schlecht hat sich die 
Positionierung des Senders auf dem Dach eines PKWs erwiesen. 

\subsection{Reichweite im Freien}
Alle Tests zur Reichweite im Freien wurden auf dem Geländer der Westhochschule der Universität Karlsruhe (TH)
durchgeführt. Hierzu wurde eine \emph{MANVNode} auf einem Pfosten in einer Höhe von einem Meter angebracht,
und auf periodisches Senden bei maximaler Sendeleistung konfiguriert.  Nun wurde ein Notebook mit einem 
\emph{MANV-USB-Stick} so lange von der \emph{MANVNode} entfernt, bis keine Pakete mehr empfangen wurden.
Nun wurde der Abstand wieder so lange verringert, bis wieder 100\% der gesendeten Pakete empfangen wurde.
Ausserdem wurde als Gegenprobe ein Befehl an die \emph{MANVNode} gesendet, und die Antwort empfangen. 
Als maximale Reichweite wurde nun die Entfernung gewertet, bei der gerade noch alle Pakete empfangen, 
sowie alle Befehle erfolgreich gesendet werden konnten. Die Entfernung zwischen Notebook und 
\emph{MANVNode} wurden nun mittels GPS vermessen. Als maximale Reichweite ergaben sich hierbei eine Entfernung
von etwas über 50m.\\
Im nächsten Schritt wurde an die Position des Notebooks ein \emph{ZigBee}-Router aufgestellt, und sich von
diesem wieder so weit entfernt, bis gerade noch alle Pakete empfangen werden konnten. Insgesamt konnte durch
die Verwendung von zwei Routern eine Gesamtreichweite von ca. 180m erreicht werden, so dass sich eine 
durchschnittliche Reichweitenverlängerung von 50-70m pro Router ausgegangen werden kann.\\
Deutliche Einschränkungen bei der Reichweite traten auf, sobald sich Hindernisse wie Container, Häuser oder 
PKWs in der Sichtverbindung zwischen Sender und Empfänger befanden. Diese ließen sich jedoch durch geschicktes
Positionieren der Router sehr gut umgehen, so dass auch hier hohe Reichweiten erzielbar waren. 

\subsection{Reichweite innerhalb geschlossener Räume}
Die Reichweite innerhalb geschlossenere Räume hängt sehr stark von der Beschaffenheit dieser ab. Insbesondere
Metallkonstruktionen wie Treppengeländer und Stahlträger in Wänden wirken sich negativ auf die
Reichweite aus. Eine generische Aussage über die Reichweite innerhalb geschlossener Räume ist daher nur 
sehr schwer zu treffen. Diese hängt meist auch weniger von der Entfernung als von der Anzahl der
Wände zwischen Sender- und Empfänger ab. In diversen Tests hat sich hierbei ergeben, dass eine
Kommunikation durch zwei nicht tragende Wände hindurch gerade noch möglich ist.
