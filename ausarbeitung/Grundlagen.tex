
%%%------------------------------------------------Grundlagen------------------------------------------------------------------
\chapter{Grundlagen}\label{Grundlagen}

\section{Kabellose Übertragungsprotokolle}
    \subsection{Einführung}
    In diesem Abschnitt werden die gängisten Funkprotokolle kurz vorgestellt. Insbesondere wird erläutert,
    inwieweit das entsprechende Protokoll als Grundlage für das zu entwickelnde Sensornetz geeignet ist.

    Mit Ausnahme von DECT und GSM bzw. UMTS ist diesen Protokollen gemein, dass sie sich alle im ISM-Band 
    befinden. 

\subsection{DECT}

Bei DECT ("`Digital Enhanced Cordless Telecommunications"') handelt es sich um einen Standard, 
der vor allem zur Anbindung von Schnurlostelefonen an eine Basisstation gedacht ist\footnote{Es gibt
jedoch auch weitere Anwendungen wie z.B. Babyfone.}. 

In Europa wird einn eigenes Frequenzband im Bereich von 1800 bis 1900 MHz verwendet, in dem 10 
Kanäle zur Verfügung stehen. Pro Kanal können maximal 32kbit Nutzdaten pro Sekunde übertragen
werden. Die maximal zulässige Sendeleistung beträgt 250mW, womit eine Reichweite von ca. 30-50
Metern in Gebäuden und ca. 300m im Freien realisiert werden kann. Jede Basisstation kann
bis zu 6 Geräte anbinden.

Beim Einsatz außerhalb Europas muss bedacht werden, dass die Verwendung der Frequenzen von
1800 bis 1900 MHZ hier evtl. nicht zulässig ist. In diesem Fall muss auf das ISM-Band 
ausgewichen werden, welches sich hier mit anderen Anwendungen geteilt werden mus.

DECT bietet eine optionale Verschlüsselung der Nutzdaten, welche jedoch im Jahr 2009 geknackt
wurde, so dass DECT mittlerweile als unsicher gelten muss.

Aufgrund der geringen Nutzdatenmenge sowie der Einschränkung auf 6 Teilnehmer ist
DECT für den Einsatz als Sensornetz-Protokoll nicht geeignet.

\subsection{GSM/UMTS}


\subsection{WLAN}\label{wlan}
\subsection{Übersicht}
WLAN oder Wi-Fi bezeichnet den heute gängigen Standard eines Funkprotokolls zum Aufbau
von kabellosen lokalen Netzwerken. Es gibt mehrere Versionen des Standards, die verbreitetsten
sind IEEE 802.11a, IEEE 802.11b/g und IEEE 802.11n. 

Es sind zwei Betriebsmodi möglich:

\begin{itemize}
    \item{Infrastruktur-Modus:} Eine zentrale Station ("`Access-Point"') dient als Basisstation
                                für alle weitere Stationen. Jede Station, die am Netzwerk
                                teilnimmt, muss hierzu die Signale des \textsl{Access-Points} 
                                empfangen können.
    \item{Ad-hoc-Modus:} Diese Betriebsmodus kommt ohne zentrale Komponente aus. Es wird eine
                         Peer-to-Peer-Verbindung zwischen allen am Netzwerk teilnehmenden 
                         Stationen aufgebaut. Hierzu ist es notwendig, dass alle Stationen
                         sich gegenseitig empfangen können.
\end{itemize}

Der Infrastruktur-Modus bietet gegenüber des Ad-Hoc-Modus' klare Vorteile: Im Gegensatz
zum Ad-hoc-Modus ist es nicht notwendig, dass alle Stationen sich gegenseitig empfangen
müssen, es reicht aus, wenn der Acces-Point empfangen werden kann. Hierdurch ist eine
größere Ausbreitung des Netzwerks möglich als im Ad-Hoc-Modus. 

Es werden verschiedene Datenübertragungsraten unterstützt. Standardmässig wird
immer die größtmögliche Übertragungsrate gewählt, die Störungsfrei verwendet
werden kann. Je weiter sich die Stationen voneinander entfernen, desto geringer
wird die Übertragungsrate, bis schliesslich die niedrigst mögliche Übertragungsrate von
1MBit/sec erreicht wird.

Es gibt eine ganze Menge von Unterstandards, die sich durch zulässige Frequenzen, 
Übertragungsraten und Sendeleistung unterscheiden. Teilweise ist auch Kanalbündelung 
vorgesehen. Nicht jeder Standard kann in jedem Land eingesetzt werden, und die meisten
Endgeräte unterstützen nur eine Teilmenge dieser Stanrdards. Hier sollen nur die
wichtigsten drei Standards erwähnt werden.

\subsubsection{IEEE 802.11b}
Bei IEEE 802.11b handelt es sich um den ältesten WLAN-Standard, der bereits 1999 
spezifiziert wurde. Dieser Standard wird praktisch von jedem WLAN fähigen 
Endgerät unterstützt. Die Kommunikation findet im ISM-Band im Bereich von 2.4GHz
statt. Je nach Land sind 11-14 Kanäle möglich, die sich jedoch teilweise überlappen.
Dies führt dazu, dass maximal 3 Netzwerke ohne Störungen gleichzeitig betrieben werden
können. Die maximale Sendeleistung beträgt 100mW. Es sind Übertragungsraten von 
5,5 bis 11 MBit (brutto) möglich. Die Nettoübertragungsrate beträgt ca. 50\% der
Bruttorate. Es sind Reichweiten bis 40m (Innen) bzw. 100m (Im Freien) möglich\cite{WirelessNetworking}.

\subsubsection{IEEE 802.11g}
Der IEEE 802.11g Standard stellt eine Erweiterung des IEEE 802.11b Standards da.
Wesentliche Neuerung ist eine Erhöhung der Bruttodatenrate von 11 auf 54MBit/sec,
von denen netto ca. 40\% zur Verfügung stehen. Erwähnenswert ist, dass die
beiden Standards Interoperabel sind, d.h. ein 802.11b Gerät kann einem
802.11g Netzwerk beitreten und umgekehrt. Dies ist auch der Grund, weshalb
dieser Standard momentan am weitesten verbreitet ist.

\subsubsection{IEEE 802.11a}
Der IEEE 802.11a Standard verwendet Frequenzen im 5GHz Bereich. Er ist daher
inkompatibel zum IEEE 802.11b/g Standard. Je nach Frequenzband sind 
Sendeleistungen zwischen 30 und 1000mW zulässig. Mit dem passenden Frequenzband
sind daher höhere Reichweiten als mit dem IEEE 802.11b/g Standard möglich.
Die Bruttodatenrate beträgt bis zu 54MBit/sec. 
Ein Vorteil des IEEE 802.11a Standards ist die Kommunikation im 5GHz Bereich.
Aktuell ist dieser Bereich noch wenig genutzt, so dass in diesem
Bereich oft ein störungsärmerer Betrieb als im 2,4GHz-Bereich möglich ist.
Es ist jedoch zu erwarten, dass sich dies in Zukunft ändern wird.

\subsubsection{Störsicherheit}\ref{WlanStoersicherheit}
IEEE 802.11 verwendet den CSMA/CA\footnote{Carrier Sense Multiple Access with 
Collision Avoidance. Zu deutsch: Gemeinsamer Medienzugriff mit Kollisionsvermeidung}-Algorithmus
zur Störungsbehandlung. Möchte ein Station senden, so muss diese zunächst
für einige Zeit lauschen, ob der zu verwendende Kanal auch wirklich frei ist.
Ist der Kanal belegt, so wartet sie eine zufällige Zeit, bis sie erneut versucht,
auf den Kanal zuzugreifen. Wichtig hierbei ist, dass es bei Funkprotokollen nicht
möglich ist, eine Kollision zu erkennen um eine bestehende Übertragung abzubrechen,
wie es z.B. bei \textsl{Ethernet} der Fall ist. Zur Vermeidung von Kollisionen
kann bei IEEE 802.11 daher zusätzlich zu CSMA/CA eine Art Token-Passing einesetzt
werden. Hier kommt RTS/CTS zum Einsatz: Möchte eine Station ein großes Datenpaket
senden, so sendet diese zuerst ein RTS-Paket\footnote{Ready to send} an den Empfänger, 
welcher dies mit einem CTS-Paket\footnote{Clear to send} quittiert. Erst wenn das
CTS-Paket empfangen wurde, wird mit der Übertragung des eigentlichen Datenpakets
begonnen. Für alle andere Stationen im Netzwerk ist nun klar, dass sie bis zur
abgeschlossenen Übertragung dieses Paketes nicht auf das Netzwerk zugreifen
dürfen.\\
\\
%Quelle: IEEE Standard
Bei dem Einsatz der in IEEE 802.11i definierten Verschlüsselungsverfahren
(WEP\footnote{Die Sicherheit von WEP ist bereits seit einigen Jahren
 kompromitiert. Es sollte nichtmehr verwendet werden}, WPA oder WPA2) wird
lediglich die Nutzdaten des Paketes verschlüsselt. Die CTS/RTS-Pakete können
weiterhin erkannt werden, so dass die Kollisionsverhinderung auch dann
funktioniert, wenn die Pakete nicht entschlüsselt werden können (Diese 
Situation ist z.B. in Mietshäusern oft anzutreffen, wo mehrere 
unterschiedliche WLANs auf dem selben Kanal senden). Jedoch geht dies
mit einer reduzierten Übertragungsrate für die einzelnen Netzwerke
einher.\\
\\
Problematisch ist die Störung durch andere Quellen wie z.B. 
Mikrowellenherden, DECT Telefonen oder Bluetooth, da diese nicht
dem CSMA/CA Verfahren unterliegen und zu beliebiger Zeit senden können.
Insbesondere die Störung durch Bluetooth ist problematisch, da 
Bluetooth und WLAN Geräte oft aufeinadertreffen (z.B. weil an einem
Notebook Bluetooth-Maus und -Tastatur verwendet werden oder weil 
ein PDA z.B. Schnittstellen für beide Protokolle besitzt).
Wie in Abschnitt \ref{bluetooth} genauer erläutert besitzt Bluetooth
79 Kanäle, welche bis zu 1600 mal pro Sekunde gewechselt werden.
Problematisch ist nun, dass 22 dieser Kanäle in das 
IEEE 802.11b/g-Frequenzspektrum fallen. Durch den häufigen
Kanalwechsel, die geringeren Übertragungsraten so die 
Möglichkeit, den wechsel auf belegte Kanäle zu vermeiden ist diese
Störung für Bluetooth deutlich unproblematischer als für Bluetooth.
Je nach Implementierung kann dies zu einer deutlichen Reduktion
der Übertragungsrate des WLANs führen; ausserdem kann die Wartezeit
für das erfolgreiche Senden von Paketen deutlich ansteigen.
Dies hat die IEEE dazu veranlasst, eine eigene Arbeitsgruppe
zu gründen, die sich mit dem Problem der gegenseitigen Störung
von WLAN und Bluetooth zu beschäftigen. Die Ergebnisse dieser
Arbeitsgruppe spiegeln sich im IEEE 802.15.2 Standard wieder.

\subsubsection{Leistungsaufnahme}
Der Energiebedarf für WLAN ist relativ hoch. Beispielsweise benötigt der
vom Hersteller Broadcom als besonders energiesparend bezeichnete Chip
BCM4326 bis zu 100mA zum Empfangen und zwischen 141 und 190mA zum Senden.
\footnote{BCM4326 Datasheet}. 

\subsubsection{Anzahl Teilnehmer}

\subsection{WPAN: Wireless Personal Area Networks}
    \subsubsection{Übersicht}
        Als WPANs ("`Wireless Personal Area Networks"') werden kabellose Kleinnetzwerke bezeichnet, die dazu dienen,
        wenige Geräte über kurze Entfernungen (mehrere Meter) miteinander zu verbinden. Sie dienen als Ersatz 
        von Kabelverbindungen zur Anbindung von Peripherie an Computergeräte (z.B. zur Verbindung von
        Headsets mit Mobiltelefonen oder von Tastatur und Maus mit einem PC).
        

    \subsubsection{IEEE 802.15}
        Der IEEE 802.15-Standard behandelt \textsl{Wireless Personal Area Networks}. Er ist in mehrere Unterstandards
        aufgeteilt:

         \begin{itemize}
            \item{IEEE 802.15.1:} Bluetooth 1.2
            \item{IEEE 802.15.2:} Zusammenarbeit zwischen IEEE 802.15 (WPAN) und IEEE 802.11 (WLAN)
            \item{IEEE 802.15.3:} WPANs mit hohen Datenübertragungsraten (20MBit/sec und höher)
            \item{IEEE 802.15.4:} WPANs mit niedriger Datenübertragungsraten
        \end{itemize}

        Für diese Arbeit sind vor allem der Bluetooth und der ZigBee Standard interessant. 

    \subsubsection{IEEE 802.15.1: Bluetooth}
        \paragraph{Überblick}
            Bluetooth wurde ursprünglich von dem Mobilfunkhersteller Ericcson als Ersatz für
            RS-232 Verbindungen entwickelt. Die Entwicklung des Bluetooth Standards erfolgt
            heute unter der Regie der \textsl{Bluetooth Special Interest Group} ("`SIG"').
            Version 1.1 des Bluetooth Standard wurde 
            von der IEEE als IEEE 802.15.1-2002 übernommen. Nach Veröffentlichen einer
            weiteren Version IEEE 802.15.1-2005, die dem Bluetooth 1.2 Standard entspricht,
            wurde von der IEEE jedoch beschlossen, nicht weiter mit der \textsl{Bluetooth SIG}
            zu kooperiern, so dass es keine weiteren Versionen des IEEE 802.15.1 Standards 
            geben wird. Aktuell ist Version 4.0 des Bluetooth Standard, wobei jedoch die
            meisten Geräte nur geringere Standards (Typischerweise 2.0 oder 2.1) unterstützen.
            Für den Bluetooth 4.0 Standard existiert zum Zeitpunkt dieser Arbeit keine 
            Implementierung auf dem Markt.\\
            Die wichtigsten Meilensteine der Bluetooth-Entwicklung kann folgendermaßen
            zusammengefasst werden:

            \begin{itemize}
                \item{Bluetooth 1.1:} Erste Version von praktischer Relevanz. Entspricht 
                                      IEEE 802.15.1-2002.
                \item{Bluetooth 1.2:} Entspricht IEEE 802.15.2-2005. Bringt einige Verbesserungen
                                      gegenüber der Version 1.1 wie z.B. schnelleres Finden
                                      von Endgeräten (Discovery), höhere Störsicherheit durch
                                      die Verwendung von AFH\footnote{Adaptive frequency-hopping spread spectrum},
                                      Übertragungsraten bis 721kbit/sec.
                \item{Bluetooth 2.0:} Einführung des EDR\footnote{Enhanced Data Rate}-Modus mit bis zu
                                      3.0 MBit/sec (2.1 MBit/sec netto).
                \item{Bluetooth 2.1:} Vereinfachung des Pairings (vgl. \ref{BlueetoothPairing}) durch 
                                      Einführung von SSP\footnote{Secure Simple Pairing}, Verbesserung
                                      der Sicherheit durch explizite Aushandlung der Verschlüsselung.
                \item{Bluetooth 3.0:} Einführung eines Hochgeschwindigkeits-Datenkanals auf Basis von
                                      IEEE 802.11 (vgl. \ref{wlan}) mit bis zu 24MBit/sec., verbessertes
                                      Powermanagement, Einführung von Verbindungslosen Datentelegrammen
                                      (Unicasts).
                \item{Bluetooth 4.0:} Einführung des \textsl{Blueetooth Low Energy} Standards (vgl. \ref{wibree}).
            \end{itemize}

        \paragraph{Pairing}\label{BluetoothPairing}
            Bei der Entwicklung von Bluetooth wurde ein besonderes Augenmerkt auf Datensicherheit gelegt. 
            Dies liegt daran, dass über Bluetooth in vielen Fällen auf sensible Daten (z.B. der Inhalt
            von Mobiltelefonen, Telefongespräche die über Headsets geführt werden etc.) zugegriffen
            werden kann. Bluetooth verwendet hierfür das Konzept des Pairings, also der Paarung. Bevor zwei 
            Bluetooth Geräte miteinander kommunizieren können, müssen sie gepaart werden. Um dies 
            durchzuführen muss zunächst die Identität der zu paarenden Geräte bestätigt werden.
            Hierzu gibt es zwei verschiedene Verfahren:

            \begin{itemize}
                \item{Legacy:} Bis Bluetooth 2.0 muss an beiden Geräten eine identische PIN
                               eingegeben werden. Die PIN ist beliebig und kann bis zu
                               16 Byte lang sein.
                \item{Secure Simple Pairing:} Bluetooth 2.1 definiert neben der PIN-Eingabe
                                              weitere Verfahren zum Paaren von Geräten.
                                              z.B. kann bei dem \textsl{Just-Works}-Verfahren
                                              die PIN komplett ausgelassen werden\footnote{Die
                                              Verbindung erfolgt trotzdem verschlüsselt,
                                              allerdings sind nun Man-in-the-middle-Angriffe 
                                              möglich} oder es wird an beiden Geräten eine
                                              Nummer angezeigt, deren Gleichheit einfach
                                              nur noch bestätigt werden muss.
            \end{itemize}

            Ist diese Überprüfung erfolgreich generieren beide Geräte einen kryptographischen
            Schlüssel und die weitere Kommunikation erfolgt verschlüsselt. Sobald zwei
            Geräte gepaart wurden können sie miteinader kommunizieren ohne eine erneute
            Paarung durchführen zu müssen.

        \paragraph{Reichweite}
            Bluetooth definiert drei verschiedene Klassen von Geräten mit jeweils
            unterschiedlicher Reichweite:

            \begin{itemize}
                \item{Klasse 1:} maximale Sendeleistung: 100mW, Reichweite ca. 100m
                \item{Klasse 2:} maximale Sendeleistung: 2.5mW, Reichweite ca. 10m
                \item{Klasse 3:} maximale Sendeleistung:   1mw, Reichweite ca. 1m
            \end{itemize}

            Diese Einteilung dient unter anderem der Datensicherheit. Da z.B. 
            ein Headset in der Regel nur die Distanz zwischen Kopf und Tasche
            des Anwenders überbrücken muss reicht hier die Verwendung eines Klasse 2
            Gerätes. Durch die Einschränkung der Sendeleistung wird nicht nur die
            Akkulaufzeit der Geräte erhöht, sondern auch die Wahrscheinlichkeit,
            dass ein Angreifer die gesendeten Daten empfangen kann, verringert.\\
            \\
            % Quelle: Def con
            Es bleibt allerdings festzustellen, dass mit Hilfe von geeigneten
            Antennen die Reichweite von Bluetooth siginfikant gesteigert werden
            kann. So ist es z.B. einer Gruppe von Hackern gelungen, mit Hilfe von
            Yaggi-Antennen mit Bluetooth eine Distanz von über 800m zu überbrücken.
        \paragraph{Übertragungsrate}
            Die Übertragungsrate von Bluetooth hängt natürlich von Faktoren wie
            der Verbindungsqualität und der Entfernung ab. Der Standard definiert
            folgende maximale Datenübertragungsraten:

            \begin{itemize}
                \item{Bluetooth 1.1:} 721kbit/sec
                \item{Bluetooth 2.0:} 3.0MBit/sec
                \item{Bluetooth 3.0:} 24 Mbit/sec (über einen 802.11 Kanal)
            \end{itemize}

        \paragraph{Störsicherheit}
            Bluetooth verwendet einen Frequenzbereich von 2,402 - 2,480 GHz. Innerhalb
            dieses Bereiches werden 79 verschiedene Kanäle definiert. Zur Minimierung
            von Störungen wird sogenannts \textsl{Channel-Hopping} verwendet. Hierbei
            wird der verwendete Kanal bis zu 1600 mal pro Sekunde gewechselt. Mit dem
            Bluetooth 1.2 Standard wurde das verbesserte  
            AFH\footnote{Adaptive frequency-hopping spread spectrum}-Verfahren 
            eingeführt, welches gestörte Kanäle erkennt, und eine Verwendung
            dieser vermeidet.\\
            \\
            Insbesondere WLAN-Netzwerke und Bluetooth Netzwerke stören sich 
            gegenseitig. Wie in Abschnitt \ref{WlanStoersicherheit} bereits
            erläutert wird WLAN deutlich stärker durch Bluetooth gestört
            als dies umgekehrt der Fall wäre. Tritt eine Störung auf einem
            Kanal auf, versucht das AFH-Verfahren die Verwendung dieses
            Kanals zu vermeiden. Hierdurch sinkt zwar die Erreichbare
            Datenübertragungsrate, allerdings kann eine Kommunikation mit
            verminderter Übertragungsrate weiterhin stattfinden.\\
             \\
             Es ist festzustellen, dass Bluetooth -- insbesondere im Vergleich
             zu WLANs -- recht robust gegenüber Störungen ist.

        \paragraph{Anzahl Teilnehmer}
            Sobald zwei oder mehr Geräte miteinader verbunden sind, formen diese ein
            sogenanntes \textsl{Piconet}. In einem \textsl{Piconet} können sich bis
            zu 255 Geräte befinden, wobei ein Gerät eine der folgenden beiden Rollen
            hat:

            \begin{itemize}
                \item{Master:} Der Master koordiniert die Kommunikation im Netzwerk.
                               Hierzu gibt er jeweils Zeitslots vor, in denen
                               Daten gesendet werden dürfen. Pro Piconet kann
                               es nur einen Master geben.
                \item{Slaves:} Slaves bekommen vom Master die Erlaubnis, Daten zu senden.
                               Es können immer nur 7 Slaves gleichzeitig aktiv sein.
                               Aktive Slaves müssen permanent empfangsbereit sein,
                               um die Anforderungen des Masters zu empfangen. 
            \end{itemize}

            Da immer nur 7 Slaves gleichzeitig aktiv sein dürfen, befinden sich 
            alle übrigen Slaves im sogenannten \textsl{Parkzustand}. Erst wenn
            ein Slave vom Master explizit dazu aufgeforder wird, darf er in den
            aktiven Zustand wechseln. \\
            \\
            Um die Anzahl der aktiven Geräte in einem Netzwerk zu erhöhen
            gibt es die Möglichkeit, ein sogenanntes Scatternet zu bilden.
            Hierbei handelt es sich um die Verbindung von mehreren Piconets
            mit jeweils maximal 8 Geräten zu einem größeren Verbund. Hierbei
            leitet jeweils ein Gerät, das in jeweils 2 der Piconetze 
            verbunden ist, Pakete vom einen Netz in das andere Netz über.
            Im Vergeleich zu Piconetzen kann hiermit eine deutlich höhere
            Anzahl von Geräten unterstützt werden. Durch die Verkettung 
            der Netzes kann es jedoch vorkommen, dass einzelne Pakete
            eine relativ hohe Anzahl von Piconetzen durchqueren müssen,
            um ihr Ziel zu erreichen.


        \paragraph{Leistungsaufnahme}
    \subsubsection{IEEE-802.15.4: ZigBee}
    \subsubsection{Wibree: Bluetooth Low Energy}\label{wibree}
    \subsection{Weitere Protokolle}
        - WiMAX
        - Mikrowellen-Richtfunk
        - Z-Wave
        - Wireless USB

    \subsection{Diskussion}

\section{Der Analog Devices ADuC702X Mikrocontroller}
\section{Java}
\section{Corba}
