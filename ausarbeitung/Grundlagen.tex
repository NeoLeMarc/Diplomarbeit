
%%%------------------------------------------------Grundlagen------------------------------------------------------------------
\chapter{Grundlagen}\label{Grundlagen}

\section{Kabellose übertragungsprotokolle}
\subsection{Einführung}
In diesem Abschnitt werden die gängisten Funkprotokolle kurz vorgestellt. Insbesondere wird erläutert,
inwieweit das entsprechende Protokoll als Grundlage für das zu entwickelnde Sensornetz geeignet ist.

Mit Ausnahme von DECT und GSM bzw. UMTS ist diesen Protokollen gemein, dass sie sich alle im ISM-Band 
befinden. 

\subsection{DECT}

Bei DECT ("`Digital Enhanced Cordless Telecommunications"') handelt es sich um einen Standard, 
der vor allem zur Anbindung von Schnurlostelefonen an eine Basisstation gedacht ist\textsl{Es gibt
jedoch auch weitere Anwendungen wie z.B. Babyfone.}. 

In Europa wird einn eigenes Frequenzband im Bereich von 1800 bis 1900 MHz verwendet, in dem 10 
Kanäle zur Verfügung stehen. Pro Kanal können maximal 32kbit Nutzdaten pro Sekunde übertragen
werden. Die maximal zulässige Sendeleistung beträgt 250mW, womit eine Reichweite von ca. 30-50
Metern in Gebäuden und ca. 300m im Freien realisiert werden kann. Jede Basisstation kann
bis zu 6 Geräte anbinden.

Beim Einsatz außerhalb Europas muss bedacht werden, dass die Verwendung der Frequenzen von
1800 bis 1900 MHZ hier evtl. nicht zulässig ist. In diesem Fall muss auf das ISM-Band 
ausgewichen werden, welches sich hier mit anderen Anwendungen geteilt werden mus.

DECT bietet eine optionale Verschlüsselung der Nutzdaten, welche jedoch im Jahr 2009 geknackt
wurde, so dass DECT mittlerweile als unsicher gelten muss.

Aufgrund der geringen Nutzdatenmenge sowie der Einschränkung auf 6 Teilnehmer ist
DECT für den Einsatz als Sensornetz-Protokoll nicht geeignet.

\subsection{GSM/UMTS}
\subsection{WLAN}

WLAN oder Wi-Fi bezeichnet den heute gängigen Standard eines Funkprotokolls zum Aufbau
von kabellosen lokalen Netzwerken. Es gibt mehrere Versionen des Standards, die verbreitetsten
sind IEEE 802.11a, IEEE 802.11b/g und IEEE 802.11n. 

Es sind zwei Betriebsmodi möglich:

\begin{itemize}
    \item{Infrastruktur-Modus:} Eine zentrale Station ("`Access-Point"') dient als Basisstation
                                für alle weitere Stationen. Jede Station, die am Netzwerk
                                teilnimmt, muss hierzu die Signale des \textsl{Access-Points} 
                                empfangen können.
    \item{Ad-hoc-Modus:} Diese Betriebsmodus kommt ohne zentrale Komponente aus. Es wird eine
                         Peer-to-Peer-Verbindung zwischen allen am Netzwerk teilnehmenden 
                         Stationen aufgebaut. Hierzu ist es notwendig, dass alle Stationen
                         sich gegenseitig empfangen können.
\end{itemize}

Der Infrastruktur-Modus bietet gegenüber des Ad-Hoc-Modus' klare Vorteile: Im Gegensatz
zum Ad-hoc-Modus ist es nicht notwendig, dass alle Stationen sich gegenseitig empfangen
müssen, es reicht aus, wenn der Acces-Point empfangen werden kann. Hierdurch ist eine
größere Ausbreitung des Netzwerks möglich als im Ad-Hoc-Modus. 

Es werden verschiedene Datenübertragungsraten unterstützt. Standardmässig wird
immer die größtmögliche Übertragungsrate gewählt, die Störungsfrei verwendet
werden kann. Je weiter sich die Stationen voneinander entfernen, desto geringer
wird die Übertragungsrate, bis schliesslich die niedrigst mögliche Übertragungsrate von
1MBit/sec erreicht wird.

Es gibt eine ganze Menge von Unterstandards, die sich durch zulässige Frequenzen, 
Übertragungsraten und Sendeleistung unterscheiden. Teilweise ist auch Kanalbündelung 
vorgesehen. Nicht jeder Standard kann in jedem Land eingesetzt werden, und die meisten
Endgeräte unterstützen nur eine Teilmenge dieser Stanrdards. Hier sollen nur die
wichtigsten drei Standards erwähnt werden.

\subsubsection{IEEE 802.11b}
Bei IEEE 802.11b handelt es sich um den ältesten WLAN-Standard, der bereits 1999 
spezifiziert wurde. Dieser Standard wird praktisch von jedem WLAN fähigen 
Endgerät unterstützt. Die Kommunikation findet im ISM-Band im Bereich von 2.4GHz
statt. Je nach Land sind 11-14 Kanäle möglich, die sich jedoch teilweise überlappen.
Dies führt dazu, dass maximal 3 Netzwerke ohne Störungen gleichzeitig betrieben werden
können. Die maximale Sendeleistung beträgt 100mW. Es sind Übertragungsraten von 
5,5 bis 11 MBit (brutto) möglich. Die Nettoübertragungsrate beträgt ca. 50\% der
Bruttorate. Es sind Reichweiten bis 40m (Innen) bzw. 100m (Außen) möglich.

\subsubsection{IEEE 802.11g}
Der IEEE 802.11g Standard stellt eine Erweiterung des IEEE 802.11b Standards da.
Wesentliche Neuerung ist eine Erhöhung der Bruttodatenrate von 11 auf 54MBit/sec,
von denen netto ca. 40\% zur Verfügung stehen. Erwähnenswert ist, dass die
beiden Standards Interoperabel sind, d.h. ein 802.11b Gerät kann einem
802.11g Netzwerk beitreten und umgekehrt. Dies ist auch der Grund, weshalb
dieser Standard momentan am weitesten verbreitet ist.

\subsubsection{IEEE 802.11a}
Der IEEE 802.11a-Standard verwendet Frequenzen im 5GHz Bereich. Er ist daher
inkompatibel zum IEEE-802.11b/g-Standard. Je nach Frequenzband sind 
Sendeleistungen zwischen 30 und 1000mW zulässig. Mit dem passenden Frequenzband
sind daher höhere Reichweiten als mit dem IEEE-802.11b/g-Standard möglich.
Die Bruttodatenrate beträgt bis zu 54MBit/sec. 
Ein Vorteil des IEEE-802.11a Standards ist die Kommunikation im 5GHz Bereich.
Aktuell ist dieser Bereich noch wenig genutzt, so dass in diesem
Bereich oft ein störungsärmerer Betrieb als im 2,4GHz-Bereich möglich ist.
Es ist jedoch zu erwarten, dass sich dies in Zukunft ändern wird.

\subsubsection{Leistungsaufnahme}
Der Energiebedarf für WLAN ist relativ hoch. Beispielsweise benötigt der
vom Hersteller Broadcom als besonders energiesparend bezeichnete Chip
BCM4326 bis zu 100mA zum Empfangen und zwischen 141 und 190mA zum Senden.
\footnote{BCM4326 Datasheet}. 

\subsubsection{Anzahl Teilnehmer}

\subsection{Bluetooth}
\subsubsection{Reichweite}
\subsubsection{Übertragungsrate}
\subsubsection{Leistungsaufnahme}
\subsubsection{Anzahl Teilnehmer}

\subsection{ZigBee}

\section{Der Analog Devices ADuC702X Mikrocontroller}

\section{Java}

\section{Corba}
