
%%%------------------------------------------------Grundlagen------------------------------------------------------------------
\chapter{Grundlagen}\label{Grundlagen}

\section{Kabellose Übertragungsprotokolle}
\subsection{Einführung}
In diesem Abschnitt werden die gängisten Funkprotokolle kurz vorgestellt. Insbesondere wird erläutert,
inwieweit das entsprechende Protokoll als Grundlage für das zu entwickelnde Sensornetz geeignet ist.

Mit Ausnahme von DECT und GSM bzw. UMTS ist diesen Protokollen gemein, dass sie sich alle im ISM-Band 
befinden. 

\subsection{DECT}

Bei DECT ("`Digital Enhanced Cordless Telecommunications"') handelt es sich um einen Standard, 
der vor allem zur Anbindung von Schnurlostelefonen an eine Basisstation gedacht ist\textsl{Es gibt
jedoch auch weitere Anwendungen wie z.B. Babyfone.}. 

In Europa wird einn eigenes Frequenzband im Bereich von 1800 bis 1900 MHz verwendet, in dem 10 
Kanäle zur Verfügung stehen. Pro Kanal können maximal 32kbit Nutzdaten pro Sekunde übertragen
werden. Die maximal zulässige Sendeleistung beträgt 250mW, womit eine Reichweite von ca. 30-50
Metern in Gebäuden und ca. 300m im Freien realisiert werden können. Jede Basisstation kann
bis zu 6 Geräte anbinden.

Beim Einsatz außerhalb Europas muss bedacht werden, dass die Verwendung der Frequenzen von
1800 bis 1900 MHZ hier evtl. nicht zulässig ist. In diesem Fall muss auf das ISM-Band 
ausgewichen werden, welches sich hier mit anderen Anwendungen geteilt werden mus.

DECT bietet eine optionale Verschlüsselung der Nutzdaten, welche jedoch im Jahr 2009 geknackt
wurde, so dass DECT mittlerweile als unsicher gelten muss.

Aufgrund der geringen Nutzdatenmenge sowie der Einschränkung auf 6 Teilnehmer ist
DECT für den Einsatz als Sensornetz-Protokoll nicht geeignet.

\subsection{GSM/UMTS}
\subsection{WLAN}

WLAN oder Wi-Fi bezeichnet den heute gängigen Standard eines Funkprotokolls zum Aufbau
von kabellosen lokalen Netzwerken. Es gibt mehrere Versionen des Standards, die verbreitetsten
sind IEEE 802.11a, IEEE 802.11b/g und IEEE 802.11n. 

Allen Standards ist gemein...

\subsubsection{IEEE 802.11b}
\subsection{Bluetooth}
\subsection{ZigBee}

\section{Der Analog Devices ADuC702X Mikrocontroller}

\section{Java}

\section{Corba}
