
%%%-----------------------------------------------UMSETZUNG/VORBEREITUNGEN-------------------------------------------------------------
\chapter{Praktische Realisierung des Sensornetzes}

\section{Entwurf}

\subsection{Hardware}
\subsubsection{MANVNode}

\subsubsection{ADuC}
Beim MANVNode handelt es sich um ein Prototyp des sp�teren Erste-Hilfe-Sensor f�r den MANV-Einsatz. Zwar exisitert der 
Erste-Hilfe-Sensor bereits, allerdings hat dieser noch keinerlei Netzwerkf�higkeit. Der Erste-Hilfe-Sensor basiert
auf einem ADuC7019 Microcontroller und erg�nzt diesen durch Detektionskomponeten, zur Patienten�berwachung.

F�r die Entwicklung der Netzwerkanbindung sind diese Detektionskomponenten nur insofern relevant, dass es zu keiner
Gegenseitigen St�rung zwischen Detetion- und Netzwerkkomponeten kommen darf. Daher wurde im ersten Schritt alle nicht
ben�tigten Komponenten weggelassen, und lediglich der reine Mikrocontroller verwendet. Sp�ter wurden die hierbei
entwickelte Netzwerkkomponenten zusammengefasst und in die Hardware des Erste-Hilfe-Sensors integriert.

Die eigentliche Entwicklung fand mit Hilfe eines ADuC7026 Evaluations-Board statt. Dieses Board hat den Vorteil, 
dass alle Anschl�sse des Mikrocontrollers auf Steckerleisten gef�hrt, und damit leicht zug�nglich sind. Ausserdem
ist eine JTAG-Schnittstelle vorhanden, die ein einfaches Debuggen des Mikrocontrollers erm�glicht.

\subsubsection{ZigBee-Schnittstelle}
F�r die Anbindung des Erste-Hilfe-Sensors an das Sensornetz wird ein ZigBit-Modul der Firma Atmel verwendet. 
Dieses Modul bietet den Vorteil, dass es bereits �ber einen kompletten ZigBee-Stack verf�gt, der einfach �ber
AT-Befehle gesteuert werden kann, die per UART gesendet werden.

Der ZigBee-Stack auf dem ZigBit Modul ist austauschbar und kann durch eine eigene Firmware ersetzt werden.
Hierzu wird von der Firma Atmel ein umfangreiches SDK\footnote{Software-Development-Kit: Eine Art Baukasten f�r
Software, die viele ben�tigte Teile bereits fertig zur Verf�gung stellt} angeboten. F�r den Rahmen dieser Diplomarbeit
ist die vorgefertigte Serial-Net-Firmware allerdings ausreichend. Einziger Wermutstropfen ist die fehlende
Verschl�sselung, welche f�r den Serieneinsatz nat�rlich erforderlich w�re.


\subsubsection{MANV-USB-Connector}
Der MANV-USB-Connector ist die Schnittstelle zwischen Sensornetz und Computer. Es handelt sich um einen USB-Stick, der einen
ZigBit-Modul beinhaltet. Zus�tzlich sind zwei weitere Bauteile enthalten, die das ZigBit-Modul mit Strom versorgen, sowie eine
Umsetzung der UART-Schnittstelle des ZigBit-Moduls auf USB vornehmen. F�r die Stromversorgung ist es notwendig, die 5V der
USB-Schnittstelle auf die 3V des ZigBit-Moduls umzusetzen.

\subsection{Firmware}

\subsection{Software}

In dieser Arbeit wurde ein Java-Treiber (MANVConnector) entworfen und implemtiert. Dieser Treiber realisiert die die Anbindung an
die von Herrn Tepelmann in \cite{Jan} entworfene MANVSuite.

\section{Implementierung}

\subsection{Hardware}

\subsection{Firmware}

\subsection{Software}



%\chapter{Praktische Realisierung der Respirationsdetektion}
%\section{Baselinewandering}
%\subsection{Vorbereiten der Hardware}
%\subsubsection{Anforderung an die Hardware}
%... 24 bit, kein HP etc....
%\subsubsection{Realisierung der Anforderungen}
%... hier kann ein Schaltplan eines 24bit-EKG verst�rker rein, oder auch ein Bild von dem Ger�t
%
%\subsection{Datenerfassung}
%... wie kommen die Daten an, wie m�ssen sie konvertiert werden...
%
%\subsection{Signalverarbeitung in Matlab}
%... eben das Matlabprog. mit evtl. Ausz�gen aus dem Quelltext (diese, wenn m�glich in den Anhang verfrachten und einen Hinweis darauf geben)
%
%
%\section{HRV-Variation}
%\subsection{Anforderung an die Hardware}
%... hier gibts nicht viele, und gerade das dann sagen, dass es mit fast jedem Ger�t m�glich w�re...
%
%\subsection{Datenerfassung}
%...
%
%\subsection{Signalverarbeitung in Matlab}
%...
%
%\section{QRS-Komplexe}
%\subsection{Anforderung an die Hardware}
%... wie oben
%
%\subsection{Datenerfassung}
%...
%
%\subsection{Signalverarbeitung in Matlab}
%
%\section{Referenzmessung}
%Hier nur kurze Erl�uterung des Verfahrens
