

%%%------------------------------------------------ERGEBNISSE(KEINE BEWERTUNG)----------------------------------------------------

\chapter{Ergebnisse}
\section{Baselinewandering}
\subsection{Messungen unter verschiedenen Bedingungen}
... nur die reinen Messergebnisse kommen hier rein mit Erl�uterung/Begr�ndung etc...
Bilder von den Messungen und Fakten

\subsection{Vergleichende Messung mit Refenzger�t}
... Ebenfalls nur Bilder und Zahlen im Vergleich zu der Referenzmethode (am Besten Bilder, in denen die Atmungskurve von der neuen Methode und der Refernzmethode gleichzeitig zu sehen sind)

\section{HRV-Variation}
\subsection{Messungen unter verschiedenen Bedingungen}
... wie oben
\subsection{Vergleichende Messung mit Refenzger�t}
... wie oben

\section{QRS-Komplexe}
\subsection{Messungen unter verschiedenen Bedingungen}
... wie oben
\subsection{Vergleichende Messung mit Refenzger�t}
... wie oben


\section{Vergleiche der Verfahren zueinander}
Hier nur Grafiken, Fakten, Zahlen etc. reinmachen, die die verschiedenen Verfahren �berlappend zeigen und kurz erl�utern, aber nicht bewerten.

\vspace{5cm}
Der Ergebnisteil (Ergebnisse, results) sollte die wesentlichen Befunde der aktuellen Arbeit in
nachvollziehbarer, durch geeignete Pr�sentation (Tabellen, Grafiken) unterst�tzter Weise darbieten.
Die Auswahl der dargebotenen Ergebnisse ist nach der Relevanz im Hinblick auf die
Fragestellung zu treffen. Dies gilt gleicherma�en f�r Positivergebnisse, welche die Argumentation
der Autoren st�tzen, wie auch f�r Negativergebnisse und Probleme bei der Durchf�hrung
der Untersuchung, sofern diese einen Einfluss auf das Ergebnis gehabt haben k�nnten. Die
Datenpr�sentation sollte einen unverf�lschten, aber durch geeignete Aufarbeitung der Daten
(Mittelwertbildung, andere zusammenfassende deskriptive Statistik, etc.) fokussierten �berblick
geben. Au�erdem sollte der Ergebnisteil verschiedene Teilergebnisse nicht isoliert pr�sentieren,
sondern den Leser in einer zusammenh�ngenden Beschreibung durch die Resultate f�hren.
Dies schlie�t eine Beschreibung der wichtigsten Befunde aus Tabellen und Grafiken ein.
--------