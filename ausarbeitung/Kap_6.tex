%%%------------------------------------------------DISKUSSION(BEWERTUNG DER ERGEBNISSE)---------------------------------------------

\chapter{Diskussion} 
Bewertung (auch subjektive Meinung) der einzelnen Verfahren. Vor- und Nachteile. Wo gibt es Probleme (z.B. bei HRV nur im unteren Frequenzbereich einsetzbar?? etc.), wie ist die Abweichung zu Referenzmessungen...
\section{Baselinewandering}
...
\section{HRV-Variation}
...
\section{QRS-Komplexe}
...
\section{Vergleich der Verfahren zueinander}

\vspace{2cm}
In der Diskussion (discussion) stellen die Autoren ihre Schlussfolgerungen aus den Ergebnissen
vor. Dabei ist eine Wiederholung der Ergebnisdarstellung zu vermeiden. Das diskutierte
Ergebnis braucht nur noch erw�hnt, nicht aber erneut dargestellt zu werden. Inwieweit konnte
eine in der Einleitung vorgestellte Hypothese gest�tzt oder widerlegt werden? Inwiefern sind
die Ergebnisse in �bereinstimmung mit bisherigen publizierten Befunden und Hypothesen oder
stehen im Gegensatz zu diesen? Neben der Einleitung ist die Diskussion derjenige Teil des
Artikels, in dem ein Schwerpunkt darauf liegt, die gerade ausgef�hrte Studie in die sonstige
Fachliteratur einzuordnen.
