%%%------------------------------------------------VORWORT----------------------------------------------------------------------------
%\thispagestyle{empty}
\chapter*{Vorwort} 

Diese Arbeit entstand in Rahmen des \emph{Erste-Hilfe-Sensor}-Projektes in Kooperation mit Herrn Jan Tepelmann,
der zur selben Zeit eine Studienarbeit geschrieben hat\cite{Jan}. Die Arbeit von Herrn Tepelmann
beschäftigt sich mit dem Entwurf und der Implementierung der Überwachungssoftware des Sensornetzwerkes
(\emph{MANVSuite}), wohingegen sich die hier vorliegende Arbeit mit dem Entwurf der Hardware und der Interaktion der
einzelnen Komponenten des Sensornetzwerkes befasst.\\
\\
Der \emph{Erste-Hilfe-Sensor} verwendet ein neuartiges Verfahren zur Überwachung von Patienten mit Hilfe eines induktiven
Verfahrens. Dies bietet einige Vorteil gegenüber klassischen \emph{EKG} basierten Verfahren. Die eigentliche Funktionsweise
des \emph{Erste-Hilfe-Sensor} ist für die vorliegende Arbeit nicht weiter relevant, lediglich auf den verwendeten
Mikroprozessor muss Rücksicht genommen wurden.
