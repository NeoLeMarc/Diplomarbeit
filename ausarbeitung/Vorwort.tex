%%%------------------------------------------------VORWORT----------------------------------------------------------------------------
%\thispagestyle{empty}
\chapter*{Vorwort} 

Diese Arbeit entstand in Rahmen des \emph{Erste-Hilfe-Sensor}-Projektes in Kooperation mit Herrn Jan Tepelmann,
der zur selben Zeit eine Studienarbeit geschrieben hat \cite{Jan}. Die Arbeit von Herrn Tepelmann
beschäftigt sich mit dem Entwurf und der Implementierung der Überwachungssoftware des Sensornetzwerkes
(\emph{MANVSuite}), wohingegen sich die hier vorliegende Arbeit mit dem Entwurf der Hardware und der Interaktion der
einzelnen Komponenten des Sensornetzwerkes befasst.\\
\\
Der \emph{Erste-Hilfe-Sensor} verwendet ein neuartiges Verfahren zur Überwachung von Patienten mit Hilfe eines induktiven
Verfahrens. Dies bietet einige Vorteil gegenüber klassischen \emph{EKG}-basierten Verfahren. Die eigentliche Funktionsweise
des \emph{Erste-Hilfe-Sensors} ist für die vorliegende Arbeit nicht weiter relevant, lediglich auf den verwendeten
\emph{Mikrocontroller} muss Rücksicht genommen wurden.\\
\\
Bei dieser Arbeit wurde ich von zahlreichen Helfern unterstüzt, bei denen ich mich auf diesem Weg bedanken möchte. 
Die wichtigste Person bei der Durchführung dieser Arbeit war mein Projektpartner Herr cand. inform Jan Tepelmann, der 
auf meine Bitten bereit war, die Implementierung der Überwachungssoftware in Form einer Studienarbeit durchzuführen. Er
war immer ein wertvoller Diskussionpartner, ohne dessen Hilfe es in der gegebenen Zeit nicht möglich gewesen, 
diese Arbeit in der vorliegenden Form anzufertigen.\\
Mein besonderer Dank gilt meinen beiden Gutachtern und meinem Betreuer: Herr Professor Dr. Armin Bolz hat durch seine 
spannende Vorlesung das Interesse an der biomedizinischen Technik erst bei mir geweckt, und sich sehr viel Zeit 
genommen, mich zu dieser interdisziplinären Arbeit zu ermutigen. Herr Professor Dr. Rüdiger Dillmann hat sich bereit 
erklärt, von Seiten der Fakultät für Informatik ein Gutachten durchzuführen -- ohne dies wäre ein Durchführen 
dieser Arbeit gar nicht möglich gewesen. Mein Betreuer, Herr Dr. Marc Jäger hat mir immer wieder mit wertvollen Gesprächen
zur Seite gestanden, und mir viele Tipps zur korrekten Durchführung gegeben. Insbesondere hat er mir als 
Diplomand ein hohes Maß an Vertrauen entgegengebracht, und mir so ermöglicht, das Geplante praktisch umzusetzen. Seine Firma,
die Neocor GmbH hat dabei die Kosten für die notwendige Hardware übernommen.\\
Von unschätzbarem Wert während des Studiums war und ist mein langjähriger guter Freund Herr Juniorprofessor Dr. Christoph
Sorge, der sich nicht nur die Zeit zum Korrekturlesen genommen, sondern mir bereits seit vielen Jahren wertvolle
Tipps und Hilfestellungen zum wissenschaftlichen Arbeiten gegeben hat, und sich immer wieder Zeit für meine Sorgen 
und Nöte nimmt.  Eine weitere sehr wertvolle Hilfe war Herr cand. inform. Alexander Neumann, der vor vielen Jahren 
die Lust auf 
Mikrocontroller bei mir geweckt hat, und mir während der Diplomarbeit ein wichtiger Diskussionspartner für 
Hardwarefragen war. Auch meinem Freund Herrn Dipl.-Inform. (FH) Markus Müller möchte ich für das Korrekturlesen danken.
Sehr herzlich wurde ich von den Mitarbeitern des IBT aufgenommen. Hierbei bedanke ich mich insbesondere bei Herrn
M.-Eng Daniel Wettach, der mir bei Fragen zum \emph{ADuC-Microcontroller} immer weitergeholfen hat, sowie Herrn
Dipl.-Ing Stefan Fernsner für die Hilfe beim Erstellen der \emph{MANVNode}-Platinen. Auch bedanken möchte ich mich bei 
Frau Dipl.-Phys. Kerstin Grimmel, Herrn Dipl.-Ing Nikolas 
Lentz und Herrn M. Sc. Firas Salih für zahlreiche anregende und hilfreiche Diskussionen.\\
Von ganzem Herzen bedanken möchte ich mich bei meiner Lebenspartnerin Frau Dipl.-Phys. Jennifer Girrbach sowie meinen Eltern,
meinen Geschwistern und meiner Familie und Freunden für den Rückhalt den sie mir in dieser anstrengenden Zeit gegeben
haben, und den Mut, den sie mir zusprachen.
