% $Header: /cvsroot/latex-beamer/latex-beamer/examples/beamerexample1.tex,v 1.47 2004/11/04 15:43:51 tantau Exp $

\documentclass{beamer}
%\documentclass{article}
%\usepackage[envcountsect]{beamerarticle}

% Do NOT take this file as a template for your own talks. Use a file
% in the directory solutions instead. They are much better suited.

% Try the class options [notes], [notes=only], [trans], [handout],
% [red], [compress], [draft] and see what happens!

% Copyright 2003 by Till Tantau <tantau@users.sourceforge.net>.
%
% This program can be redistributed and/or modified under the terms
% of the LaTeX Project Public License Distributed from CTAN
% archives in directory macros/latex/base/lppl.txt.

% For a green structure color use:
%\colorlet{structure}{green!50!black}

\usepackage{array,paralist}
\usepackage{tabularx}

\mode<article> % only for the article version
{
  \usepackage{fullpage}
  \usepackage{hyperref}
}


\mode<presentation>
{
  \setbeamertemplate{background canvas}[vertical shading][bottom=red!10,top=blue!10]

  \usetheme{Warsaw}
  \usefonttheme[onlysmall]{structurebold}
}

%\setbeamercolor{math text}{fg=green!50!black}
%\setbeamercolor{normal text in math text}{parent=math text}

\usepackage{amsmath,amssymb}
\usepackage[latin1]{inputenc}
\usepackage{colortbl}
\usepackage[english]{babel}

%\usepackage{lmodern}
%\usepackage[T1]{fontenc} 

\usepackage{times}
\setbeamercovered{dynamic}

%% Titel etc
\title[Kabellose Sensornetzwerke f�r MANV-Einsatz]{Entwurf und Implementierung
       eines kabellosen Sensornetzwerkes zur �berwachung von Patienten
       in einem MANV-Szenario}
\author[Noe, Tepelmann]{Marcel Noe \and Jan Tepelmann}
\institute[Karlsruher Institut f�r Technologie]{
        \inst{} Institut f�r Biomedizinische Technik (IBT) \\
                Karlsruher Institut f�r Technologie } 
\date[2010]


%% Hauptseite
\begin{document}
\frame{\titlepage}

\section<presentation>*{Einleitung}

\begin{frame}
  \frametitle{�berblick}
  \tableofcontents[part=1,pausesections]
\end{frame}

\part<presentation>{Vortrag}


\section{Motivation}

\begin{frame}
    \frametitle{Motivation}

    \begin{itemize}
        \item Item
    \end{itemize}

    \begin{center}
    \end{center}
\end{frame}


\section{Analyse}
\subsection{Anforderungen}

\begin{frame}
    \frametitle{Anforderungsanalyse}

    \begin{itemize}
        \item Foo
    \end{itemize}
\end{frame}


\subsection{Stand der Technik}

\begin{frame}
    \frametitle{Stand der Technik}

    \begin{itemize}
        \item Item
    \end{itemize}

\end{frame}


\section{L�sung}
\begin{frame}
    \frametitle{�berblick}

    \begin{itemize}
        \item item
    \end{itemize}
\end{frame}

\begin{frame}
    \frametitle{Features}

    \begin{itemize}
        \item Item
    \end{itemize}
\end{frame}

\section{Praktische Vorf�hrung}

\section{Fazit}

\begin{frame}
    \frametitle{Verbesserungen gegen�ber dem Stand der Technik}
    \begin{itemize}
        \item Item
    \end{itemize}
\end{frame}

\end{document}
